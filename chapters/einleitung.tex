\chapter{Einleitung}
In den letzten Jahren hat sich die digitale Kommunikation signifikant gewandelt. 2013 war das Short Message System (SMS) 
noch dominierend, doch in den darauf folgenden Jahren verlagerte sich der Fokus deutlich hin zum Mobile Instant Messaging. 
Die Nutzung von Instant Messaging unter der deutschen Bevölkerung stieg von etwa 24\% im Jahr 2013 auf beeindruckende 73\% 
im Jahr 2017 \parencite{digiKommunikationVeraendert}. \\
In der heutigen digitalen Welt, geprägt von ständiger Vernetzung und dem Bedarf an schnellem und sicheren 
Informationsaustausch, stehen sicheres Instant Messaging und die Blockchain-Technologie im Zentrum eines tiefgreifenden 
Wandels. Während Instant Messaging die Art und Weise revolutioniert hat, wie Menschen miteinander kommunizieren, indem 
es Echtzeitkommunikation über Text, Bilder und Videos ermöglicht, hat die Blockchain-Technologie in den Bereichen Finanzen, 
Datensicherheit und digitale Transaktionen eine disruptive Veränderung eingeleitet. Interessanterweise sind diese beiden 
Konzepte keineswegs voneinander isoliert, sondern vielmehr eng miteinander verknüpft.

Instant Messaging-Plattformen sehen sich mit der Herausforderung konfrontiert, die Privatsphäre und Sicherheit 
ihrer Nutzer zu gewährleisten, insbesondere angesichts wachsender Bedenken hinsichtlich Datenschutz und Sicherheit. 
Hier kommt die Blockchain ins Spiel. Ihre dezentrale Natur und die Fähigkeit, Transaktionen und Daten in einer 
fälschungssicheren Umgebung zu speichern, bietet eine vielversprechende Lösung. Blockchain kann dazu beitragen, 
die Vertraulichkeit von Instant Messaging-Nachrichten und die Identität der Nutzer zu schützen, da sie eine 
manipulationssichere Aufzeichnung aller Transaktionen ermöglicht.

Darüber hinaus hat die Blockchain das Potenzial, die Integrität und Authentizität von Dateien und Dokumenten in 
Instant Messaging-Plattformen sicherzustellen. Dies ist von entscheidender Bedeutung, insbesondere in geschäftlichen 
Kontexten, in denen Verträge und wichtige Informationen ausgetauscht werden. Die Integration von Blockchain in 
Instant Messaging-Plattformen verspricht somit eine sicherere und vertrauenswürdigere Kommunikation.

In dieser Wechselwirkung zwischen sicheren Instant Messaging-Plattformen und Blockchain-Technologie zeigt sich, 
wie diese beiden Konzepte gemeinsam dazu beitragen, die Sicherheit und Integrität von digitalen Kommunikationsprozessen 
zu gewährleisten. Es ist ein aufregendes Zusammenspiel, das die Art und Weise, wie wir kommunizieren und Geschäfte tätigen, 
nachhaltig beeinflusst und in Zukunft noch weiterentwickelt werden könnte.
\\
\\
Sicheres Instant Messaging und Blockchain sind zwei Schlüsselkonzepte, die in der heutigen digitalen Ära eine herausragende 
Rolle spielen. Während Instant Messaging die Art und Weise verändert hat, wie Menschen in Echtzeit miteinander kommunizieren, 
hat Blockchain eine Revolution in der Art und Weise ausgelöst, wie Transaktionen und Datenmanagement in einer dezentralen, 
sicheren Umgebung stattfinden. Diese beiden Technologien sind in vielerlei Hinsicht miteinander verknüpft, da sie gemeinsam 
das Potenzial bieten, die Sicherheit, Integrität und Transparenz in der digitalen Kommunikation und im Datenaustausch zu 
gewährleisten. In diesem Zusammenhang ist es entscheidend, das Zusammenspiel von sicheren Instant-Messaging-Plattformen 
und der Blockchain-Technologie zu betrachten, um ein umfassendes Verständnis davon zu erlangen, wie sie unsere moderne 
Kommunikationslandschaft und Geschäftswelt prägen