\chapter{Schlussfolgerung und Ausblick}

In dieser Arbeit wurde ein Peer-to-Peer-Instant-Messaging-Protokoll entwickelt, das einen Schwerpunkt auf Sicherheit legt. Die Sicherheit wird durch die Verwendung von Ende-zu-Ende-Verschlüsselung und der Integration der Ethereum-Blockchain gewährleistet. Die Verwendung von Kademlia als Routing-Algorithmus ermöglicht eine schnelle Suche nach Teilnehmern und die Verwendung von ICE ermöglicht die Kommunikation zwischen Teilnehmern, die sich hinter NAT befinden.

Diese Arbeit hatte nicht den Anspruch eine vollständige Definition eines Protokolls zu liefern, sondern sollte einen Prototypen entwickeln, der die Machbarkeit eines solchen Protokolls aufzeigt. Deshalb wurden nicht alle Aspekte mit der eigentlich nötigen Tiefe beleuchtet. 

Es hat sich herausgestellt, dass die Verwendung von Peer-to-Peer-Mechanismen in Instant-Messaging-Protokollen durchaus sinnvoll, aber sehr komplex ist. Es müssen viele Aspekte beachtet werden, die in zentralisierten Protokollen nicht vorhanden sind, wie beispielsweise der Umgang mit NAT.

Die Entwicklung einer Ende-zu-Ende-Verschlüsselung war ebenfalls nicht trivial. Es musste ein geeignetes Verfahren gefunden werden, das die Anforderungen an die Sicherheit erfüllt und gleichzeitig eine gute Performance aufweist. Die Auswahl und Parametrisierung der verwendeten Algorithmen war sehr herausfordernd, da es viele verschiedene Möglichkeiten gibt. 

Die Integration der Blockchain ist jedoch eine sinnvolle Maßnahme, um die Sicherheit des Protokolls zu erhöhen, da keiner zentralen Instanz vertraut werden muss. Dies könnte die Akzeptanz des Protokolls erhöhen.