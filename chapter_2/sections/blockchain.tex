\section{Blockchain-Technologie}
\textbf{\textcolor{red}{Ralph Merkle!}}
Die Blockchain-Technologie ist... . Blockchains sind auch bekannt als Distributed Ledger 
Technology (DLT) und das sind auch nur verteilte Datenbanken. Die Blockchain ist eine
spezielle Form der DLT. Die Blockchain ist eine Kette von Blöcken, die jeweils einen
Hash des vorherigen Blocks enthalten. Die Blockchain ist eine dezentrale Datenbank, die
von allen Teilnehmern des Netzwerks verwaltet wird.

Proof of Work und Proof of Stake sind zwei verschiedene Konsensmechanismen, die verwendet
werden, um die Blockchain zu validieren. Proof of Work ist der Konsensmechanismus, der
von Bitcoin verwendet wird. Proof of Stake ist der Konsensmechanismus, der von Ethereum (2.0)
verwendet wird.

Der bekannteste Anwendungsfall für Blockchain ist Bitcoin. Bitcoin ist eine Kryptowährung,
die auf der Blockchain-Technologie basiert. Bitcoin wurde 2008 von Satoshi Nakamoto entwickelt
und ist seitdem stetig gewachsen. Bitcoin ist eine dezentrale Kryptowährung, die von niemandem
kontrolliert wird.

Eine weitere bekannte Anwendung für Blockchain ist Ethereum. Ethereum ist eine dezentrale
Plattform, die es ermöglicht, Smart Contracts zu erstellen und auszuführen. Smart Contracts
sind Programme, die auf der Blockchain ausgeführt werden und die Möglichkeit bieten, Transaktionen und
Verträge zu erstellen, die automatisch ausgeführt werden, sobald bestimmte Bedingungen erfüllt sind.
Ethereum ist eine der bekanntesten und am weitesten verbreiteten Blockchain-Plattformen. Die Plattform
wurde 2015 von Vitalik Buterin entwickelt und ist seitdem stetig gewachsen.


Auf der Ethereum-Blockchain wird hauptsächlich Programmiersprache Solidity verwendet, um Smart Contracts zu erstellen. Solidity ist eine objektorientierte Programmiersprache, die stark an JavaScript angelehnt ist \parencite[S. 131]{Antonopoulos_MasteringEthereum}. Die Smart Contracts, die für das Protokoll benötigt werden, sind in Solidity implementiert. Die Smart Contracts werden auf der Ethereum-Blockchain ausgeführt und können von jedem Teilnehmer des Netzwerks aufgerufen werden. Die Smart Contracts werden in Abschnitt \ref{subsec:registrierung} (\nameref{subsec:registrierung}) und Abschnitt \ref{subsec:kommunikation} (\nameref{subsec:kommunikation}) beschrieben.

\begin{itemize}
    \item Warum ist Blockchain sicher?
    \item Angriffe auf Blockchain (Sybil, 51\%, etc.)
\end{itemize}

\subsection{Ethereum}
Ethereum 1.0 und 2.0
\subsection{Smart Contracts}
Smart Contracts sind Programme, die auf derBlockchain ausgeführt werden. Smart Contracts 
sind in der Programmiersprache Solidity geschrieben. Smart Contracts sind in der Blockchain
gespeichert und werden von allen Teilnehmern des Netzwerks ausgeführt. Smart Contracts sind
nicht veränderbar und können nicht gelöscht werden. Smart Contracts können nur durch andere
Smart Contracts oder durch Transaktionen aufgerufen werden. Smart Contracts können Daten
in der Blockchain speichern. Smart Contracts können auch Daten aus der Blockchain lesen.
Smart Contracts können auch Transaktionen auslösen. Smart Contracts können auch andere
Smart Contracts aufrufen.‚