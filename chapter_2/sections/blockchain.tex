\section{Blockchain-Technologie}
\label{sec:blockchain_basics}

%Als Grundlage für Blockchains dient die \textit{Distributed Ledger Technology} (DLT). Was alle DLTs gemeinsam haben, ist, dass die Daten auf allen Teilnehmern des Netzwerkes gespeichert werden und die Verwendung von kryptografischen Technologien, wie Hash-Funktionen und Konsensmechanismen \parencite{Ioini_DLTReview}.

%Bei der DLT handelt es sich um eine Technologie, die es ermöglicht, Daten in einem dezentralen Netzwerk zu speichern. Die Daten werden dabei auf allen Teilnehmern des Netzwerkes gespeichert. Dadurch ist es nicht möglich, die Daten zu manipulieren, da die Daten auf allen Teilnehmern des Netzwerkes gespeichert sind. Wenn ein Teilnehmer versucht, die Daten zu manipulieren, wird dies von den anderen Teilnehmern bemerkt und die manipulierten Daten werden nicht akzeptiert \parencite[S. 10]{Fill_BlockchainGrundlagen}.

Christoph Meinel und Tatiana Gayvoronskaya beschreiben die Blockchain-Technologie in ihrem Buch \textit{Blockchain - Hype oder Innovation?} wie folgt:

\begin{quote}
    \textit{Die Innovation der Blockchain-Technologie ist weder ein neuer Verschlüsselungsalgorithmus
    noch eine „Alientechnologie“, sondern eine erfolgreiche Kombination bereits
    vorhandener technologischen Ansätze wie Kryptografie, dezentrale Netzwerke und Konsensfindungsmodelle.}\parencite[S. 17]{Meinel_BlockchainHypeInnovation}
\end{quote}

\noindent Daraus ergibt sich die Frage, was Blockchain eigentlich ist und wie die bereits vorhandenen technologischen Ansätze miteinander vereint werden. Diese Fragen werden in diesem Kapitel beantwortet. Zunächst wird der Begriff Blockchain definiert und anschließend die Funktionsweise erläutert. Daraufhin werden die Konsensmechanismen\textit{Proof-of-Work} und \textit{Proof-of-Stake} vorgestellt. Abschließend wird auf die Sicherheit von Blockchain eingegangen und Angriffe auf Blockchain werden erläutert.
% #TODO: Distributed Ledger als Vorgänger zu Blockchain erwähnen?

\subsection{Definition von Blockchain}
\label{subsec:blockchain_definition}

Der Begriff \textit{Blockchain} setzt sich aus den englischen Wörtern \textit{block} (Block) und \textit{chain} (Kette) zusammen. Eine Blockchain ist also eine Kette von Blöcken. Wie in Abbildung \ref{fig:block} zu sehen ist, beinhaltet ein Block den Block-Header und mehrere Transaktionen. 

\begin{figure}[H]
    \centering
    \includegraphics[width=0.4\linewidth]{images/blockchain_block.png}
    \caption{Aufbau eines Blocks \parencite[S. 11]{Fill_BlockchainGrundlagen}}
    \label{fig:block}
\end{figure}


\noindent Der Block-Header enthält Informationen über den Block, wie zum Beispiel den Hash des vorherigen Blocks, die Schwierigkeit, den Zeitstempel und die Nonce. Die Schwierigkeit wird durch das \textit{Target}-Feld angegeben und beschreibt, wie schwer es ist, das kryptografische Puzzle zu lösen und damit auch wie schwer es ist, einen neuen Block an die Blockchain anzuhängen. Die Suche nach der Lösung dieses Puzzles wird auch als \textit{Mining} bezeichnet. Der erste, der das Puzzle löst, präsentiert als Beweis dieser Lösung die sogenannte \textit{Nonce}. Nachdem diese neue Version der Blockchain (mit dem neu angehängten Block) an alle anderen Teilnehmer verteilt wurde, kann jeder Teilnehmer überprüfen, ob die Lösung korrekt ist. Die Nonce ist eine zufällige Zahl, die bei der Lösung des Puzzles verwendet wird und aufwändig zu berechnen ist. Durch den Verweis im Block-Header auf den vorherigen Block entsteht eine Kette von Blöcken (siehe Abbildung \ref{fig:chain_of_blocks}), die \textit{Blockchain} genannt wird \parencite[S. 10-12]{Fill_BlockchainGrundlagen}.

\begin{figure}[H]
    \centering
    \includegraphics[width=0.9\linewidth]{images/chain_of_blocks.png}
    \caption{Kette aus Blöcken \parencite[S. 12]{Fill_BlockchainGrundlagen}}
    \label{fig:chain_of_blocks}
\end{figure}


\subsection{Kryptografische Grundlagen}
\label{subsec:cryptography_basics}

Aus der Kryptografie werden Hash-Funktionen, kryptografische Puzzles, Hash-Bäume und digitale Signaturen verwendet. Hash-Funktionen können dazu verwendet werden, um die Integrität von Daten zu gewährleisten. Eine Hash-Funktion bildet eine beliebig lange Eingabe auf eine feste Länge ab. Kryptografische Hash-Funktionen sind Hash-Funktionen, die bestimmte Eigenschaften erfüllen müssen. Die beiden wichtigsten Eigenschaften sind die \textit{Einwegfunktion} und die \textit{Kollisionsresistenz}. Eine Hash-Funktion erfüllt die Eigenschaft der Einwegfunktion, wenn es nicht möglich ist, von der Ausgabe auf die Eingabe zu schließen. Das bedeutet, dass es nicht möglich ist, aus dem Hash-Wert die ursprünglichen Daten zu rekonstruieren. Die Eigenschaft der \textit{Kollisionsresistenz} ist erfüllt, wenn es nicht möglich ist, zwei verschiedene Eingaben zu finden, die auf den gleichen Hash-Wert abgebildet werden \parencites[S. 12-13]{Brünnler_BlockchainKurzGut}[S. 6]{Fill_BlockchainGrundlagen}.

Kryptografische Puzzles werden dazu verwendet, um die Schwierigkeit beim Mining zu erhöhen. Dabei soll mittels Hash-Funktionen ein bestimmter Ausgabewert gefunden werden. Laut Definition der Einwegfunktion ist es nicht möglich, von der Ausgabe auf die Eingabe zu schließen. Daher kann die gesuchte Eingabe nur durch Ausprobieren gefunden werden. Die Schwierigkeit kann durch die Anzahl der Nullen, die am Anfang des Hash-Wertes stehen müssen, angegeben werden. Je mehr Nullen am Anfang des Hash-Wertes stehen müssen, desto schwieriger ist es, die Lösung zu finden, da dadurch der Lösungsraum verkleinert wird \parencites[S. 6-7]{Fill_BlockchainGrundlagen}[S. 320]{Antonopoulos_MasteringEthereum}.

Ein \textit{Merkle-Baum} ist ein binärer Baum, bei dem jeder Knoten den Hash-Wert seiner Kinder enthält. Der nach seinem Erfinder Ralph Merkle benannten Hash-Baum wird in der Blockchain zum Aufbau von Datenstrukturen verwendet. Der Hash-Wert der Wurzel des Baumes wird auch als \textit{Root-Hash} oder \textit{Merkle-Root} bezeichnet (siehe Abbildung \ref{fig:block}: \textit{Root-Hash}). Wenn sich ein Blatt des Baumes ändert, ändert sich auch der Hash-Wert der Wurzel. Dadurch kann überprüft werden, ob sich die Daten geändert haben. Wenn sich die Daten geändert haben, ändert sich auch der Root-Hash. Wenn sich die Daten nicht geändert haben, bleibt der Root-Hash gleich. Dadurch kann die Integrität der Daten überprüft werden \parencite[S. 7-8]{Fill_BlockchainGrundlagen}. In Abbildung is der Aufbau eines Merkle-Baumes zu sehen, der aus vier Transaktionen besteht.

\begin{figure}[H]
    \centering
    \includegraphics[width=0.9\linewidth]{images/merkle_tree_modified.png}
    \caption{Aufbau eines Merkle-Baumes (in Anlehnung an \cite[S. 8]{Fill_BlockchainGrundlagen})}
    \label{fig:merkle_tree}
\end{figure}

\noindent Von jeder Transaktion $T_{1}$, $T_{2}$, $T_{3}$ und $T_{4}$ wird ein Hash-Wert berechnet und in einem Blatt des Baumes gespeichert. Die Hash-Werte der Blätter werden dann paarweise gehasht und in den Knoten darüber gespeichert. Dieser Vorgang wird solange wiederholt, bis nur noch ein Knoten übrig ist. Dieser Knoten enthält den Root-Hash. Sollte sich eine Transaktion ändern, ändert sich auch der Root-Hash. Außerdem kann bewiesen werden, dass beispielsweise $T_{2}$ Teil des Baumes ist, indem mit dem Hash-Wert von $T_{2}$ und den Hash-Werten von $T_{1}$, $T_{3}$ und $T_{4}$ versucht wird, den Root-Hash zu berechnen. Wenn der berechnete Root-Hash mit dem tatsächlichen Root-Hash übereinstimmt, ist bewiesen, dass $T_{2}$ Teil des Baumes ist \parencite[S. 9]{Fill_BlockchainGrundlagen}.

Die folgende Abbildung zeigt das Zusammenspiel von Blöcken, Transaktionen und Merkle-Bäumen:

\begin{figure}[H]
    \centering
    \includegraphics[width=1\linewidth]{images/chain_of_blocks_with_merkle_tree.png}
    \caption{Verkettete Blöcke mit Transaktionen im Block-Body \parencite[S. 95035]{Nam_51PercentAttacks}}
    \label{fig:chain_of_blocks_with_merkle_tree}
\end{figure}


\noindent Eine weitere Technologie aus der Kryptografie, die in der Blockchain verwendet wird, sind digitale Signaturen. Digitale Signaturen werden verwendet, um die Authentizität von Daten zu gewährleisten. Wenn zwei Benutzer Alice und Bob miteinander kommunizieren wollen, kann Alice eine Nachricht an Bob senden. Jeder Benutzer hat einen privaten und einen öffentlichen Schlüssel. Der öffentliche Schlüssel darf beliebig verteilt werden, während der private Schlüssel geheim gehalten werden muss. Wenn Alice eine Nachricht an Bob senden möchte, kann sie die Nachricht mit ihrem privaten Schlüssel signieren. Bob kann dann die Signatur mit dem öffentlichen Schlüssel von Alice überprüfen. Wenn die Signatur gültig ist, kann Bob sicher sein, dass die Nachricht von Alice stammt. Wenn die Signatur wiederum ungültig ist, kann Bob sicher sein, dass die Nachricht nicht von Alice stammt \parencite[S. 9-10]{Fill_BlockchainGrundlagen}.


\subsection{Dezentrale Netzwerke}

Blockchains basieren auf einem Peer-to-Peer-Netzwerk. Ein Peer-to-Peer-Netzwerk ist ein Netzwerk, bei dem alle Teilnehmer gleichberechtigt sind. Es gibt keinen zentralen Server, der die Daten verwaltet. Stattdessen werden die Daten auf allen Teilnehmern des Netzwerkes gespeichert. Wenn ein Teilnehmer Daten an das Netzwerk senden möchte, sendet er die Daten an alle anderen Teilnehmer des Netzwerkes. Wenn ein Teilnehmer Daten vom Netzwerk empfangen möchte, empfängt er die Daten von allen anderen Teilnehmern des Netzwerkes. Dadurch ist es nicht möglich, das Netzwerk zu manipulieren, da die Daten auf allen Teilnehmern des Netzwerkes gespeichert sind. Wenn ein Teilnehmer versucht, die Daten zu manipulieren, wird dies von den anderen Teilnehmern bemerkt und die manipulierten Daten werden nicht akzeptiert \parencite[S. 10/31]{Fill_BlockchainGrundlagen}.

\subsection{Konsensmechanismen}

In einem dezentralen Netzwerk müssen sich die Teilnehmer auf einen gemeinsamen, für alle \textit{richtigen}, Zustand einigen. Dieser Zustand kann beispielsweise die Reihenfolge der getätigten Transaktionen sein. Wenn sich die Teilnehmer nicht auf einen gemeinsamen \textit{Konsens} einigen können, kann das Netzwerk nicht funktionieren. Die Aufgabe der Konsensmechanismen ist es also, einen gemeinsamen Zustand zu finden. Die beiden bekanntesten Konsensmechanismen sind \textit{Proof-of-Work} und \textit{Proof-of-Stake} \textcolor{red}{(QUELLE)}.

\subsubsection{Proof-of-Work vs. Proof-of-Stake}

Der \textit{Proof-of-Work} (PoW) ist der Konsensmechanismus, der in der Bitcoin-Blockchain verwendet wird. Bitcoin wurde 2008 von Satoshi Nakamoto in einem Whitepaper vorgestellt und ist die erste dezentrale Kryptowährung \parencites{Nakamoto_Bitcoin}{Zhang_DoubleSpendingWithASybilAttack}. Der PoW besteht aus dem bereits in Abschnitt \ref{subsec:blockchain_definition} \nameref{subsec:blockchain_definition} beschriebenen kryptografischen Puzzle. Das Lösen des Puzzles durch einen beliebigen Teilnehmer dient als Nachweis für geleistete Rechenarbeit - daher die Bezeichnung \textit{Proof-of-Work} \parencite[S. 27]{Brünnler_BlockchainKurzGut}. Ein großer Nachteil bei Proof-of-Work ist der hohe Energieverbrauch, der dem hohen Rechenaufwand zum Lösen des Puzzles geschuldet ist \parencite{Zhang_EvaluationOfEnergyConsumptionInBlockChains}. Die Website \textit{Digiconomist.net} führt einen Energieindex für Bitcoin, der den Energieverbrauch von Bitcoin in Relation zu verschiedenen Ländern setzt. Aktuell benötigt Bitcoin jährlich rund $144$ TWh, was ungefähr dem Energieverbrauch von Schweden entspricht \parencite{Digiconomist_BitcoinEnergyConsumption}.


Der \textit{Proof-of-Stake} (PoS) ist ein alternativer Konsensmechanismus. Im Gegensatz zum PoW wird beim PoS kein kryptografisches Puzzle gelöst. Stattdessen wird ein Teilnehmer ausgewählt, der einen neuen Block an die Blockchain anhängen darf. Die Auswahl erfolgt zufällig, wobei die Wahrscheinlichkeit, ausgewählt zu werden, von der Anzahl der Coins abhängt, die der Teilnehmer besitzt (Stake). Wenn ein Teilnehmer beispielsweise 10\% aller Coins besitzt, hat er eine 10\%ige Chance, ausgewählt zu werden. Der Teilnehmer, der ausgewählt wurde, wird auch als \textit{Validator} bezeichnet. Der Validator kann dann einen neuen Block an die Blockchain anhängen. Wenn der Validator einen Block an die Blockchain anhängt, erhält er eine Belohnung. Wenn er allerdings einen Block an die Blockchain anhängt, der ungültig ist, verliert er einen Teil seiner Coins. Dadurch wird sichergestellt, dass von den Validatoren nur gültige Blöcke an die Blockchain angehängt werden \parencites[S. 96-97]{Kapengut_EthereumTransitionToProofOfStake}[S. 34]{Meinel_BlockchainHypeInnovation}[S. 320-321]{Antonopoulos_MasteringEthereum}.

Die beiden Konsensmechanismen unterscheiden sich grundlegend in ihrer Funktionsweise. PoW, das von Bitcoin genutzt wird, erfordert von Minern, komplexe mathematische Rätsel zu lösen, welche rechenintensiv sind und deshalb viel Energie benötigen. Der Miner, der das Rätsel zuerst löst, kann einen neuen Block hinzufügen und erhält eine Belohnung in Form von Kryptowährung \parencite{Fill_BlockchainGrundlagen}. PoW ist bekannt für:

\begin{itemize}
    \item Energieintensität: PoW erfordert hohe Rechenleistung und verbraucht viel Energie, was Umweltbedenken aufwirft \parencite[S. 96-97]{Kapengut_EthereumTransitionToProofOfStake}.
    \item Sicherheit: Das Netzwerk ist widerstandsfähig gegen Angriffe, da ein Angreifer die Kontrolle über die Mehrheit der Rechenleistung benötigt, um das Netzwerk zu übernehmen, was sehr teuer ist.
\end{itemize}

\noindent Im Gegensatz dazu nutzt PoS den Besitz von Kryptowährung als Sicherheitsfaktor. Validatoren werden auf Basis ihres Einsatzes (engl. \textit{Stake}) ausgewählt, um Transaktionen zu bestätigen \parencites[S. 34]{Meinel_BlockchainHypeInnovation}[S. 320-321]{Antonopoulos_MasteringEthereum}. PoS bietet:

\begin{itemize}
    \item Energieeffizienz: PoS ist energieeffizienter, da es nicht die immense Rechenleistung von PoW erfordert \parencite[S. 96-97]{Kapengut_EthereumTransitionToProofOfStake}.
    \item Sicherheit durch Einsätze: Die ValidatorInnen haben einen Anreiz, sich ehrlich zu verhalten, da sie ihren Einsatz verlieren könnten, wenn sie betrügen \parencites[S. 34]{Meinel_BlockchainHypeInnovation}[S. 320-321]{Antonopoulos_MasteringEthereum}.
\end{itemize}

\noindent Zusammenfassend: PoW ist energieintensiver und bietet Sicherheit durch Rechenleistung, während PoS energieeffizienter ist und Sicherheit durch den Einsatz von Kryptowährung gewährleistet. Beide Mechanismen haben Vor- und Nachteile, und ihre Anwendung hängt von den spezifischen Anforderungen und Zielen des jeweiligen Blockchain-Netzwerks ab.


\subsection{Sicherheit von Blockchain}

Die Blockchain gilt als sicher, da sie auf verschiedenen kryptografischen Technologien basiert (siehe Abschnitt \ref{subsec:cryptography_basics} \nameref{subsec:cryptography_basics}). Hinzu kommt die Unveränderlichkeit der Blockchain, was bedeutet, dass einmal geschriebene Transaktionen nach der Bestätigung durch den Konsensmechanismus als unveränderlich gelten und die Blockchain damit eine Fälschungssicherheit bietet \Parencites[S. 1-2]{Landerreche_ImmutabilityOfBlockchains}[S. 70]{Brünnler_BlockchainKurzGut}. Da Blöcke durch kryptografische Hashes verknüpft sind, wäre eine nachträgliche Änderung äußerst rechenintensiv und offensichtlich, da alle nachfolgenden Blöcke ebenfalls verändert werden müssten \Parencite[S. 12]{Fill_BlockchainGrundlagen}. Zusätzlich zu diesen Grundlagen sind Sicherheitsmaßnahmen entscheidend. Anreizstrukturen belohnen Miner/Validatoren für ehrliches Handeln und bestrafen bösartiges Verhalten, was als Abschreckung dient \parencite[S. 320-321]{Antonopoulos_MasteringEthereum}. Die dezentrale Natur der Blockchain macht sie widerstandsfähig gegen verschiedene Angriffe, da es keinen einzelnen Angriffspunkt gibt \parencite[S. 31]{Fill_BlockchainGrundlagen}.

Letztendlich unterliegt die Sicherheit der Blockchain jedoch keinem absoluten Schutz. Risiken wie 51\%-Angriffe oder Schwachstellen in spezifischen Implementierungen können weiterhin Bedrohungen darstellen. Daher sind kontinuierliche Forschung, Protokollaktualisierungen und bewährte Sicherheitspraktiken entscheidend, um Risiken zu minimieren und die Sicherheit der Blockchain weiter zu verbessern \parencites{Nam_51PercentAttacks}{Perez_SmartContractVulnerabilities}.


\subsection{Angriffe auf Blockchain}

Die Blockchain-Technologie ist nicht immun gegen Angriffe. Es gibt verschiedene Angriffe, die auf Blockchains durchgeführt werden können. Die bekanntesten Angriffe sind der 51\%-Angriff, der Sybil-Angriff und für Ethereum 2.0 die Smart Contract Exploits \Parencites[S. 95034]{Nam_51PercentAttacks}[S. 251]{Douceur_SybilAttack}. Diese Angriffe werden in den folgenden Abschnitten erläutert.

\subsubsection{51\%-Angriff}

Ein 51\%-Angriff ist ein potenziell bedrohlicher Angriff auf eine Blockchain, bei dem eine einzelne Entität oder eine koordinierte Gruppe die Kontrolle über die Mehrheit der Rechenleistung (bei PoW) oder der Stake (bei PoS) eines Netzwerks erlangt. Dies könnte dazu führen, dass die Angreifer die Blockchain manipulieren, doppelte Ausgaben tätigen (auch als \textit{Double Spending} bekannt) oder Transaktionen zensieren können \Parencites[S. 95034]{Nam_51PercentAttacks}[S. 2]{Rosenfeld_DoubleSpending}.

\subsubsection{Smart Contract Exploits}

Smart Contracts sind Programme, die auf der Blockchain ausgeführt werden. Sie werden mit Hilfe einer Programmiersprache geschrieben und wie auch bei jedem anderen Programm können auch Smart Contracts Fehler enthalten. Wenn ein Fehler in einem Smart Contract identifiziert wird, kann dieser Fehler ausgenutzt werden, um den Smart Contract zu manipulieren \Parencite[S. 1-2]{Perez_SmartContractVulnerabilities}. Ein Beispiel für einen Smart Contract Exploit ist der \textit{DAO Hack}, der im Juni 2016 stattfand. Der DAO Hack ist ein Angriff auf den Smart Contract \textit{The DAO}, der auf der Ethereum-Blockchain ausgeführt wurde. Der Smart Contract wurde so manipuliert, dass der Angreifer Ether im Wert von 50 Millionen US-Dollar stehlen konnte \Parencite{Pratap_DAOHack}.


\subsubsection{Sybil-Angriff}

Wie in Abschnitt \ref{subsubsec:sybil_attack_p2p} \textit{\nameref{subsubsec:sybil_attack_p2p}} bereits ausgeführt, ist ein Sybil-Angriff ein Angriff auf ein Peer-to-Peer-Netzwerk, bei dem ein einzelner Angreifer mehrere Identitäten verwendet, um das Netzwerk zu manipulieren. Da Blockchains auf einem Peer-to-Peer-Netzwerk basieren, sind sie ebenfalls anfällig für Sybil-Angriffe.  

\subsection{Ethereum}
Ethereum 1.0 und 2.0
\subsection{Smart Contracts}
Smart Contracts sind Programme, die auf derBlockchain ausgeführt werden. Smart Contracts 
sind in der Programmiersprache Solidity geschrieben. Smart Contracts sind in der Blockchain
gespeichert und werden von allen Teilnehmern des Netzwerks ausgeführt. Smart Contracts sind
nicht veränderbar und können nicht gelöscht werden. Smart Contracts können nur durch andere
Smart Contracts oder durch Transaktionen aufgerufen werden. Smart Contracts können Daten
in der Blockchain speichern. Smart Contracts können auch Daten aus der Blockchain lesen.
Smart Contracts können auch Transaktionen auslösen. Smart Contracts können auch andere
Smart Contracts aufrufen.