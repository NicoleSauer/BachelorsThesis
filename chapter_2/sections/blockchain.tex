\section{Blockchain-Technologie}
\textbf{\textcolor{red}{Ralph Merkle!}}
Die Blockchain-Technologie ist... . Blockchains sind auch bekannt als Distributed Ledger 
Technology (DLT) und das sind auch nur verteilte Datenbanken. Die Blockchain ist eine
spezielle Form der DLT. Die Blockchain ist eine Kette von Blöcken, die jeweils einen
Hash des vorherigen Blocks enthalten. Die Blockchain ist eine dezentrale Datenbank, die
von allen Teilnehmern des Netzwerks verwaltet wird.

Proof of Work und Proof of Stake sind zwei verschiedene Konsensmechanismen, die verwendet
werden, um die Blockchain zu validieren. Proof of Work ist der Konsensmechanismus, der
von Bitcoin verwendet wird. Proof of Stake ist der Konsensmechanismus, der von Ethereum (2.0)
verwendet wird.

Der bekannteste Anwendungsfall für Blockchain ist Bitcoin. Bitcoin ist eine Kryptowährung,
die auf der Blockchain-Technologie basiert. Bitcoin wurde 2008 von Satoshi Nakamoto entwickelt
und ist seitdem stetig gewachsen. Bitcoin ist eine dezentrale Kryptowährung, die von niemandem
kontrolliert wird.

Eine weitere bekannte Anwendung für Blockchain ist Ethereum. Ethereum ist eine dezentrale
Plattform, die es ermöglicht, Smart Contracts zu erstellen und auszuführen. Smart Contracts
sind Programme, die auf der Blockchain ausgeführt werden und die Möglichkeit bieten, Transaktionen und
Verträge zu erstellen, die automatisch ausgeführt werden, sobald bestimmte Bedingungen erfüllt sind.
Ethereum ist eine der bekanntesten und am weitesten verbreiteten Blockchain-Plattformen. Die Plattform
wurde 2015 von Vitalik Buterin entwickelt und ist seitdem stetig gewachsen.


Auf der Ethereum-Blockchain wird hauptsächlich Programmiersprache Solidity verwendet, um Smart Contracts zu erstellen. Solidity ist eine objektorientierte Programmiersprache, die stark an JavaScript angelehnt ist \parencite[S. 131]{Antonopoulos_MasteringEthereum}. Die Smart Contracts, die für das Protokoll benötigt werden, sind in Solidity implementiert. Die Smart Contracts werden auf der Ethereum-Blockchain ausgeführt und können von jedem Teilnehmer des Netzwerks aufgerufen werden. Die Smart Contracts werden in Abschnitt \ref{subsec:registrierung} (\nameref{subsec:registrierung}) und Abschnitt \ref{subsec:kommunikation} (\nameref{subsec:kommunikation}) beschrieben.

\begin{itemize}
    \item Warum ist Blockchain sicher?
    \item Angriffe auf Blockchain (Sybil, 51\%, etc.)
\end{itemize}

\subsection{Ethereum}
\label{sec:ethereum_basics}

Ethereum ist eine der führenden Blockchain-Plattformen und wurde 2015 von Vitalik Buterin, Gavin Wood und anderen entwickelt. Im Gegensatz zu Bitcoin, das hauptsächlich als digitale Währung fungiert, ermöglicht Ethereum die Ausführung von Smart Contracts und die Entwicklung von dezentralen Anwendungen (engl. \textit{decentralized Applications}, kurz \textit{DApps}) \Parencites[S. 720]{Sorge_BitcoinZahlungsmittelDerZukunft}[S. 1-2]{Perez_SmartContractVulnerabilities}. Bis 2022 wurde als Konsensmechanismus Proof-of-Work verwendet, der jedoch 2022 durch Proof-of-Stake ersetzt wurde. Seitdem existieren zwei Versionen von Ethereum: \textit{Ethereum Classic} und \textit{Ethereum 2.0}. Ethereum Classic verwendet Proof-of-Work, während Ethereum 2.0 Proof-of-Stake verwendet \Parencite{EthereumClassic_ResearcherFAQs}.

Ether (ETH) ist die native Kryptowährung von Ethereum und wird für Transaktionen innerhalb des Netzwerks verwendet. Es dient auch als Anreiz für diejenigen, die an der Sicherung des Netzwerks durch Mining (bei PoW) oder Validierung (bei PoS) von Transaktionen beteiligt sind \parencite[S. 320-321]{Antonopoulos_MasteringEthereum}. Die Flexibilität von Ethereum und seine Fähigkeit, innovative Lösungen zu unterstützen, haben es zu einer der wichtigsten Plattformen in der Blockchain-Welt gemacht.

\subsection{Smart Contracts}
\label{subsection:smart_contracts}
% Smart Contracts werden per RPC aufgerufen.


Ein Smart Contract (zu Deutsch: \textit{intelligenter Vertrag}) ist im Grunde genommen ein selbstausführender Vertrag, der automatisch Aktionen auslöst, wenn bestimmte Bedingungen erfüllt sind \Parencite[S. 1-2]{Perez_SmartContractVulnerabilities}. Die Bezeichnung \textit{Smart Contract} ist eigentlich eine Fehlbezeichnung, da es sich weder um einen Vertrag im rechtlichen Sinne, noch um einen \textit{intelligenten} Vertrag handelt, doch der Begriff hat sich in der Blockchain-Community etabliert und wird deshalb weiterhin verwendet. Ein Lesezugriff auf einen Smart Contract ist kostenlos, ein Schreibzugriff hingegen kostet Geld, da die Transaktion in der Blockchain gespeichert werden muss. Dieses Geld wird als \textit{Gas} bezeichnet und ist eine Art Gebühr, die gezahlt werden muss, um die Rechenleistung des Netzwerks zu nutzen \Parencite[S. 127]{Antonopoulos_MasteringEthereum}. Um Gas zu erhalten, muss der Nutzer Ether eintauschen, die Währung der Ethereum-Blockchain.
Für das Protokoll dieser Arbeit wurden sowohl Lese- als auch Schreibzugriffe auf Smart Contracts implementiert (siehe Kapitel \ref{chap:entwurf_und_architektur} \textit{\nameref{chap:entwurf_und_architektur}}).

Die Plattform verwendet die objektorientierte Programmiersprache \textit{Solidity}, die speziell für Smart Contracts entwickelt wurde und stark an JavaScript angelehnt ist. Entwickler können mithilfe von Solidity Smart Contracts erstellen, die dann in der Ethereum-Blockchain ausgeführt werden und von jedem Teilnehmer des Netzwerks aufgerufen werden können \Parencite[S. 127-133]{Antonopoulos_MasteringEthereum}.
‚