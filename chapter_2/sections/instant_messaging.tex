\section{Instant Messaging}
% Was ist Instant Messaging?
% Wofür wird es verwendet?
% Welche Anwendungen gibt es? (WhatsApp, Signal, Telegram, Threema, etc.) -> die meisten sind zentralisiert
% Welche Probleme gibt es bei zentralisierten Anwendungen?
% Client-Server-Architektur und Peer to Peer als Überleitung zur nächsten Sektion verwenden


Instant Messaging bezeichnet eine Form der Kommunikation, bei der Nachrichten in Echtzeit zwischen zwei oder mehreren Personen über das Internet ausgetauscht werden können \Parencite[S. 69]{nist_mobileDeviceForensics}. Diese Form der digitalen Kommunikation ermöglicht es Nutzern, sofortige Nachrichten, Bilder und andere Mediendateien auszutauschen. Instant Messaging-Dienste umfassen eine Vielzahl von Plattformen und Anwendungen, die es Benutzern ermöglichen, miteinander zu kommunizieren, sei es eins zu eins oder in Gruppenchats. Instant-Messaging-Dienste reichen von plattformübergreifenden Anwendungen wie \textit{WhatsApp}, \textit{Signal} und \textit{Telegram} bis hin zu spezialisierten Unternehmenslösungen wie \textit{Slack} oder \textit{Microsoft Teams}. Die Vielfalt an Funktionen in Instant-Messaging-Plattformen ist groß. Neben einfachen Textnachrichten können Benutzer zum Beispiel mit der App \textit{WhatsApp} Emojis, Aufkleber, GIFs und Sprachnachrichten teilen, was die Kommunikation dynamisch und ausdrucksstark gestaltet \Parencite{whatsapp_funktionen}. Gruppenchats ermöglichen es mehreren Personen, in einer einzigen Unterhaltung zusammenzukommen, was die Zusammenarbeit, soziale Interaktion und Informationsverbreitung erleichtert.
Sicherheit und Datenschutz sind in der Welt des Instant Messaging von entscheidender Bedeutung. Verschlüsselungstechnologien werden verwendet, um die Vertraulichkeit der Nachrichten zu gewährleisten und die Privatsphäre der Nutzer zu schützen. Die Authentifizierung von Benutzern, End-to-End-Verschlüsselung und andere Sicherheitsmechanismen sind unerlässlich, um die Integrität der Kommunikation zu gewährleisten.
Instant Messaging-Anwendungen können auf verschiedene Arten strukturiert sein, wobei zwei Hauptansätze hervorstechen: die Client-Server-Architektur und das Peer-to-Peer-Modell.
Die Client-Server-Architektur ist bei vielen gängigen Instant Messaging-Diensten verbreitet. Hierbei fungiert ein zentraler Server als Vermittler zwischen den Benutzergeräten. Die Nachrichten werden von den Clients (den Benutzergeräten) an den Server gesendet, der sie dann an die Empfänger weiterleitet. Dieser Ansatz bietet eine einfache Verwaltung der Kommunikation, zentralisierte Datenverwaltung und ermöglicht Dienstleistungen wie das Offline-Speichern von Nachrichten. Beispiele hierfür sind Plattformen wie WhatsApp und Signal, die diese Architektur nutzen, um Nachrichten zwischen Benutzern zu vermitteln.
Im Gegensatz dazu basiert das Peer-to-Peer-Modell (kurz: P2P) auf direkten Verbindungen zwischen den Benutzergeräten ohne die Notwendigkeit eines zentralen Servers. Jedes Gerät agiert sowohl als Client als auch als Server, wodurch die Kommunikation direkt zwischen den Teilnehmern stattfindet. Diese Struktur bietet potenzielle Vorteile in Bezug auf Datenschutz, da die Nachrichten nicht über einen zentralen Server geleitet werden müssen. P2P-IM-Anwendungen wie beispielsweise BitTorrent Chat oder Tox setzen auf diese Architektur, um eine dezentralisierte und möglicherweise sicherere Kommunikationsumgebung zu schaffen.
Beide Ansätze haben ihre eigenen Vor- und Nachteile. Die Client-Server-Architektur ermöglicht eine einfachere Verwaltung und Zuverlässigkeit, birgt jedoch potenzielle Datenschutzrisiken, da Daten zentralisiert gespeichert werden. Auf der anderen Seite bietet das Peer-to-Peer-Modell potenziell mehr Privatsphäre und Sicherheit, aber es könnte Schwierigkeiten bei der Skalierbarkeit und Verlässlichkeit geben, da es keine zentrale Instanz gibt, die die Kommunikation steuert. Die Wahl zwischen diesen Architekturen hängt von den spezifischen Anforderungen, Sicherheitsbedenken und dem Nutzungskontext der Instant Messaging-Anwendung ab.




Als Alternative zur Client-Server-Architektur gibt es auch Peer-to-Peer-Instant-\\Messaging-Anwendungen, bei denen die Kommunikation direkt zwischen den Teilnehmern stattfindet. Diese Anwendungen sind in der Regel dezentralisiert und verwenden keine zentrale Instanz, um die Kommunikation zu vermitteln. Stattdessen werden die Nachrichten direkt zwischen den Teilnehmern ausgetauscht. Die dezentrale Natur dieser Anwendungen macht sie unabhängig von zentralen Servern und ermöglicht es den Benutzern, die Kontrolle über ihre Daten zu behalten. Die dezentrale Architektur bietet auch eine höhere Ausfallsicherheit, da die Kommunikation nicht von einem zentralen Server abhängig ist. Die dezentrale Architektur ist jedoch auch mit einigen Herausforderungen verbunden. Die dezentrale Natur der Anwendung macht es schwieriger, die Identität der Benutzer zu verifizieren, da es keine zentrale Instanz gibt, die die Authentizität der Benutzer überprüfen kann. Darüber hinaus ist es schwieriger, die Integrität der Nachrichten zu gewährleisten, da die Nachrichten nicht über einen zentralen Server weitergeleitet werden. Die dezentrale Architektur ist auch anfälliger für Angriffe, da die Kommunikation nicht über einen zentralen Server vermittelt wird.