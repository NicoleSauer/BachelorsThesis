\section{Instant Messaging}
\label{sec:instant_messaging_basics}

\textit{Instant Messaging}, was übersetzt \textit{sofortige Nachrichtenübermittlung} bedeutet, bezeichnet eine Form der Kommunikation, bei der Nachrichten in Echtzeit zwischen zwei oder mehreren Personen über das Internet ausgetauscht werden können \Parencite[S. 69]{nist_mobileDeviceForensics}. Diese Form der digitalen Kommunikation ermöglicht es Nutzern, sofortige Nachrichten, Bilder und andere Mediendateien auszutauschen. Instant-Messaging-Dienste reichen von plattformübergreifenden Anwendungen wie \textit{WhatsApp}, \textit{Signal} und \textit{Telegram} bis hin zu spezialisierten Unternehmenslösungen wie \textit{Slack} oder \textit{Microsoft Teams} \parencite{Plett_IMDefinition}. Die Vielfalt an Funktionen in Instant-Messaging-Plattformen ist groß. Neben einfachen Textnachrichten können zum Beispiel Benutzer der Anwendung \textit{WhatsApp} Emojis, Aufkleber, GIFs (animierte Bilder) und Sprachnachrichten teilen, was die Kommunikation dynamisch und ausdrucksstark gestaltet \Parencite{whatsapp_funktionen}.

Sicherheit und Datenschutz sind in der Welt des Instant Messaging von entscheidender Bedeutung. Verschlüsselungstechnologien werden verwendet, um die Vertraulichkeit der Nachrichten zu gewährleisten und die Privatsphäre der Nutzer zu schützen \parencite[S. 13706]{Wang_IMSecurity}.


\subsection{IM-Protokolle in der Übersicht(?)}

% #TODO: Würde gerne behaupten, dass die meisten proprietäre Protokolle verwenden, aber wie soll ich das belegen?
Durch die Verwendung von Instant Messaging-Anwendungen können Benutzer Nachrichten in Echtzeit austauschen. Die Übertragung der Nachrichten erfolgt über ein Netzwerk \Parencite[S. 69]{nist_mobileDeviceForensics}. Um die Kommunikation zu ermöglichen, müssen die Teilnehmer ein gemeinsames Protokoll verwenden. Ein Protokoll besteht aus einer Reihe von Regeln, die die Kommunikation zwischen Geräten (Computer oder auch Smartphones) ermöglichen. Damit beschreibt ein Instant Messaging-Protokoll eine Reihe von Regeln, die die Kommunikation zwischen Benutzern einer Instant Messaging-Anwendung realisieren \Parencite{Novalnet_Protokoll}. Im Jahr 2023 waren in Deutschland die folgenden Instant Messaging-Anwendungen am beliebtesten: \textit{WhatsApp}, \textit{Facebook Messenger} und \textit{Telegram} \parencite{Statista_MessengerNutzung}. Diese Anwendungen verwenden unterschiedliche Ansätze, um die Kommunikation zu ermöglichen \parencite[S. 103]{Luntovskyy_ModRechnernetze}. 


\subsection{Architekturmodelle}

\subsubsection{Client-Server-Architektur}

Bei dieser Art von Protokoll wird die Kommunikation über einen zentralen Server ermöglicht. Die Teilnehmer kommunizieren nicht direkt miteinander, sondern senden ihre Nachrichten an den Server, der sie dann an den Empfänger weiterleitet\Parencite[S. 3]{Hanson_ServerManagement}. Einige Beispiele für Instant Messaging-Anwendungen, die ein Protokoll mit zentralem Server verwenden, sind \textit{WhatsApp} \parencite{Vanerio_WhatsAppArchitecture}, \textit{Signal} \parencite{Github_libsignal}, \textit{Telegram} \parencite{Telegram_ServerSourceCode} und \textit{Threema} \parencite{Threema_ServerLocation}.

\subsubsection{Peer-to-Peer-Modell}

Bei dieser Art von Protokoll kommunizieren die Teilnehmer direkt miteinander, ohne einen zentralen Server zu verwenden. Die Kommunikation erfolgt direkt zwischen den Teilnehmern \parencites[S. 6-8]{Mahlmann_P2PNetzwerke}{Galuba_P2POverlayNetworks}. Einige Beispiele für Instant Messaging-Anwendungen, die ein Protokoll mit Peer-to-Peer-Modell verwenden, sind \textit{Tox} \parencite{Tox_FAQ} und \textit{Briar} \parencite{Briar_HowItWorks}.