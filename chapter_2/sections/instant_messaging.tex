\section{Instant Messaging}
% Was ist Instant Messaging?
% Wofür wird es verwendet?
% Welche Anwendungen gibt es? (WhatsApp, Signal, Telegram, Threema, etc.) -> die meisten sind zentralisiert
% Welche Probleme gibt es bei zentralisierten Anwendungen?
% Client-Server-Architektur und Peer to Peer als Überleitung zur nächsten Sektion verwenden


\textit{Instant Messaging}, was übersetzt \textit{sofortige Nachrichtenübermittlung} bedeutet, bezeichnet eine Form der Kommunikation, bei der Nachrichten in Echtzeit zwischen zwei oder mehreren Personen über das Internet ausgetauscht werden können \Parencite[S. 69]{nist_mobileDeviceForensics}. Diese Form der digitalen Kommunikation ermöglicht es Nutzern, sofortige Nachrichten, Bilder und andere Mediendateien auszutauschen. Instant-Messaging-Dienste reichen von plattformübergreifenden Anwendungen wie \textit{WhatsApp}, \textit{Signal} und \textit{Telegram} bis hin zu spezialisierten Unternehmenslösungen wie \textit{Slack} oder \textit{Microsoft Teams} \parencite{Plett_IMDefinition}. Die Vielfalt an Funktionen in Instant-Messaging-Plattformen ist groß. Neben einfachen Textnachrichten können zum Beispiel Benutzer der Anwendung \textit{WhatsApp} Emojis, Aufkleber, GIFs (animierte Bilder(?)) und Sprachnachrichten teilen, was die Kommunikation dynamisch und ausdrucksstark gestaltet \Parencite{whatsapp_funktionen}.

Sicherheit und Datenschutz sind in der Welt des Instant Messaging von entscheidender Bedeutung. Verschlüsselungstechnologien werden verwendet, um die Vertraulichkeit der Nachrichten zu gewährleisten und die Privatsphäre der Nutzer zu schützen \parencite[S. 13706]{Wang_IMSecurity}. Die Authentifizierung von Benutzern, Ende-zu-Ende-Verschlüsselung und andere Sicherheitsmechanismen sind unerlässlich, um die Integrität der Kommunikation zu gewährleisten.


\subsection{IM-Protokolle in der Übersicht}

Durch die Verwendung von Instant Messaging-Anwendungen können Benutzer Nachrichten in Echtzeit austauschen. Die Übertragung der Nachrichten erfolgt über ein Netzwerk. Um die Kommunikation zu ermöglichen, müssen die Teilnehmer ein gemeinsames Protokoll verwenden. Ein Protokoll besteht aus einer Reihe von Regeln, die die Kommunikation zwischen Geräten (Computer oder auch Smartphones) ermöglichen. Damit beschreibt ein Instant Messaging-Protokoll eine Reihe von Regeln, die die Kommunikation zwischen Benutzern einer Instant Messaging-Anwendung ermöglichen. Es gibt unterschiedliche Ansätze, um die Kommunikation zu ermöglichen. 


\subsubsection{Protokolle mit zentralem Server}
Bei dieser Art von Protokoll wird die Kommunikation über einen zentralen Server ermöglicht. Die Teilnehmer kommunizieren nicht direkt miteinander, sondern senden ihre Nachrichten an den Server, der sie dann an den Empfänger weiterleitet. Die Kommunikation zwischen den Teilnehmern erfolgt über das Client-zu-Server-Protokoll (C2S). Die Kommunikation zwischen den Servern erfolgt über das Server-zu-Server-Protokoll (S2S). Die meisten Instant Messaging-Anwendungen verwenden diese Art von Protokoll. Einige Beispiele für Instant Messaging-Anwendungen, die ein Protokoll mit zentralem Server verwenden, sind \textit{WhatsApp}, \textit{Signal}, \textit{Telegram} und \textit{Threema}.



Protokolle mit zentralem Server:

\begin{itemize}
    \item XMPP, Matrix, Telegram, Signal, Threema, WhatsApp, IRC, Skype
\end{itemize}

Protokolle mit P2P:

\begin{itemize}
    \item Tox, Ricochet, Briar, Jami, Ring, Bitmessage, RetroShare
\end{itemize}

\subsubsection{XMPP}
Dieses Protokoll ist ein offener Standard für Instant Messaging und basiert auf XML. Es ist ein
Protokoll mit zentralem Server. Es gibt verschiedene XMPP-Server, die miteinander kommunizieren
können. Die Kommunikation zwischen den Servern erfolgt über das Server-zu-Server-Protokoll (S2S).
Die Kommunikation zwischen den Teilnehmern erfolgt über das Client-zu-Server-Protokoll (C2S).

\subsubsection{Matrix}
Dieses Protokoll ist ein offener Standard für Instant Messaging und basiert auf JSON. Es ist ein
Protokoll mit zentralem Server. Es gibt verschiedene Matrix-Server, die miteinander kommunizieren
können. Die Kommunikation zwischen den Servern erfolgt über das Server-zu-Server-Protokoll (S2S).
Die Kommunikation zwischen den Teilnehmern erfolgt über das Client-zu-Server-Protokoll (C2S).

\subsubsection{Telegram}
Dieses Protokoll ist ein proprietäres Protokoll für Instant Messaging und basiert auf JSON. Es ist
ein Protokoll mit zentralem Server. Es gibt verschiedene Telegram-Server, die miteinander
kommunizieren können. Die Kommunikation zwischen den Servern erfolgt über das Server-zu-Server-
Protokoll (S2S). Die Kommunikation zwischen den Teilnehmern erfolgt über das Client-zu-Server-
Protokoll (C2S).


\subsection{Client-Server-Architektur und Peer-to-Peer-Modell}

Instant Messaging-Anwendungen können auf verschiedene Arten strukturiert sein, wobei zwei Hauptansätze hervorstechen: die Client-Server-Architektur und das Peer-to-Peer-Modell.
Die Client-Server-Architektur ist bei vielen gängigen Instant Messaging-Diensten verbreitet. Hierbei fungiert ein zentraler Server als Vermittler zwischen den Benutzergeräten. Die Nachrichten werden von den Clients (den Benutzergeräten) an den Server gesendet, der sie dann an die Empfänger weiterleitet. Dieser Ansatz bietet eine einfache Verwaltung der Kommunikation, zentralisierte Datenverwaltung und ermöglicht Dienstleistungen wie das Offline-Speichern von Nachrichten.
Im Gegensatz dazu basiert das Peer-to-Peer-Modell (kurz: P2P) auf direkten Verbindungen zwischen den Benutzergeräten ohne die Notwendigkeit eines zentralen Servers. Jedes Gerät agiert sowohl als Client als auch als Server, wodurch die Kommunikation direkt zwischen den Teilnehmern stattfindet. Diese Struktur bietet potenzielle Vorteile in Bezug auf Datenschutz, da die Nachrichten nicht über einen zentralen Server geleitet werden müssen.
Beide Ansätze haben ihre eigenen Vor- und Nachteile. Die Client-Server-Architektur ermöglicht eine einfachere Verwaltung und Zuverlässigkeit, birgt jedoch potenzielle Datenschutzrisiken, da Daten zentralisiert gespeichert werden. Auf der anderen Seite bietet das Peer-to-Peer-Modell potenziell mehr Privatsphäre und Sicherheit, aber es könnte Schwierigkeiten bei der Skalierbarkeit und Verlässlichkeit geben, da es keine zentrale Instanz gibt, die die Kommunikation steuert. Die Wahl zwischen diesen Architekturen hängt von den spezifischen Anforderungen, Sicherheitsbedenken und dem Nutzungskontext der Instant Messaging-Anwendung ab.