\section{Instant Messaging}
% Was ist Instant Messaging?
% Wofür wird es verwendet?
% Welche Anwendungen gibt es? (WhatsApp, Signal, Telegram, Threema, etc.) -> die meisten sind zentralisiert
% Welche Probleme gibt es bei zentralisierten Anwendungen?
% Client-Server-Architektur und Peer to Peer als Überleitung zur nächsten Sektion verwenden


Instant Messaging bezeichnet eine Form der Kommunikation, bei der Nachrichten in Echtzeit zwischen zwei oder mehreren Personen über das Internet ausgetauscht werden können \Parencite[S. 69]{nist_mobileDeviceForensics}. Diese Form der digitalen Kommunikation ermöglicht es Nutzern, sofortige Nachrichten, Bilder und andere Mediendateien auszutauschen. Instant Messaging-Dienste umfassen eine Vielzahl von Plattformen und Anwendungen, die es Benutzern ermöglichen, miteinander zu kommunizieren, sei es eins zu eins oder in Gruppenchats. Instant -Messaging-Dienste reichen von plattformübergreifenden Anwendungen wie WhatsApp, Facebook Messenger und Telegram bis hin zu spezialisierten Unternehmenslösungen wie Slack oder Microsoft Teams. Die Vielfalt an Funktionen in Instant-Messaging-Plattformen ist groß. Neben einfachen Textnachrichten können Benutzer Emojis, Aufkleber, GIFs und Multimedia-Dateien teilen, was die Kommunikation dynamisch und ausdrucksstark gestaltet. Gruppenchats ermöglichen es mehreren Personen, in einer einzigen Unterhaltung zusammenzukommen, was die Zusammenarbeit, soziale Interaktion und Informationsverbreitung erleichtert.