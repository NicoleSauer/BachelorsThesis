\section{Peer-to-Peer-Technologie}


Peer-to-Peer (P2P) ist ein Netzwerkprotokoll, bei dem alle Teilnehmer gleichberechtigt sind und es 
keinen zentralen Server gibt, der die Kommunikation steuert. Die Kommunikation erfolgt direkt zwischen 
den Teilnehmern. Alle Teilnehmer sind in einem Netzwerk organisiert, das aus verschiedenen Knoten 
besteht, wobei jeder Knoten einen Teilnehmer des Netzwerks repräsentiert. Alle Knoten im Netzwerk 
sind untereinander verbunden und können so Textnachrichten und Dateien austauschen 
\parencite[\textcolor{red}{S. x-y}]{Mahlmann_P2PNetzwerke}.

\begin{itemize}
    \item NAT-Problem erklären (siehe RFC 5128)
    \item Mögliche Lösungen des NAT-Problems (siehe RFC 5128)
\end{itemize}