\section{Peer-to-Peer-Technologie}
\label{subsec:peer_to_peer_technologie}
% #TODO: Funktion des Kademlia Protokolls nennen und erklären (vielleicht in Grundlagen)
% Im Kademlia-Protokoll sind vier Funktionen definiert, die für die Suche nach
% Knoten und Werten verwendet werden. Diese Funktionen sind \texttt{FIND\_NODE},
% \texttt{FIND\_VALUE}, \texttt{PING} und \texttt{STORE}. Die Funktionen
% \texttt{FIND\_NODE} und \texttt{FIND\_VALUE} werden verwendet, um nach Knoten
% oder Werten zu suchen. Die Funktion \texttt{PING} wird verwendet, um die
% Erreichbarkeit eines Knotens zu überprüfen. Die Funktion \texttt{STORE} wird
% verwendet, um einen Wert in einem Knoten zu speichern.

Peer-to-Peer (P2P) bezeichnet ein Netzwerkmodell, bei dem Computer und andere Geräte direkt 
miteinander verbunden sind, um Ressourcen gemeinsam zu nutzen, ohne dass ein zentraler Server 
erforderlich ist. Bei diesem Netzwerkprotokoll sind alle 
Teilnehmer gleichberechtigt, es existiert kein zentraler Server, der die Kommunikation steuert. 
Die Kommunikation erfolgt direkt zwischen den Teilnehmern, die in einem Netzwerk aus verschiedenen 
Knoten organisiert sind. Jeder Knoten repräsentiert dabei einen Teilnehmer des Netzwerks und ermöglicht 
den direkten Austausch von Textnachrichten und Dateien zwischen den verbundenen Knoten.
\parencite[S. 6-8]{Mahlmann_P2PNetzwerke}.

Viele werden Peer-to-Peer schnell mit Filesharing in Verbindung bringen, da diese Technologie
in der Vergangenheit vor allem dafür genutzt wurde. Das bekannteste Beispiel ist das Filesharing-Netzwerk
Napster, das 1999 von Shawn \enquote{Napster} Fanning entwickelt wurde. Napster war das erste weit verbreitete
Filesharing-Netzwerk, das auf Peer-to-Peer-Technologie basierte. Es ermöglichte den Austausch von
Musikdateien zwischen den Teilnehmern. Die Musikdateien wurden dabei auf den Computern der Teilnehmer
gespeichert und konnten von anderen Teilnehmern heruntergeladen werden. Da diese Art des Datenaustauschs
oftmals illegal war, wurde Napster 2001 aufgrund von Urheberrechtsverletzungen abgeschaltet 
\parencite[S. 4-6]{Mahlmann_P2PNetzwerke}.


\begin{itemize}
    \item NAT-Problem erklären (siehe RFC 5128)
    \item Mögliche Lösungen des NAT-Problems (siehe RFC 5128)
    \item Kademlia-Protokoll und DHTs erklären
    \item STUN, TURN und ICE erklären (je nach dem, was ich davon verwende)
\end{itemize}