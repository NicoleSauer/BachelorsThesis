\subsection{Smart Contracts}
\label{subsection:smart_contracts}
Smart Contracts sind Programme, die auf derBlockchain ausgeführt werden. Smart Contracts 
sind in der Programmiersprache Solidity geschrieben. Smart Contracts sind in der Blockchain
gespeichert und werden von allen Teilnehmern des Netzwerks ausgeführt. Smart Contracts sind
nicht veränderbar und können nicht gelöscht werden. Smart Contracts können nur durch andere
Smart Contracts oder durch Transaktionen aufgerufen werden. Smart Contracts können Daten
in der Blockchain speichern. Smart Contracts können auch Daten aus der Blockchain lesen.
Smart Contracts können auch Transaktionen auslösen. Smart Contracts können auch andere
Smart Contracts aufrufen.

Smart Contracts werden per RPC aufgerufen.

Auf der Ethereum-Blockchain wird hauptsächlich Programmiersprache Solidity verwendet, um Smart Contracts zu erstellen. Solidity ist eine objektorientierte Programmiersprache, die stark an JavaScript angelehnt ist \parencite[S. 131]{Antonopoulos_MasteringEthereum}. Die Smart Contracts, die für das Protokoll benötigt werden, sind in Solidity implementiert. Die Smart Contracts werden auf der Ethereum-Blockchain ausgeführt und können von jedem Teilnehmer des Netzwerks aufgerufen werden.

Smart Contracts sind Programme, die auf der Blockchain ausgeführt werden. Sie sind in einer Turing-vollständigen Programmiersprache geschrieben und können daher alle möglichen Berechnungen durchführen. Smart Contracts können auch als \textit{Decentralized Applications} (DApps) bezeichnet werden. Smart Contracts sind nicht immun gegen Fehler. Wenn ein Fehler in einem Smart Contract auftritt, kann dieser Fehler ausgenutzt werden, um den Smart Contract zu manipulieren. Ein Beispiel für einen Smart Contract Exploit ist der \textit{DAO Hack}. Der DAO Hack ist ein Angriff auf den Smart Contract \textit{The DAO}, der auf der Ethereum-Blockchain ausgeführt wurde. Der Smart Contract wurde so manipuliert, dass der Angreifer Ether im Wert von 50 Millionen US-Dollar stehlen konnte. Der DAO Hack ist ein Beispiel dafür, dass Smart Contracts nicht immun gegen Fehler sind und dass diese Fehler ausgenutzt werden können, um den Smart Contract zu manipulieren.