\subsection{Smart Contracts}
\label{subsection:smart_contracts}
% Smart Contracts werden per RPC aufgerufen.


Ein Smart Contract (zu Deutsch: \textit{intelligenter Vertrag}) ist im Grunde genommen ein selbstausführender Vertrag, der automatisch Aktionen auslöst, wenn bestimmte Bedingungen erfüllt sind \Parencite[S. 1-2]{Perez_SmartContractVulnerabilities}. Die Bezeichnung \textit{Smart Contract} ist eigentlich eine Fehlbezeichnung, da es sich weder um einen Vertrag im rechtlichen Sinne, noch um einen \textit{intelligenten} Vertrag handelt, doch der Begriff hat sich in der Blockchain-Community etabliert und wird deshalb weiterhin verwendet. Ein Lesezugriff auf einen Smart Contract ist kostenlos, ein Schreibzugriff hingegen kostet Geld, da die Transaktion in der Blockchain gespeichert werden muss. Dieses Geld wird als \textit{Gas} bezeichnet und ist eine Art Gebühr, die gezahlt werden muss, um die Rechenleistung des Netzwerks zu nutzen \Parencite[S. 127]{Antonopoulos_MasteringEthereum}. Um Gas zu erhalten, muss der Nutzer Ether eintauschen, die Währung der Ethereum-Blockchain.
Für das Protokoll dieser Arbeit wurden sowohl Lese- als auch Schreibzugriffe auf Smart Contracts implementiert (siehe Kapitel \ref{chap:entwurf_und_architektur} \textit{\nameref{chap:entwurf_und_architektur}}).

Die Plattform verwendet die objektorientierte Programmiersprache \textit{Solidity}, die speziell für Smart Contracts entwickelt wurde und stark an JavaScript angelehnt ist. Entwickler können mit Hilfe von Solidity Smart Contracts erstellen, die dann in der Ethereum-Blockchain ausgeführt werden und von jedem Teilnehmer des Netzwerks aufgerufen werden können \Parencite[S. 127-133]{Antonopoulos_MasteringEthereum}.
