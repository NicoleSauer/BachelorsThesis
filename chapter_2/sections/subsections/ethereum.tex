\subsection{Ethereum}
\label{sec:ethereum_basics}

Ethereum ist eine der führenden Blockchain-Plattformen und wurde 2013 von Vitalik Buterin entwickelt. Im darauffolgenden Jahr veröffentlichte er seine Idee in einem Whitepaper mit dem Titel \textit{Ethereum: A Next-Generation Smart Contract and Decentralized Application Platform}. Darin bezeichnete er die Plattform als \textit{eine dezentrale Struktur mit einer eingebauten Turing-vollständigen Programmiersprache} \parencite{Buterin_EthereumWhitepaper}. Im Gegensatz zu Bitcoin, das hauptsächlich als digitale Währung fungiert, ermöglicht Ethereum die Ausführung von Smart Contracts und die Entwicklung von dezentralen Anwendungen (engl. \textit{decentralized Applications}, kurz \textit{DApps}) \Parencites[S. 720]{Sorge_BitcoinZahlungsmittelDerZukunft}[S. 1-2]{Perez_SmartContractVulnerabilities}. Bis 2022 wurde als Konsensmechanismus Proof-of-Work verwendet, der jedoch 2022 durch Proof-of-Stake ersetzt wurde. Seitdem existieren zwei Versionen von Ethereum: \textit{Ethereum} und \textit{Ethereum 2.0}. Ethereum verwendet Proof-of-Work, während Ethereum 2.0 Proof-of-Stake verwendet \Parencite{EthereumClassic_ResearcherFAQs}.

Ether (ETH) ist die native Kryptowährung von Ethereum und wird für Transaktionen innerhalb des Netzwerks verwendet. Es dient auch als Anreiz für diejenigen, die an der Sicherung des Netzwerks durch Mining (bei PoW) oder Validierung (bei PoS) von Transaktionen (siehe \ref{consensus_mechanisms} \textit{\nameref{consensus_mechanisms}}) beteiligt sind \parencite[S. 320-321]{Antonopoulos_MasteringEthereum}. Die Flexibilität von Ethereum und seine Fähigkeit, innovative Lösungen zu unterstützen, haben es zu einer der wichtigsten Plattformen in der Blockchain-Welt gemacht.
