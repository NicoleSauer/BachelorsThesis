\subsection{Ethereum}
\label{sec:ethereum_basics}

Ethereum ist eine der führenden Blockchain-Plattformen und wurde 2015 von Vitalik Buterin, Gavin Wood und anderen entwickelt. Im Gegensatz zu Bitcoin, das hauptsächlich als digitale Währung fungiert, ermöglicht Ethereum die Ausführung von Smart Contracts und die Entwicklung von dezentralen Anwendungen (engl. \textit{decentralized Applications}, kurz \textit{DApps}) \Parencites[S. 720]{Sorge_BitcoinZahlungsmittelDerZukunft}[S. 1-2]{Perez_SmartContractVulnerabilities}. Bis 2022 wurde als Konsensmechanismus Proof-of-Work verwendet, der jedoch 2022 durch Proof-of-Stake ersetzt wurde. Seitdem existieren zwei Versionen von Ethereum: \textit{Ethereum Classic} und \textit{Ethereum 2.0}. Ethereum Classic verwendet Proof-of-Work, während Ethereum 2.0 Proof-of-Stake verwendet \Parencite{EthereumClassic_ResearcherFAQs}.

Ether (ETH) ist die native Kryptowährung von Ethereum und wird für Transaktionen innerhalb des Netzwerks verwendet. Es dient auch als Anreiz für diejenigen, die an der Sicherung des Netzwerks durch Mining (bei PoW) oder Validierung (bei PoS) von Transaktionen beteiligt sind \parencite[S. 320-321]{Antonopoulos_MasteringEthereum}. Die Flexibilität von Ethereum und seine Fähigkeit, innovative Lösungen zu unterstützen, haben es zu einer der wichtigsten Plattformen in der Blockchain-Welt gemacht.
