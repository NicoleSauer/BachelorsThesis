\section{Sicherheit}
\label{sec:sicherheit_basics}

Sicherheit im Internet wird durch verschiedene Protokolle und Algorithmen gewährleistet. Die Kryptografie ist ein wichtiger Bestandteil der Sicherheit im Internet und heutzutage nicht mehr wegzudenken. Die Kryptografie beschreibt die Wissenschaft der Verschlüsselung und Entschlüsselung von Informationen und ist ein Teil der Kryptologie. Die Kryptoanalyse bildet den anderen Teil der Kryptologie und beschäftigt sich mit der Informationsgewinnung aus verschlüsselten Daten, ohne den Schlüssel zu kennen \Parencite[S. 21]{Ertel_AngewandteKryptographie}.


Sie wird verwendet, um die Vertraulichkeit, Integrität und Authentizität von Daten zu gewährleisten. Die Vertraulichkeit wird durch Verschlüsselung erreicht, die Integrität durch Hashfunktionen und die Authentizität durch digitale Signaturen. Man unterscheidet in symmetrischer und asymmetrischer Kryptografie. Bei der symmetrischen Kryptografie wird ein gemeinsamer Schlüssel verwendet, um Nachrichten zu verschlüsseln und zu entschlüsseln. Bei der asymmetrischen Kryptografie werden zwei Schlüssel verwendet, einen öffentlichen und einen privaten. Der öffentliche Schlüssel wird verwendet, um Nachrichten zu verschlüsseln und der private Schlüssel wird verwendet, um Nachrichten zu entschlüsseln. Die asymmetrische Kryptografie wird auch Public-Key-Kryptografie genannt.

