\section{Sicherheit}
\label{sec:sicherheit_im}

Hashfunktionen, digitale Signaturen (wegen Relay), Public-Key-Infrastruktur, symmetrische und asymmetrische Verschlüsselung, Diffie-Hellman, elliptische Kurven Kryptografie \\

\noindent Sicherheit im Internet wird durch verschiedene Protokolle und Algorithmen gewährleistet. Die Kryptografie ist ein wichtiger Bestandteil der Sicherheit im Internet und heutzutage nicht mehr wegzudenken. Sie wird verwendet, um die Vertraulichkeit, Integrität und Authentizität von Daten zu gewährleisten. Die Vertraulichkeit wird durch Verschlüsselung erreicht, die Integrität durch Hashfunktionen und die Authentizität durch digitale Signaturen. Man unterscheidet in symmetrischer und asymmetrischer Kryptografie. Bei der symmetrischen Kryptografie wird ein gemeinsamer Schlüssel verwendet, um Nachrichten zu verschlüsseln und zu entschlüsseln. Bei der asymmetrischen Kryptografie werden zwei Schlüssel verwendet, einen öffentlichen und einen privaten. Der öffentliche Schlüssel wird verwendet, um Nachrichten zu verschlüsseln und der private Schlüssel wird verwendet, um Nachrichten zu entschlüsseln. Die asymmetrische Kryptografie wird auch Public-Key-Kryptografie genannt. \\

\noindent Die Public-Key-Kryptografie wird verwendet, um die Vertraulichkeit und Authentizität von Nachrichten zu gewährleisten. Die Vertraulichkeit wird durch Verschlüsselung mit dem öffentlichen Schlüssel erreicht. Die Authentizität wird durch digitale Signaturen erreicht. Digitale Signaturen werden verwendet, um die Authentizität von Nachrichten zu gewährleisten. Sie werden mit dem privaten Schlüssel des Senders erstellt und mit dem öffentlichen Schlüssel des Senders überprüft. \\

\begin{itemize}
    \item Hashfunktionen
    \item Digitale Signaturen
    \item Public-Key-Infrastruktur
    \item Symmetrische und asymmetrische Verschlüsselung
    \item Diffie-Hellman
    \item Elliptische Kurven Kryptografie
    \item Angriffe auf Instant Messaging
\end{itemize}