\section{Existierende IM-Protokolle}
Es gibt bereits diverse IM-Protokolle, die P2P verwenden. Diese Protokolle werden in diesem Kapitel
vorgestellt und verglichen. Die Protokolle werden in zwei Kategorien unterteilt: Protokolle, die
einen zentralen Server verwenden und Protokolle, die P2P verwenden. Die Protokolle, die einen
zentralen Server verwenden, werden in zwei weitere Kategorien unterteilt: Protokolle, die
einen zentralen Server für die Kommunikation verwenden und Protokolle, die einen zentralen
Server für die Vermittlung verwenden. Die Protokolle, die P2P verwenden, werden in zwei weitere
Kategorien unterteilt: Protokolle, die ein hybrides P2P-Netzwerk verwenden und Protokolle, die
ein reines P2P-Netzwerk verwenden. Die Protokolle werden in den folgenden Abschnitten vorgestellt
und verglichen.

Protokolle mit zentralem Server:

\begin{itemize}
    \item XMPP
    \item Matrix
    \item Telegram
    \item Signal
    \item Threema
    \item WhatsApp
    \item IRC
    \item Skype
\end{itemize}

Protokolle mit P2P:

\begin{itemize}
    \item Tox
    \item Briar
    \item Jami
    \item Ring
    \item Ricochet
    \item Bitmessage
    \item RetroShare
\end{itemize}

XMPP:
Dieses Protokoll ist ein offener Standard für Instant Messaging und basiert auf XML. Es ist ein
Protokoll mit zentralem Server. Es gibt verschiedene XMPP-Server, die miteinander kommunizieren
können. Die Kommunikation zwischen den Servern erfolgt über das Server-zu-Server-Protokoll (S2S).
Die Kommunikation zwischen den Teilnehmern erfolgt über das Client-zu-Server-Protokoll (C2S).

Matrix:
Dieses Protokoll ist ein offener Standard für Instant Messaging und basiert auf JSON. Es ist ein
Protokoll mit zentralem Server. Es gibt verschiedene Matrix-Server, die miteinander kommunizieren
können. Die Kommunikation zwischen den Servern erfolgt über das Server-zu-Server-Protokoll (S2S).
Die Kommunikation zwischen den Teilnehmern erfolgt über das Client-zu-Server-Protokoll (C2S).

Telegram:
Dieses Protokoll ist ein proprietäres Protokoll für Instant Messaging und basiert auf JSON. Es ist
ein Protokoll mit zentralem Server. Es gibt verschiedene Telegram-Server, die miteinander
kommunizieren können. Die Kommunikation zwischen den Servern erfolgt über das Server-zu-Server-
Protokoll (S2S). Die Kommunikation zwischen den Teilnehmern erfolgt über das Client-zu-Server-
Protokoll (C2S).
