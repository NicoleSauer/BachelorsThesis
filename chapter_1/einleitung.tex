\chapter{Einleitung}

% #TODO: Erwähnen, dass hier ein Prototyp entwickelt wird, der die Machbarkeit eines solchen Protokolls aufzeigt

In den letzten Jahren hat sich die digitale Kommunikation signifikant gewandelt. 2013 war das Short Message System (SMS) noch dominierend, doch in den darauf folgenden Jahren verlagerte sich der Fokus deutlich hin zum Mobile Instant Messaging. Die Nutzung von Instant Messaging unter der deutschen Bevölkerung stieg von etwa 24\% im Jahr 2013 auf beeindruckende 73\% im Jahr 2017 \parencite{Hedda_digiKommunikationVeraendert}. 

Aber Instant-Messaging ist nicht gleich Instant-Messaging. Es gibt verschiedene Arten von Instant-Messaging-Diensten, die sich in ihrer Funktionsweise und ihren Eigenschaften unterscheiden. Die bekanntesten sind zentralisierte und dezentrale Instant-Messaging-Dienste. Zentralisierte Dienste werden von einem zentralen Server verwaltet, der die Kommunikation zwischen den Benutzern vermittelt. Dezentrale Dienste hingegen nutzen eine Peer-to-Peer-Infrastruktur, bei der die Kommunikation direkt zwischen den Benutzern stattfindet. Doch 2013 veröffentlichte Edward Snowden Dokumente, die zeigten, dass die Kommunikation von Millionen von Menschen von Geheimdiensten überwacht wurde\parencite{greenwald_NSA}. Dadurch rückten die Themen Sicherheit und Privatsphäre in den Fokus der Öffentlichkeit.

Durch diese Enthüllungen entstand zum Beispiel das Peer-to-Peer-Instant-Messaging-Protokoll \textit{Tox}. Es wurde von einer Gruppe von Entwicklern ins Leben gerufen, die sich zum Ziel gesetzt haben, ein sicheres und leicht zu bedienendes Instant-Messaging-Protokoll zu entwickeln, das die Privatsphäre seiner Benutzer schützt \parencite{tox_about}. Die Entwicklung eines solchen Instant-Messaging-Protokolls ist jedoch nicht leicht. Es müssen viele Aspekte beachtet werden, die in zentralisierten Protokollen nicht vorhanden sind und die die Komplexität des Protokolls erhöhen. Im Verlauf dieser Bachelorarbeit werden verschiedene Aspekte der Peer-to-Peer-Kommunikation, einschließlich Sicherheit, Effizienz und Benutzerfreundlichkeit, beleuchtet. Dabei werden auch existierende Protokolle betrachtet, um eine Grundlage für die Entwicklung eines eigenen Protokolls zu schaffen. Darüber hinaus hat die Blockchain das Potenzial, die Integrität und Authentizität von Daten und Dokumenten in Instant Messaging-Plattformen sicherzustellen. Die Integration einer geeigneten Blockchain könnte die Sicherheit des Protokolls erhöhen.

Ziel dieser Arbeit ist es, einen Prototypen für ein Peer-to-Peer-Instant-Messaging-Protokoll mit Blockchainintegration zu entwickeln, der die Machbarkeit eines solchen Protokolls aufzeigt.