\chapter{Einleitung}
\label{chap:einleitung}

% Motivation aufzeigen: Diese Bachelorarbeit setzt sich mit der Herausforderung auseinander, ein Peer-to-Peer-Instant-Messaging-Protokoll zu entwickeln, das nicht nur die Privatsphäre seiner Benutzer schützt, sondern auch durch die Integration von Blockchain-Technologie die Integrität und Authentizität der Kommunikation gewährleisten kann.

In den letzten Jahren hat sich die digitale Kommunikation signifikant gewandelt. 2013 war das Short Message System (SMS) noch dominierend, doch in den darauf folgenden Jahren verlagerte sich der Fokus deutlich hin zum Mobile-Instant-Messaging. Die Nutzung von Instant-Messaging unter der deutschen Bevölkerung stieg von etwa 24\% im Jahr 2013 auf beeindruckende 73\% im Jahr 2017 \parencite{Hedda_digiKommunikationVeraendert}. Die Gründe für diesen Wandel sind vielfältig. Instant-Messaging ist in der Regel kostenlos, schnell und einfach zu bedienen. Darüber hinaus bietet es viele nützliche Funktionen wie Gruppenchats, Sprach- und Videoanrufe, Dateiübertragung und vieles mehr. Instant-Messaging ist zu einem wichtigen Bestandteil unseres täglichen Lebens geworden. Es ist ein wesentlicher Bestandteil der Kommunikation zwischen Freunden und Familie, aber auch zwischen Kollegen und Geschäftspartnern. 

Aber Instant-Messaging ist nicht gleich Instant-Messaging. Es gibt verschiedene Arten von Instant-Messaging-Diensten, die sich in ihrer Funktionsweise und ihren Eigenschaften unterscheiden. Die bekanntesten sind zentralisierte und dezentrale Instant-Messaging-Dienste. Zentralisierte Dienste werden von einem zentralen Server verwaltet, der die Kommunikation zwischen den Benutzern vermittelt. Dezentrale Dienste hingegen nutzen eine Peer-to-Peer-Infrastruktur, bei der die Kommunikation direkt zwischen den Benutzern stattfindet. 

Zentralisierte Dienste sind in der Regel einfach zu bedienen und bieten eine gute Benutzererfahrung. Sie sind jedoch anfällig für Sicherheitsbedrohungen, da die Kommunikation über einen zentralen Server vermittelt wird. Sollte dieser Server kompromittiert werden, könnte die Kommunikation der Benutzer abgefangen oder manipuliert werden. Darüber hinaus könnten die Betreiber des Dienstes die Kommunikation ihrer Benutzer überwachen und speichern. Dies stellt ein großes Problem für die Privatsphäre der Benutzer dar.

Dezentrale Dienste hingegen haben das Potenzial mehr Sicherheit und Privatsphäre für ihre Nutzer zu bieten, da die Kommunikation direkt zwischen den Benutzern stattfindet. Es gibt keinen zentralen Server, der angegriffen werden könnte. Darüber hinaus können die Betreiber des Dienstes die Kommunikation nicht überwachen, da sie nicht an der Kommunikation beteiligt sind. Dezentrale Dienste sind jedoch in der Regel komplexer und schwieriger zu bedienen als zentralisierte Dienste. Und auch sie sind nicht perfekt. Auch in dezentralen Diensten können Sicherheitslücken auftreten, die die Sicherheit und Privatsphäre der Benutzer gefährden.


Durch die im Jahr 2013 von Edward Snowden veröffentlichten Dokumente wurde deutlich, dass die Kommunikation von Millionen von Menschen von Geheimdiensten überwacht wurde \parencite{greenwald_NSA}. Dadurch rückten die Themen Sicherheit und Privatsphäre in den Fokus der Öffentlichkeit. Durch diese Enthüllungen entstand zum Beispiel das Peer-to-Peer-Instant-Messaging-Protokoll \textit{Tox}. Es wurde von einer Gruppe von Entwicklern ins Leben gerufen, die sich zum Ziel gesetzt haben, ein sicheres und leicht zu bedienendes Instant-Messaging-Protokoll zu entwickeln, das die Privatsphäre seiner Benutzer schützt \parencite{tox_about}. Die Entwicklung eines solchen Instant-Messaging-Protokolls ist jedoch nicht leicht. Es müssen viele Aspekte beachtet werden, die in zentralisierten Protokollen nicht vorhanden sind und die die Komplexität des Protokolls erhöhen. Im Verlauf dieser Bachelorarbeit werden verschiedene Aspekte der Peer-to-Peer-Kommunikation, einschließlich Sicherheit, Effizienz und Benutzerfreundlichkeit, beleuchtet. Dabei werden auch bereits verfügbare Protokolle betrachtet, um eine Grundlage für die Entwicklung eines eigenen Protokolls zu schaffen. Darüber hinaus hat die Blockchain das Potenzial, die Integrität und Authentizität von Daten und Dokumenten in Instant Messaging-Plattformen sicherzustellen, weshalb auch diese Technologie in dieser Arbeit betrachtet wird.

Diese Bachelorarbeit konzentriert sich auf die Herausforderung, ein Peer-to-Peer-Instant-Messaging-Protokoll zu konzipieren. Dabei liegt der Fokus nicht nur darauf, die Privatsphäre der Nutzer zu bewahren, sondern auch die Integrität und Authentizität der Kommunikation durch die Integration von Blockchain-Technologie zu gewährleisten. Das Hauptziel besteht darin, einen Prototypen für ein derartiges Protokoll zu entwickeln, der durch den Einsatz von Blockchain-Technologie die Sicherheit der Kommunikation verbessern soll.
