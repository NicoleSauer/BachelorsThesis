\documentclass[11pt, a4paper]{report}
\pagestyle{headings}
\usepackage[pagestyles]{titlesec}
\usepackage{lipsum}
\usepackage[ngerman]{babel}
\usepackage[utf8]{inputenc}
\usepackage[T1]{fontenc}
\usepackage{graphicx}
\usepackage{geometry}
\usepackage{caption}
\usepackage[hidelinks]{hyperref}
\usepackage{listingsutf8}
\usepackage{float}
\usepackage{setspace}
\usepackage{xcolor}
\usepackage{amsmath}
\usepackage{amssymb}
\usepackage{csquotes}
\usepackage[backend=biber, style=authoryear]{biblatex}
\addbibresource{sources.bib}
\setlength{\bibitemsep}{1em}
\DeclareLabeldate{
    \field{date}
    \field{eventdate}
    \field{origdate}
    \literal{nodate}
}

% Add substitutes for umlauts
\lstset{
    inputencoding=utf8,
    extendedchars=true,
    literate=%
    {Ä}{{\"A}}1
    {Ö}{{\"O}}1
    {Ü}{{\"U}}1
    {ä}{{\"a}}1
    {ö}{{\"o}}1
    {ü}{{\"u}}1
}

% Define Solidity as a language for listings because it is not supported by default
\lstdefinelanguage{Solidity}{
  keywords={address, bool, string, uint, mapping, contract, function, modifier, event, require, assert, bytes},
  keywordstyle=\color{red}\bfseries,
  identifierstyle=\color{black},
  sensitive=false,
  comment=[l]{//},
  morecomment=[s]{/*}{*/},
  commentstyle=\color{gray}\ttfamily,
  stringstyle=\color{violet}\ttfamily,
  morestring=[b]',
  morestring=[b]"
}

% Define the Solidity style for listings
\lstset{
  language=Solidity,
  basicstyle=\small\ttfamily,
  breaklines=true,
  showstringspaces=false,
  numbers=left,
  numberstyle=\tiny\color{gray},
  numbersep=5pt,
  tabsize=2,
  frame=lines,
  showtabs=false,
  showspaces=false,
  extendedchars=true,
  inputencoding=utf8,
  captionpos=b
}

% \makeatletter
% \def\thickhrulefill{\leavevmode \leaders \hrule height 1ex \hfill \kern \z@}
% \def\@makechapterhead#1{%
%   \vspace*{10\p@}%
%   {\parindent \z@ 
%     {\raggedleft \reset@font
%       \fontsize{15ex}{15ex}\selectfont %Problème avec les substitutions...
%       \bfseries\thechapter\par\nobreak}%
%     \par\nobreak
%     \interlinepenalty\@M
%     {\raggedright \Huge \bfseries #1}%
%     \par\nobreak
%     \hrulefill
%     \par\nobreak
%     \vskip 100\p@
%   }}
% \def\@makeschapterhead#1{%
%   \vspace*{10\p@}%
%   {\parindent \z@ 
%     {\raggedleft \reset@font
%       \fontsize{15ex}{15ex}\selectfont %Problème avec les substitutions...
%       \bfseries\vphantom{\thechapter}\par\nobreak}%
%     \par\nobreak
%     \interlinepenalty\@M
%     {\raggedright \Huge \bfseries #1}%
%     \par\nobreak
%     \hrulefill
%     \par\nobreak
%     \vskip 100\p@
%   }}
  
%%%%%%%%%%%%%%%%%%%%%%%%%%%%%% NOTES %%%%%%%%%%%%%%%%%%%%%%%%%%%%%%

% - Keep the standard LaTeX font?
%     NOTE: when using Tahoma font, you have to compile this with XeLaTeX (xelatex) or 
%    LuaLaTeX (lualatex) instead of regular LaTeX.
% - Why Kademlia and not Chord? Kademlia:
% Efficient Asymmetric Routing: Kademlia uses XOR metric-based routing to perform lookups. It's efficient and has an asymmetric routing mechanism, allowing for faster lookups in certain scenarios, especially when there are geographical considerations or when the system has a high rate of node churn.
% Resilience to Churn: Kademlia tends to handle high rates of node joining and leaving more gracefully than some other DHTs, making it more suitable for dynamic networks.
% Decentralization and Redundancy: Kademlia offers decentralized storage and redundancy, ensuring that data is stored in multiple nodes for fault tolerance.
% #TODO: Change signature to a nicer version!

%%%%%%%%%%%%%%%%%%%%%%%%%%%%%%%%%%%%%%%%%%%%%%%%%%%%%%%%%%%%%%%%%%%

\begin{document}
    % Custom title page
    \begin{titlepage}
        \begin{center}
            
            \vspace*{1cm}
            \LARGE
            \textbf{Bachelorarbeit}\\

            \Large
            \bigbreak
            im Studiengang \\
            Wirtschaftsinformatik und digitale Medien\\

            \vspace*{1cm}
            \LARGE
            \textbf{Ein sicheres Peer-to-Peer-Instant-Messaging-Protokoll unter
            Berücksichtigung von Blockchain-Technologie}\\
            
            \large
            \vspace*{1cm}
            vorgelegt von \\
            \vspace*{0.5cm}
            \textbf{Nicole Sauer} \\
            \vspace*{1cm}

            \includegraphics[width=0.2\linewidth]{images/hdm_logo.png} \\

            \vspace*{0.5cm}
            an der Hochschule der Medien Stuttgart \\
            am 29.01.2024 \\
            zur Erlangung des akademischen Grades eines \\
            \textbf{Bachelor of Science}\\
            \vspace*{1.5cm}
            \large 
            \begin{tabular}{ll}
                Erstprüfer & Prof. Dr. Peter Thies \\
                Zweitprüfer & Prof. Dr. Stephan Wilczek \\
            \end{tabular} \\
        \end{center}
    \end{titlepage}
    
    % Change vertical spacing to 1.5 for the rest of the document
    \onehalfspacing

    %%%%%%%%%%%%%%%%%%%%%%%% BEGIN: ToC, LoF, Abstract DE/ENG, Ehrenw. Erkl. %%%%%%%%%%%%%%%%%%%%%%%%
    %\chapter*{Ehrenwörtliche Erklärung}

Hiermit versichere ich, Nicole Sauer, ehrenwörtlich, dass ich die vorliegende
Bachelorarbeit mit dem Titel: „Ein sicheres Peer-to-Peer-Instant-Messaging-Protokoll unter
Berücksichtigung von Blockchain-Technologie“ selbstständig und ohne fremde Hilfe verfasst und keine anderen 
als die angegebenen Hilfsmittel benutzt habe. Die Stellen der Arbeit, die dem Wortlaut oder dem Sinn nach 
anderen Werken entnommen wurden, sind in jedem Fall unter Angabe der Quelle kenntlich gemacht. Ebenso sind alle
Stellen, die mit Hilfe eines KI-basierten Schreibwerkzeugs erstellt oder überarbeitet wurden, kenntlich
gemacht. Die Arbeit ist noch nicht veröffentlicht oder in anderer Form als Prüfungsleistung vorgelegt worden. \\
\\
Ich habe die Bedeutung der ehrenwörtlichen Versicherung und die prüfungsrechtlichen Folgen 
(§ 24 Abs. 2 Bachelor-SPO), der HdM einer unrichtigen oder unvollständigen 
ehrenwörtlichen Versicherung zur Kenntnis genommen.
\\
\\

\noindent\begin{tabular}{ll}
    Leutenbach, den 29.01.2024 & \includegraphics[width=0.25\linewidth]{images/signature.png} \\
    \makebox[6cm]{\hrulefill} & \makebox[6cm]{\hrulefill}\\
    Ort, Datum & Unterschrift\\
\end{tabular}

    \tableofcontents
    %\listoffigures
    %\lstlistoflistings
    %\chapter*{Kurzfassung}

% #TODO: Kurzfassung schreiben


Diese Bachelorarbeit präsentiert die Entwicklungsarbeit an einem innovativen Peer-to-Peer-Instant-Messaging-Protokoll, das auf einer Kombination von Kademlia und Interactive Connectivity Establishment (ICE) basiert. Der Fokus liegt auf der Schaffung eines dezentralen, robusten Peer-to-Peer Kommunikationsprotokolls, das unabhängig von zentralen Servern agiert.

Die Authentifizierung der Teilnehmer erfolgt über öffentliche Schlüssel, die sicher in der Ethereum Blockchain hinterlegt sind. Diese Blockchain-basierte Authentifizierung gewährleistet eine vertrauenswürdige Identitätsüberprüfung, wobei die dezentrale Natur von Ethereum eine hohe Sicherheit und Manipulationssicherheit bietet.

Ein zentraler Beitrag dieser Arbeit ist die Integration von Kademlia, einem verteilten Peer-to-Peer Routing-Algorithmus, und ICE, einem Mechanismus zur Überquerung von Netzwerkgrenzen. Diese Kombination ermöglicht eine effiziente und zuverlässige P2P-Kommunikation, wodurch die Abhängigkeit von zentralen Servern minimiert wird. Die Implementierung wurde darauf ausgerichtet, eine hohe Skalierbarkeit und Robustheit des Systems sicherzustellen.

Die präsentierte Lösung bietet nicht nur eine sichere Peer-to-Peer Kommunikationsumgebung, sondern betont auch die Bedeutung der Blockchain-Technologie für die Authentifizierung in dezentralen Systemen. Das entwickelte Protokoll demonstriert, wie eine Kombination von Ethereum, Kademlia und ICE die Grundlage für eine vertrauenswürdige und effiziente Peer-to-Peer-Instant-Messaging-Infrastruktur schaffen kann.
    %\chapter*{Abstract}

This bachelor's thesis introduces the development of an innovative Peer-to-Peer Instant Messaging protocol, leveraging a combination of Kademlia and Interactive Connectivity Establishment (ICE). The primary focus lies in creating a decentralized, robust Peer-to-Peer communication protocol that operates independently of central servers.

Participant authentication is achieved through public keys securely stored in the Ethereum blockchain. This blockchain-based authentication ensures a trustworthy identity verification process, with Ethereum's decentralized nature providing high security and tamper resistance.

A significant contribution of this work is the integration of Kademlia, a distributed Peer-to-Peer routing algorithm, and ICE, a mechanism for traversing network boundaries. This combination enables efficient and reliable P2P communication, minimizing reliance on central servers. The implementation is designed for high scalability and system robustness.

The presented solution not only offers a secure Peer-to-Peer communication environment but also underscores the importance of blockchain technology for authentication in decentralized systems. The developed protocol demonstrates how a combination of Ethereum, Kademlia, and ICE can form the foundation for a trusted and efficient Peer-to-Peer Instant Messaging infrastructure.
    %%%%%%%%%%%%%%%%%%%%%%%%% END: ToC, LoF, Abstract DE/ENG, Ehrenw. Erkl. %%%%%%%%%%%%%%%%%%%%%%%%%


    %%%%%%%%%%%%%%%%%%%%%%%%%%%%%%%%%%%% BEGIN: Numbered Chapters %%%%%%%%%%%%%%%%%%%%%%%%%%%%%%%%%%%
    \chapter{Einleitung}
\label{chap:einleitung}

% Motivation aufzeigen: Diese Bachelorarbeit setzt sich mit der Herausforderung auseinander, ein Peer-to-Peer-Instant-Messaging-Protokoll zu entwickeln, das nicht nur die Privatsphäre seiner Benutzer schützt, sondern auch durch die Integration von Blockchain-Technologie die Integrität und Authentizität der Kommunikation gewährleisten kann.

In den letzten Jahren hat sich die digitale Kommunikation signifikant gewandelt. 2013 war das Short Message System (SMS) noch dominierend, doch in den darauf folgenden Jahren verlagerte sich der Fokus deutlich hin zum Mobile-Instant-Messaging. Die Nutzung von Instant-Messaging unter der deutschen Bevölkerung stieg von etwa 24\% im Jahr 2013 auf beeindruckende 73\% im Jahr 2017 \parencite{Hedda_digiKommunikationVeraendert}. Die Gründe für diesen Wandel sind vielfältig. Instant-Messaging ist in der Regel kostenlos, schnell und einfach zu bedienen. Darüber hinaus bietet es viele nützliche Funktionen wie Gruppenchats, Sprach- und Videoanrufe, Dateiübertragung und vieles mehr. Instant-Messaging ist zu einem wichtigen Bestandteil unseres täglichen Lebens geworden. Es ist ein wesentlicher Bestandteil der Kommunikation zwischen Freunden und Familie, aber auch zwischen Kollegen und Geschäftspartnern. 

Aber Instant-Messaging ist nicht gleich Instant-Messaging. Es gibt verschiedene Arten von Instant-Messaging-Diensten, die sich in ihrer Funktionsweise und ihren Eigenschaften unterscheiden. Die bekanntesten sind zentralisierte und dezentrale Instant-Messaging-Dienste. Zentralisierte Dienste werden von einem zentralen Server verwaltet, der die Kommunikation zwischen den Benutzern vermittelt. Dezentrale Dienste hingegen nutzen eine Peer-to-Peer-Infrastruktur, bei der die Kommunikation direkt zwischen den Benutzern stattfindet. 

Zentralisierte Dienste sind in der Regel einfach zu bedienen und bieten eine gute Benutzererfahrung. Sie sind jedoch anfällig für Sicherheitsbedrohungen, da die Kommunikation über einen zentralen Server vermittelt wird. Sollte dieser Server kompromittiert werden, könnte die Kommunikation der Benutzer abgefangen oder manipuliert werden. Darüber hinaus könnten die Betreiber des Dienstes die Kommunikation ihrer Benutzer überwachen und speichern. Dies stellt ein großes Problem für die Privatsphäre der Benutzer dar.

Dezentrale Dienste hingegen haben das Potenzial mehr Sicherheit und Privatsphäre für ihre Nutzer zu bieten, da die Kommunikation direkt zwischen den Benutzern stattfindet. Es gibt keinen zentralen Server, der angegriffen werden könnte. Darüber hinaus können die Betreiber des Dienstes die Kommunikation nicht überwachen, da sie nicht an der Kommunikation beteiligt sind. Dezentrale Dienste sind jedoch in der Regel komplexer und schwieriger zu bedienen als zentralisierte Dienste. Und auch sie sind nicht perfekt. Auch in dezentralen Diensten können Sicherheitslücken auftreten, die die Sicherheit und Privatsphäre der Benutzer gefährden.


Durch die im Jahr 2013 von Edward Snowden veröffentlichten Dokumente wurde deutlich, dass die Kommunikation von Millionen von Menschen von Geheimdiensten überwacht wurde \parencite{greenwald_NSA}. Dadurch rückten die Themen Sicherheit und Privatsphäre in den Fokus der Öffentlichkeit. Durch diese Enthüllungen entstand zum Beispiel das Peer-to-Peer-Instant-Messaging-Protokoll \textit{Tox}. Es wurde von einer Gruppe von Entwicklern ins Leben gerufen, die sich zum Ziel gesetzt haben, ein sicheres und leicht zu bedienendes Instant-Messaging-Protokoll zu entwickeln, das die Privatsphäre seiner Benutzer schützt \parencite{tox_about}. Die Entwicklung eines solchen Instant-Messaging-Protokolls ist jedoch nicht leicht. Es müssen viele Aspekte beachtet werden, die in zentralisierten Protokollen nicht vorhanden sind und die die Komplexität des Protokolls erhöhen. Im Verlauf dieser Bachelorarbeit werden verschiedene Aspekte der Peer-to-Peer-Kommunikation, einschließlich Sicherheit, Effizienz und Benutzerfreundlichkeit, beleuchtet. Dabei werden auch bereits verfügbare Protokolle betrachtet, um eine Grundlage für die Entwicklung eines eigenen Protokolls zu schaffen. Darüber hinaus hat die Blockchain das Potenzial, die Integrität und Authentizität von Daten und Dokumenten in Instant Messaging-Plattformen sicherzustellen, weshalb auch diese Technologie in dieser Arbeit betrachtet wird.

Diese Bachelorarbeit konzentriert sich auf die Herausforderung, ein Peer-to-Peer-Instant-Messaging-Protokoll zu konzipieren. Dabei liegt der Fokus nicht nur darauf, die Privatsphäre der Nutzer zu bewahren, sondern auch die Integrität und Authentizität der Kommunikation durch die Integration von Blockchain-Technologie zu gewährleisten. Das Hauptziel besteht darin, einen Prototypen für ein derartiges Protokoll zu entwickeln, der durch den Einsatz von Blockchain-Technologie die Sicherheit der Kommunikation verbessern soll.

    \chapter{Grundlagen}
\label{chap:grundlagen}

\section{Instant Messaging}
% Was ist Instant Messaging?
% Wofür wird es verwendet?
% Welche Anwendungen gibt es? (WhatsApp, Signal, Telegram, Threema, etc.) -> die meisten sind zentralisiert
% Welche Probleme gibt es bei zentralisierten Anwendungen?
% Client-Server-Architektur und Peer to Peer als Überleitung zur nächsten Sektion verwenden


Instant Messaging bezeichnet eine Form der Kommunikation, bei der Nachrichten in Echtzeit zwischen zwei oder mehreren Personen über das Internet ausgetauscht werden können \Parencite[S. 69]{nist_mobileDeviceForensics}. Diese Form der digitalen Kommunikation ermöglicht es Nutzern, sofortige Nachrichten, Bilder und andere Mediendateien auszutauschen. Instant Messaging-Dienste umfassen eine Vielzahl von Plattformen und Anwendungen, die es Benutzern ermöglichen, miteinander zu kommunizieren, sei es eins zu eins oder in Gruppenchats. Instant-Messaging-Dienste reichen von plattformübergreifenden Anwendungen wie \textit{WhatsApp}, \textit{Signal} und \textit{Telegram} bis hin zu spezialisierten Unternehmenslösungen wie \textit{Slack} oder \textit{Microsoft Teams}. Die Vielfalt an Funktionen in Instant-Messaging-Plattformen ist groß. Neben einfachen Textnachrichten können Benutzer zum Beispiel mit der App \textit{WhatsApp} Emojis, Aufkleber, GIFs und Sprachnachrichten teilen, was die Kommunikation dynamisch und ausdrucksstark gestaltet \Parencite{whatsapp_funktionen}. Gruppenchats ermöglichen es mehreren Personen, in einer einzigen Unterhaltung zusammenzukommen, was die Zusammenarbeit, soziale Interaktion und Informationsverbreitung erleichtert.
Sicherheit und Datenschutz sind in der Welt des Instant Messaging von entscheidender Bedeutung. Verschlüsselungstechnologien werden verwendet, um die Vertraulichkeit der Nachrichten zu gewährleisten und die Privatsphäre der Nutzer zu schützen. Die Authentifizierung von Benutzern, End-to-End-Verschlüsselung und andere Sicherheitsmechanismen sind unerlässlich, um die Integrität der Kommunikation zu gewährleisten.
Instant Messaging-Anwendungen können auf verschiedene Arten strukturiert sein, wobei zwei Hauptansätze hervorstechen: die Client-Server-Architektur und das Peer-to-Peer-Modell.
Die Client-Server-Architektur ist bei vielen gängigen Instant Messaging-Diensten verbreitet. Hierbei fungiert ein zentraler Server als Vermittler zwischen den Benutzergeräten. Die Nachrichten werden von den Clients (den Benutzergeräten) an den Server gesendet, der sie dann an die Empfänger weiterleitet. Dieser Ansatz bietet eine einfache Verwaltung der Kommunikation, zentralisierte Datenverwaltung und ermöglicht Dienstleistungen wie das Offline-Speichern von Nachrichten. Beispiele hierfür sind Plattformen wie WhatsApp und Signal, die diese Architektur nutzen, um Nachrichten zwischen Benutzern zu vermitteln.
Im Gegensatz dazu basiert das Peer-to-Peer-Modell (kurz: P2P) auf direkten Verbindungen zwischen den Benutzergeräten ohne die Notwendigkeit eines zentralen Servers. Jedes Gerät agiert sowohl als Client als auch als Server, wodurch die Kommunikation direkt zwischen den Teilnehmern stattfindet. Diese Struktur bietet potenzielle Vorteile in Bezug auf Datenschutz, da die Nachrichten nicht über einen zentralen Server geleitet werden müssen. P2P-IM-Anwendungen wie beispielsweise BitTorrent Chat oder Tox setzen auf diese Architektur, um eine dezentralisierte und möglicherweise sicherere Kommunikationsumgebung zu schaffen.
Beide Ansätze haben ihre eigenen Vor- und Nachteile. Die Client-Server-Architektur ermöglicht eine einfachere Verwaltung und Zuverlässigkeit, birgt jedoch potenzielle Datenschutzrisiken, da Daten zentralisiert gespeichert werden. Auf der anderen Seite bietet das Peer-to-Peer-Modell potenziell mehr Privatsphäre und Sicherheit, aber es könnte Schwierigkeiten bei der Skalierbarkeit und Verlässlichkeit geben, da es keine zentrale Instanz gibt, die die Kommunikation steuert. Die Wahl zwischen diesen Architekturen hängt von den spezifischen Anforderungen, Sicherheitsbedenken und dem Nutzungskontext der Instant Messaging-Anwendung ab.




Als Alternative zur Client-Server-Architektur gibt es auch Peer-to-Peer-Instant-\\Messaging-Anwendungen, bei denen die Kommunikation direkt zwischen den Teilnehmern stattfindet. Diese Anwendungen sind in der Regel dezentralisiert und verwenden keine zentrale Instanz, um die Kommunikation zu vermitteln. Stattdessen werden die Nachrichten direkt zwischen den Teilnehmern ausgetauscht. Die dezentrale Natur dieser Anwendungen macht sie unabhängig von zentralen Servern und ermöglicht es den Benutzern, die Kontrolle über ihre Daten zu behalten. Die dezentrale Architektur bietet auch eine höhere Ausfallsicherheit, da die Kommunikation nicht von einem zentralen Server abhängig ist. Die dezentrale Architektur ist jedoch auch mit einigen Herausforderungen verbunden. Die dezentrale Natur der Anwendung macht es schwieriger, die Identität der Benutzer zu verifizieren, da es keine zentrale Instanz gibt, die die Authentizität der Benutzer überprüfen kann. Darüber hinaus ist es schwieriger, die Integrität der Nachrichten zu gewährleisten, da die Nachrichten nicht über einen zentralen Server weitergeleitet werden. Die dezentrale Architektur ist auch anfälliger für Angriffe, da die Kommunikation nicht über einen zentralen Server vermittelt wird.

\section{Peer-to-Peer-Technologie}
\label{subsec:peer_to_peer_technologie}


Peer-to-Peer-Technologien können in zwei Kategorien unterteilt werden: \textit{Peer-to-Peer-Anwendungen} und \textit{Peer-to-Peer-Infrastrukturen} \parencite[S. 730]{Khatibi_StructuredUnstructuredP2P}. 

Die Kategorie der Peer-to-Peer-Anwendungen umfasst den Dienst der \textit{Inhaltsverteilung}, bei dem die Teilnehmer Inhalte wie Musik, Videos oder andere Dateien direkt untereinander austauschen \Parencite[730-731]{Khatibi_StructuredUnstructuredP2P}. 

Daher werden viele Peer-to-Peer schnell mit Filesharing in Verbindung bringen, da diese Technologie in der Vergangenheit vor allem dafür genutzt wurde. Das bekannteste Beispiel ist das Filesharing-Netzwerk \textit{Napster}, das 1999 von Shawn \enquote{Napster} Fanning entwickelt wurde. \textit{Napster} war das erste weit verbreitete Filesharing-Netzwerk, das auf Peer-to-Peer-Technologie basierte. Es ermöglichte den Austausch von Musikdateien zwischen den Teilnehmern. Die Musikdateien wurden dabei auf den Computern der Teilnehmer gespeichert und konnten von anderen Teilnehmern heruntergeladen werden. Da diese Art des Datenaustauschs oftmals illegal war, wurde Napster 2001 aufgrund von Urheberrechtsverletzungen abgeschaltet \parencite[S. 55-57]{Mahlmann_P2PNetzwerke}.

Die zweite Kategorie ist die Peer-to-Peer-Infrastruktur, welche die Peer-to-Peer-Netz-\\werke umfasst, die für die Kommunikation zwischen den Teilnehmern des Netzwerks verwendet werden \parencite[S. 730-731]{Khatibi_StructuredUnstructuredP2P}. Diese Art von Peer-to-Peer-Netzwerken wird in dieser Arbeit behandelt.

Im Bereich des Instant Messaging stellt das Peer-to-Peer-Modell eine dezentrale Struktur dar, die die Kommunikation zwischen den Nutzern eines Instant-Messaging-Dienstes ermöglicht. Im Gegensatz zum Client-Server-Ansatz, bei dem ein zentraler Server die Kommunikation steuert, ermöglicht das Peer-to-Peer-Netzwerk eine direkte Kommunikation zwischen den Teilnehmern. Beide Modelle haben ihre Vor- und Nachteile. Während das Client-Server-Modell eine zentrale Instanz erfordert, um die Kommunikation zu verwalten, ist das Peer-to-Peer-Netzwerk dezentralisiert und benötigt eine solche Instanz nicht. Die Implementierung und Wartung eines Client-Server-Modells sind im Vergleich zu einem komplexeren und aufwendigeren Peer-to-Peer-Netzwerk einfacher. Das Client-Server-Modell ist weniger flexibel, da es von einer zentralen Instanz abhängt, während das Peer-to-Peer-Netzwerk aufgrund des Fehlens dieser Instanz flexibler ist. Ein weiterer Unterschied betrifft die Skalierbarkeit: Das Client-Server-Modell ist durch die Kapazität des Servers begrenzt, während das Peer-to-Peer-Netzwerk auf die Kapazität der Teilnehmer zurückgreift, was seine Skalierbarkeit verbessert. Hinsichtlich der Sicherheit ist das Client-Server-Modell weniger robust, da es auf eine zentrale Instanz angewiesen ist, während das Peer-to-Peer-Netzwerk als sicherer gilt, da es ohne eine solche Abhängigkeit auskommt \parencite[S. 6-8]{Mahlmann_P2PNetzwerke}.


\subsection{Typen von Peer-to-Peer-Netzwerken}

Nicht jedes Peer-to-Peer-Netzwerk ist gleich aufgebaut. Es gibt verschiedene Typen von Peer-to-Peer-Netzwerken, die sich in ihrer Struktur und Funktionsweise unterscheiden. Abbildung \ref{p2p_typen} zeigt die zwei Haupttypen von Peer-to-Peer-Netzwerken: unstrukturierte und strukturierte Netzwerke \parencite[S. 362-363]{Luntovskyy_ModRechnernetze}.

\begin{center}
    \captionsetup{type=figure}
    \includegraphics[width=1\linewidth]{images/p2p_typen.png}
    \captionof{figure}{Typen von Peer-to-Peer-Netzwerken (in Anlehnung an \cite[S. 363]{Luntovskyy_ModRechnernetze})}
    \label{p2p_typen}
\end{center}

\noindent Unstrukturierte und strukturierte Peer-to-Peer-Netzwerke sind unterschiedliche Ansätze zur Organisation von Knoten (engl. \textit{Nodes}) und Ressourcen in dezentralen Netzwerken.

Unstrukturierte Netzwerke sind charakterisiert durch ihre fehlende explizite Organisationsstruktur, was eine einfache Konnektivität ermöglicht. \textit{Typ A} in Abbildung \ref{p2p_typen} zeigt ein zentralisiertes Netzwerk, was bedeutet, dass alle Teilnehmer mit einem zentralen Server verbunden sind. Als Beispiel für diese Form des Peer-to-Peer dient \textit{Napster}. Bei \textit{Napster} gab es mehrere Server, die die Dateien der Teilnehmer indizierten. Die Teilnehmer konnten Dateien von anderen Teilnehmern herunterladen, indem sie eine Anfrage an einen der Server stellten, der dann die IP-Adresse des Teilnehmers zurückgab, der die gesuchte Datei zur Verfügung stellte \parencite[S. 171]{Saroiu_MeasuringAndAnalyzingNapsterAndGnutellaHosts}. Diese Form ermöglicht eine schnelle und effiziente Suche nach Ressourcen, da die Ressourcen zentral verwaltet werden, aber die Abhängigkeit von einem zentralen Server macht das Netzwerk nicht skalierbar und anfällig für Ausfälle \parencite[S. 732]{Khatibi_StructuredUnstructuredP2P}.
Bei \textit{Typ B} handelt es sich um ein reines Peer-to-Peer-Netzwerk, bei dem die Teilnehmer direkt miteinander verbunden sind und jeder sowohl als Client als auch als Server fungiert \parencite[S. 732]{Khatibi_StructuredUnstructuredP2P}. Ein Beispiel für diese Form des Peer-to-Peer ist \textit{Gnutella}. Bei \textit{Gnutella} gab es keine zentrale Instanz, die die Ressourcen der Teilnehmer indizierte. Die Suche nach Ressourcen oder Informationen erfolgt durch Broadcasts oder zufällige Weiterleitungen, was jedoch zu ineffizienten Suchprozessen führen kann, da keine klare Routing-Struktur vorhanden ist \parencite[S. 171]{Saroiu_MeasuringAndAnalyzingNapsterAndGnutellaHosts}. Beim dritten und letzten Typ (\textit{Typ C}) der unstrukturierten Netzwerke handelt es sich um ein hybrides Peer-to-Peer-Netzwerk, das Elemente aus den beiden anderen Typen kombiniert. In einem hybriden Netzwerk gibt es besondere Knoten, die die Funktionen eines Servers, wie beispielsweise Indexierung der Ressourcen, für eine bestimmte Gruppe von Teilnehmern übernehmen. Diese Knoten werden als Superknoten (engl. \textit{Super Nodes}) bezeichnet. Die Super Nodes selbst sind untereinander dezentralisiert miteinander verbunden. Ein Beispiel für diese Form des Peer-to-Peer ist \textit{Gnutella2} \parencite[S. 732]{Khatibi_StructuredUnstructuredP2P}. 

Strukturierte Peer-to-Peer-Netzwerke hingegen weisen klare Regeln und Algorithmen zur Organisation der Knoten auf. Diese Netzwerke verfügen über eine explizite Organisationsstruktur, sei es eine Ringstruktur, k-bucket basierte Systeme oder andere, die es ermöglichen, effizientes Routing und eine optimierte Ressourcenverwaltung zu erreichen. Durch diese klar definierte Struktur sind strukturierte Netzwerke oft stabiler und bieten eine effizientere Ressourcenlokalisierung im Vergleich zu ihren unstrukturierten Gegenstücken. Allerdings kann diese Stabilität auf Kosten von Flexibilität und Anpassungsfähigkeit gehen, da Änderungen in der Netzwerktopologie oder hohe Dynamik der Knoten schwerer zu handhaben sind \parencite[S. 40]{Vu_P2PComputing}.


\subsection{Problemstellung und mögliche Lösungen}
\label{subsec:problemstellung_und_moegliche_loesungen}

Leider bringen Peer-to-Peer-Netzwerke auch einige Probleme mit sich. Eines der Probleme stellen die \textit{Network Address Translators} (kurz: \textit{NATs}) dar. NATs sind dafür zuständig, private IP-Adressen in öffentliche IP-Adressen umzuwandeln und umgekehrt. Sie werden in Routern oder Gateways eingesetzt und dienen dazu, den Zugang von Geräten im lokalen Netzwerk, welche private IP-Adressen verwenden, zum Internet zu ermöglichen, indem sie den Datenverkehr zwischen dem lokalen Netzwerk und dem externen Netzwerk, wie dem Internet, verwalten. Peer-to-Peer Verbindungen stoßen bei Network Address Translators oft auf Probleme, was daran liegt, dass NATs normalerweise nicht erlauben, dass externe Geräte direkt mit internen Geräten kommunizieren. Zudem werden Ports dynamisch für ausgehenden Traffic zugewiesen, was das Weiterleiten eingehender Verbindungen erschwert. Symmetrische NATs verschärfen dieses Problem, da sie für ausgehende Verbindungen eine eindeutige Kombination von IP-Adresse und Port verwenden, die sich bei jeder neuen Verbindung ändert \Parencite[S. 1-9]{rfc2663_NAT_Terminology}.

NATs stellen also eine große Herausforderung für Peer-to-Peer-Netzwerke dar, da sie die direkte Kommunikation zwischen den Teilnehmern erschweren, wenn sie sich hinter einem NAT befinden. Um dieses Problem zu lösen, gibt es verschiedene Lösungsansätze. Einer davon ist das \textit{Relaying}. Das Verb \textit{to relay} kann mit \textit{weiterleiten} oder \textit{weitergeben} übersetzt werden. Beim Relaying wird also ein Server als Vermittler zwischen den Teilnehmern verwendet, der die Nachrichten zwischen den Teilnehmern weiterleitet. Damit sich die Teilnehmer mit dem Server verbinden können, muss dieser eine öffentliche IP-Adresse haben, die jedem Teilnehmer bekannt ist. Dieser Ansatz ist einfach zu implementieren, erfordert aber einen zentralen Server, der diese Aufgabe übernimmt. Zudem ist er nicht sehr effizient, da der Server die Nachrichten zwischen den Teilnehmern weiterleiten muss, was zu einer höheren Latenz führen kann. Dazu kommt, dass ein solcher Server einen \textit{Single Point of Failure} oder zu Deutsch \textit{einzelner Ausfallpunkt} darstellt, da die Kommunikation zwischen den Teilnehmern unterbrochen wird, wenn der Server ausfällt.

Ein weiterer Ansatz ist die \textit{Connection Reversal}. Bei der Connection Reversal wird ein \textit{Rendezvous-Server} verwendet, um eine Verbindung zwischen den Teilnehmern herzustellen. Der Teilnehmer hinter dem NAT verbindet sich mit dem Rendezvous-Server und teilt diesem seine öffentliche IP-Adresse und Port mit. Der andere Teilnehmer verbindet sich ebenfalls mit dem Rendezvous-Server und erhält die IP-Adresse und Port des  Teilnehmers hinter dem NAT. Bei dieser Technik darf sich nur einer der Teilnehmer hinter einem NAT befinden.
\textit{Hole Punching} beschreibt einen weiteren Lösungsansatz. Zwei Geräte, die eine direkte Verbindung miteinander aufbauen möchten, initiieren gleichzeitig eine Verbindung zu einem Server, der sich außerhalb des NATs befindet. Der Server sammelt die IP-Adressen und Ports der beiden Geräte und leitet diese an die jeweils andere Partei weiter. Die beiden Geräte versuchen dann, eine Verbindung zueinander herzustellen, indem sie gleichzeitig Datenpakete an die IP-Adresse und den Port des anderen Geräts senden. Dabei wird versucht, das NAT dazu zu bringen, die Verbindung zu öffnen, indem es die ankommenden Pakete als Antwort auf die ausgehenden Pakete erkennt. Wenn dies gelingt, wird ein temporäres Loch im NAT geöffnet, das es den Geräten ermöglicht, direkt miteinander zu kommunizieren. Diese Technik erfordert eine präzise Koordination und die Fähigkeit der beiden Geräte zur gleichen Zeit Datenpakete zu senden und zu empfangen. Zudem ist es nicht immer möglich, ein temporäres Loch im NAT zu öffnen, da es von der Implementierung des NATs abhängt. Eine Abwandlung vom Hole Punching ist die \textit{Port Number Prediction}. Hierbei wird versucht, die Portnummer vorherzusagen, die das NAT für die Verbindung verwenden wird. Durch Beobachtung und Analyse vorheriger Verbindungen wird versucht Muster oder Trends in der Art und Weise zu erkennen, wie Portnummern zugewiesen werden. Dies könnte auf bestimmte Algorithmen oder Verhaltensweisen des Systems hinweisen, woraus dann die Portnummer vorhergesagt werden kann. Diese Technik ist jedoch nicht immer zuverlässig, da es rein auf Annahmen basiert und das Risiko besteht, dass sich das Portzuweisungsmuster jederzeit ändern könnte \Parencite[S. 7-21]{rfc5128_P2P_NATs}.

Um diese Problematik von Peer-to-Peer-Netzwerken zu lösen, können verschiedene Protokolle zum Einsatz kommen. Eines dieser Protokolle ist \textit{STUN} (kurz für \textit{Session Traversal Utilities for NAT}). STUN ist ein Netzwerkprotokoll, das es Geräten, die sich hinter einem NAT befinden, ermöglicht, ihre öffentliche IP-Adresse und Port zu ermitteln. Es bietet an sich keine Möglichkeit für eine Umgehung des NATs, sondern ist dafür gedacht, als eines von mehreren Werkzeugen verwendet zu werden, um ein NAT zu umgehen. Mittels STUN lässt sich nur ermitteln, ob sich ein Gerät hinter einem NAT befindet und wenn dies zutrifft, welche IP-Adresse und Port es verwendet \parencite[S. 4]{rfc8489_STUN}.

\textit{TURN} (kurz für \textit{Traversal Using Relays around NAT}) ist ein weiteres Netzwerkprotokoll, das im Zusammenhang mit NATs verwendet werden kann. Aus der Spezifikation ist zu entnehmen, dass TURN ein Protokoll ist, das es Geräten, die sich hinter einem NAT befinden, ermöglicht, eine Verbindung zu einem anderen Gerät herzustellen, indem es einen Server als Vermittler verwendet (siehe weiter oben in \ref{subsec:problemstellung_und_moegliche_loesungen} \nameref{subsec:problemstellung_und_moegliche_loesungen}). Das funktioniert auch, wenn sich beide Geräte hinter einem NAT befinden \parencite[S. 7]{rfc8656_TURN}.

\textit{ICE} (kurz für \textit{Interactive Connectivity Establishment}) ist ein Framework, das mehrere Techniken kombiniert, um eine Verbindung zwischen zwei Endpunkten herzustellen, die sich hinter NATs befinden. Es verwendet sowohl STUN als auch TURN, um die öffentliche IP-Adresse und Port eines Geräts zu ermitteln und den Datenverkehr über einen Server zu leiten \Parencite[S. 6]{rfc8445_ICE}.


\subsection{Overlay-Netzwerke}
\label{subsec:overlay_netzwerke}

Peer-to-Peer-Netzwerke können als Overlay-Netzwerke betrachtet werden. Ein Overlay-Netzwerk ist ein virtuelles Netzwerk, das, wie in Abbildung \ref{overlay_network} (\textit{\nameref{overlay_network}}) zu sehen ist, über ein physisches Netzwerk gelegt wird.

\begin{center}
    \captionsetup{type=figure}
    \includegraphics[width=0.9\linewidth]{images/overlay_network.png}
    \captionof{figure}{Peer-to-Peer-Overlay-Netzwerk mit darunterliegendem physischen Netzwerk \parencite{Kunzmann_OverlayNetworksImageSource}}
    \label{overlay_network}
\end{center}

\noindent In diesem Fall ist das physische Netzwerk das Internet. Das Overlay-Netzwerk ist eine logische Struktur, die es ermöglicht, die Kommunikation zwischen den Teilnehmern zu organisieren. Es besteht aus einer Reihe von Knoten, die über eine logische Verbindung miteinander verbunden sind. Die Verbindungen zwischen den Knoten werden durch Routing-Algorithmen verwaltet \parencite{Lua_P2POverlayNetworksPaper}.

Beispiele für diese Routing-Algorithmen sind \textit{Kademlia}, \textit{Chord} und \textit{Pastry}. Diese drei Algorithmen verwenden sogenannte \textit{Distributed Hash Tables} (zu Deutsch: \textit{Verteilte Hashtabellen}), um die Knoten zu verwalten. Distributed Hash Tables (kurz: \textit{DHTs}) sind verteilte Datenstrukturen, die in Peer-to-Peer-Netzwerken verwendet werden, um effizient Schlüssel-Wert-Paare zu speichern und abzurufen. Anders als herkömmliche zentralisierte Datenbanken oder Speicherlösungen, welche die Daten auf einem zentralen Server speichern, benötigen DHTs keinen solchen Server zur Speicherung oder Verwaltung der Daten. Sie funktionieren auf Basis von Hashfunktionen (siehe \ref{subsec:integritaet_signatur} \textit{\nameref{subsec:integritaet_signatur}}), die einen Schlüssel in einen eindeutigen Hash umwandeln. Diese Hashes dienen als Adressen, um zu bestimmen, wo die entsprechenden Daten im Netzwerk gespeichert sind. Die Daten werden über verschiedene Peers im Netzwerk verteilt, wobei jeder Peer nur einen Teil der Daten basierend auf seinem Verantwortungsbereich speichert. Diese Algorithmen ermöglichen es, Peers im Netzwerk zu finden, die für die Speicherung oder Abfrage von Daten zuständig sind, selbst wenn sich die Netzwerktopologie ständig verändert. Ein großer Vorteil von DHTs ist ihre Skalierbarkeit. Sie können mit der Netzwerkgröße wachsen, ohne an Effizienz zu verlieren. Neue Peers können nahtlos hinzugefügt werden, und die Struktur der DHT passt sich dynamisch an Veränderungen im Netzwerk an \parencites{Stoica_Chord}{Rowstron_Pastry}{Maymounkov_Kademlia}[S. 43-46]{Balakrishnan_LookingUpDataInP2PSystems}.


\subsection{Kademlia vs. Chord vs. Pastry}

\textit{Kademlia} ist ein Routing-Algorithmus, der auf einer k-Bucket-Struktur basiert. Das \textit{k} in k-Bucket steht für die Anzahl der Buckets, die verwendet werden. Sie enthalten eine Liste von Knoten für verschiedene Schlüsselbereiche basierend auf ihrer Nähe. Die Distanz wird durch die XOR-Distanzen der IDs berechnet wird. Die Verbindungen zwischen den Knoten sind asymmetrisch, das bedeutet, dass ein Knoten  \textit{A} eine Verbindung zu einem anderen Knoten \textit{B} haben kann, aber deshalb \textit{B} keine Verbindung zu \textit{A} haben muss. Die Knoten speichern Informationen über andere Knoten in ihren k-Buckets, wobei jeder Bucket für einen bestimmten Schlüsselbereich verantwortlich ist. Bei der Suche nach einem bestimmten Schlüssel erfolgt das Routing durch die XOR-Entfernung, wodurch die nächsten Knoten für diesen Schlüssel gefunden werden. Dieses Verfahren ermöglicht eine logarithmische Anzahl von Schritten für die Suche und bietet eine robuste Struktur, die gut mit dynamischen Netzwerkänderungen umgehen kann \parencite[S. 1-2]{Maymounkov_Kademlia}.

\textit{Chord} ist ebenfalls ein Routing-Algorithmus, basiert allerdings auf einer Ringstruktur. Die Knoten sind in einem Ring angeordnet und jeder Knoten ist für einen bestimmten Schlüsselbereich verantwortlich. Die Verbindungen zwischen den Knoten sind durch ihren Platz im Ring definiert, wobei jeder Knoten eine Verbindung zu seinem nächsten Nachbarn im Uhrzeigersinn hat. Bei der Suche nach einem bestimmten Schlüssel durchläuft eine Anfrage einen logarithmischen Pfad im Ring, wobei die Knoten auf dem Weg begrenzte Informationen über andere Knoten behalten, um Anfragen weiterzuleiten. Dieses Modell ist recht einfach und effizient für viele Anwendungsfälle, aber es könnte anfällig sein für Engpässe oder längere Suchzeiten, insbesondere wenn das Netzwerk dynamisch ist und sich die Konfiguration häufig ändert \parencite[S. 1-3]{Stoica_Chord}.

% #TODO: Pastry entfernen
\textit{Pastry} ist ein weiterer Routing-Algorithmus, der auf einer eindimensionalen Struktur basiert. Die Knoten sind in einem eindimensionalen Adressraum angeordnet, wobei jeder Knoten für einen bestimmten Schlüsselbereich verantwortlich ist. Die Verbindungen zwischen den Knoten werden durch eine gemeinsame Präfix-Länge definiert, wobei jeder Knoten eine Verbindung zu seinem nächsten Nachbarn hat. Jeder Knoten besitzt drei Listen: eine \textit{Routing-Tabelle}, eine \textit{Nachbarn-Liste} und eine \textit{Blatt-Liste}. Die Routing-Tabelle enthält Informationen über andere Knoten im Netzwerk, die für bestimmte Präfixe verantwortlich sind. Die Nachbarn-Liste enthält Informationen über die Knoten im Netzwerk, die am nächsten an diesem Knoten liegen. Die Blatt-Liste enthält Informationen über Knoten, die am nächsten und am weitesten von diesem Knoten entfernt sind. Dies hilft bei der Suche nach einem bestimmten Schlüssel, der nicht durch die Routing-Tabelle abgedeckt wird. Bei der Suche wird nach einem Knoten gesucht, der die größte Übereinstimmung im ID-Präfix besitzt. Dieser Knoten wird dann die Anfrage an den nächsten Knoten weiterleiten, der eine größere Übereinstimmung im ID-Präfix besitzt. Dieser Vorgang wird wiederholt, bis der Knoten gefunden wird, der für den Schlüssel verantwortlich ist. Dieses Verfahren ermöglicht eine logarithmische Anzahl von Schritten für die Suche und bietet eine gute Balance zwischen Effizienz und Skalierbarkeit \parencite{Rowstron_Pastry}.

Insgesamt bieten alle drei Algorithmen Lösungen für die mögliche Anwendung in einem Instant-Messaging-Kontext. Je nach Anwendungsfall können sie unterschiedliche Vorteile bieten. Kademlia ist gut geeignet für große, dynamische Netzwerke, Chord für statischere Netzwerke und Pastry für mittelgroße Netzwerke mit moderater Dynamik. Für den Anwendungsfall des Instant-Messaging ist die effektive Bewältigung von \textit{Churn} von entscheidender Bedeutung. Churn bezieht sich auf die häufigen Ein- und Austritte von Teilnehmern in einem Peer-to-Peer-Netzwerk. In einem Instant-Messaging Kontext bedeutet dies, dass Benutzer sich ständig anmelden oder abmelden. Ein Protokoll, das gut mit Churn umgehen kann, ist entscheidend, um eine zuverlässige und nahtlose Kommunikation zu gewährleisten. Das richtige Handling von Churn ist daher ein Schlüsselfaktor für die Leistungsfähigkeit und Stabilität eines Instant-Messaging-Protokolls \parencite[S. 316-317]{Peris_KademliaChurn}.


\subsection{Angriffe auf Peer-to-Peer-Netzwerke}

Peer-to-Peer-Netzwerke sind anfällig für verschiedene Arten von Angriffen. Diese Angriffe können in zwei Kategorien unterteilt werden: \textit{Angriffe auf die Integrität} und \textit{Angriffe auf die Verfügbarkeit}.

\subsubsection{Angriffe auf die Integrität}
\label{subsubsec:sybil_or_eclipse_attack_p2p}

Bei Angriffen auf die Integrität geht es darum, die Integrität der Daten zu gefährden, die im Netzwerk gespeichert sind. Ein Beispiel für einen solchen Angriff ist der \textit{Sybil-Angriff}. Bei diesem Angriff erstellt ein einzelner Angreifer mehrere Identitäten, um die Kontrolle über das Netzwerk zu erlangen \parencite[S. 251]{Douceur_SybilAttack}. Nach einer erfolgreichen Übernahme von Teilen des Netzwerks, besteht die Möglichkeit der Durchführung eines \textit{Eclipse}-Angriffs \parencite[S. 13-15]{Baptiste_AttacksOnP2PNetworks}. Bei einem \textit{Eclipse}-Angriff wird versucht, einen Peer von allen anderen Peers im Netzwerk zu isolieren. Dies wird erreicht, indem bereits durch den vorhergegangenen Sybil-Angriff die Mehrheit der Peers kontrolliert wird. Der Angreifer kann dann die Verbindungen des Opfers zu anderen Peers im Netzwerk verhindern, indem er die Verbindungen zu anderen Peers kontrolliert \Parencite[S. 14]{Baptiste_AttacksOnP2PNetworks}. 


\subsubsection{Angriffe auf die Verfügbarkeit}
\label{subsubsec:denial_of_service_attack_p2p}

Bei Angriffen auf die Verfügbarkeit geht es darum, die Verfügbarkeit des Netzwerks zu gefährden. Ein Beispiel für einen solchen Angriff ist der \textit{Denial-of-Service-Angriff}. Mit einem \textit{Denial-of-Service}-Angriff (kurz: \textit{DoS-Angriff}) wird versucht, die Verfügbarkeit eines Dienstes zu beeinträchtigen, indem die Ressourcen des Dienstes erschöpft werden, sodass diese ausfallen \parencite{Bicakci_DoSAttacks}. In einem Peer-to-Peer-Netzwerk kann ein DoS-Angriff auf verschiedene Arten durchgeführt werden. Eine Möglichkeit ist es, einen Peer mit Anfragen zu überfluten, um ihn zu überlasten. Eine andere Möglichkeit ist es, einen Peer mit gefälschten Informationen zu überfluten, um ihn zu täuschen. Beide Methoden führen dazu, dass der Peer nicht mehr in der Lage ist, seine Aufgaben zu erfüllen, was zu einem Ausfall des Dienstes führt. Dieser Angriff wird noch effektiver, wenn er von mehreren Angreifern gleichzeitig durchgeführt wird. Dies wird als \textit{Distributed Denial-of-Service}-Angriff (kurz: \textit{DDoS-Angriff}) bezeichnet. Bei einem DDoS-Angriff können mehrere Peers gleichzeitig mit Anfragen überflutet werden, was es schwieriger macht, den Angriff zu stoppen \Parencite[S. 6]{Baptiste_AttacksOnP2PNetworks}. Aber auch ein erfolgreicher Sybil-Angriff kann dazu führen, dass die Verfügbarkeit des Netzwerks gefährdet wird. Wenn ein Angreifer die Mehrheit der Peers kontrolliert, kann er die Verbindungen zu anderen Peers kontrollieren. Die Kommunikation zwischen den Peers kann dann durch falsches Routing verlangsamt oder auch ganz verhindert werden \parencite[S. 13]{Baptiste_AttacksOnP2PNetworks}.





\section{Blockchain-Technologie}
\label{sec:blockchain_basics}

%Als Grundlage für Blockchains dient die \textit{Distributed Ledger Technology} (DLT). Was alle DLTs gemeinsam haben, ist, dass die Daten auf allen Teilnehmern des Netzwerkes gespeichert werden und die Verwendung von kryptografischen Technologien, wie Hash-Funktionen und Konsensmechanismen \parencite{Ioini_DLTReview}.

%Bei der DLT handelt es sich um eine Technologie, die es ermöglicht, Daten in einem dezentralen Netzwerk zu speichern. Die Daten werden dabei auf allen Teilnehmern des Netzwerkes gespeichert. Dadurch ist es nicht möglich, die Daten zu manipulieren, da die Daten auf allen Teilnehmern des Netzwerkes gespeichert sind. Wenn ein Teilnehmer versucht, die Daten zu manipulieren, wird dies von den anderen Teilnehmern bemerkt und die manipulierten Daten werden nicht akzeptiert \parencite[S. 10]{Fill_BlockchainGrundlagen}.

Christoph Meinel und Tatiana Gayvoronskaya beschreiben die Blockchain-Technologie in ihrem Buch \textit{Blockchain - Hype oder Innovation} wie folgt:

\begin{quote}
    \textit{Die Innovation der Blockchain-Technologie ist weder ein neuer Verschlüsselungsalgorithmus
    noch eine „Alientechnologie“, sondern eine erfolgreiche Kombination bereits
    vorhandener technologischen Ansätze wie Kryptografie, dezentrale Netzwerke und Konsensfindungsmodelle.}\parencite[S. 17]{Meinel_BlockchainHypeInnovation}
\end{quote}

\noindent Daraus ergibt sich die Frage, was Blockchain eigentlich ist und wie die bereits vorhandenen technologischen Ansätze miteinander vereint werden. Diese Fragen werden in diesem Kapitel beantwortet. Zunächst wird der Begriff Blockchain definiert und anschließend die Funktionsweise erläutert. Daraufhin werden die Konsensmechanismen Proof-of-Work und Proof-of-Stake vorgestellt. Abschließend wird auf die Sicherheit von Blockchain eingegangen und Angriffe auf Blockchain werden erläutert.
% #TODO: Distributed Ledger als Vorgänger zu Blockchain erwähnen?

\subsection{Definition von Blockchain}
\label{subsec:blockchain_definition}

Der Begriff \textit{Blockchain} setzt sich aus den englischen Wörtern \textit{block} (Block) und \textit{chain} (Kette) zusammen. Eine Blockchain ist also eine Kette von Blöcken. Wie in Abbildung \ref{fig:blockchain} zu sehen ist, beinhaltet ein Block den Block-Header und mehrere Transaktionen. 

\begin{figure}[H]
    \centering
    \includegraphics[width=0.4\linewidth]{images/blockchain_block.png}
    \caption{Aufbau eines Blocks \parencite[S. 11]{Fill_BlockchainGrundlagen}}
    \label{fig:blockchain}
\end{figure}


\noindent Der Block-Header enthält Informationen über den Block, wie zum Beispiel den Hash des vorherigen Blocks, die Schwierigkeit, den Zeitstempel und die Nonce. Die Schwierigkeit wird durch das \textit{Target}-Feld angegeben und beschreibt, wie schwer es ist, das kryptografische Puzzle zu lösen und damit auch wie schwer es ist, einen neuen Block an die Blockchain anzuhängen. Die Suche nach der Lösung dieses Puzzles wird auch als \textit{Mining} bezeichnet. Der erste, der das Puzzle löst, präsentiert als Beweis dieser Lösung die sogenannte \textit{Nonce}. Nachdem diese neue Version der Blockchain (mit dem neu angehängten Block) an alle anderen Teilnehmer verteilt wurde, kann jeder Teilnehmer überprüfen, ob die Lösung korrekt ist. Die Nonce ist eine zufällige Zahl, die bei der Lösung des Puzzles verwendet wird und aufwändig zu berechnen ist. Durch den Verweis im Block-Header auf den vorherigen Block entsteht eine Kette von Blöcken (siehe Abbildung \ref{fig:chain_of_blocks}), die \textit{Blockchain} genannt wird \parencite[S. 10-12]{Fill_BlockchainGrundlagen}.

\begin{figure}[H]
    \centering
    \includegraphics[width=0.9\linewidth]{images/chain_of_blocks.png}
    \caption{Kette aus Blöcken \parencite[S. 12]{Fill_BlockchainGrundlagen}}
    \label{fig:chain_of_blocks}
\end{figure}

% Die Transaktionen enthalten jeweils Informationen zum Sender, dem Empfänger und dem Betrag.


\subsection{Kryptografische Grundlagen}
Aus der Kryptografie werden Hash-Funktionen, kryptografische Puzzles, Hash-Bäume und digitale Signaturen verwendet. Hash-Funktionen können dazu verwendet werden, um die Integrität von Daten zu gewährleisten. Eine Hash-Funktion bildet eine beliebig lange Eingabe auf eine feste Länge ab. Kryptografische Hash-Funktionen sind Hash-Funktionen, die bestimmte Eigenschaften erfüllen müssen. Die beiden wichtigsten Eigenschaften sind die \textit{Einwegfunktion} und die \textit{Kollisionsresistenz}. Eine Hash-Funktion erfüllt die Eigenschaft der Einwegfunktion, wenn es nicht möglich ist, von der Ausgabe auf die Eingabe zu schließen. Das bedeutet, dass es nicht möglich ist, aus dem Hash-Wert die ursprünglichen Daten zu rekonstruieren. Die Eigenschaft der \textit{Kollisionsresistenz} ist erfüllt, wenn es nicht möglich ist, zwei verschiedene Eingaben zu finden, die auf den gleichen Hash-Wert abgebildet werden \parencites[S. 12-13]{Brünnler_BlockchainKurzGut}[S. 6]{Fill_BlockchainGrundlagen}.

Kryptografische Puzzles werden dazu verwendet, um die Schwierigkeit beim Mining zu erhöhen. Dabei soll mittels Hash-Funktionen ein bestimmter Ausgabewert gefunden werden. Laut Definition der Einwegfunktion ist es nicht möglich, von der Ausgabe auf die Eingabe zu schließen. Daher kann die gesuchte Eingabe nur durch Ausprobieren gefunden werden. Die Schwierigkeit kann durch die Anzahl der Nullen, die am Anfang des Hash-Wertes stehen müssen, angegeben werden. Je mehr Nullen am Anfang des Hash-Wertes stehen müssen, desto schwieriger ist es, die Lösung zu finden, da dadurch der Lösungsraum verkleinert wird \parencite[S. 6-7]{Fill_BlockchainGrundlagen}.

Ein \textit{Merkle-Baum} ist ein binärer Baum, bei dem jeder Knoten den Hash-Wert seiner Kinder enthält. Der nach seinem Erfinder Ralph Merkle benannten Hash-Baum wird in der Blockchain zum Aufbau von Datenstrukturen verwendet. Der Hash-Wert der Wurzel des Baumes wird auch als \textit{Root-Hash} oder \textit{Merkle-Root} bezeichnet. Wenn sich ein Blatt des Baumes ändert, ändert sich auch der Hash-Wert der Wurzel. Dadurch kann überprüft werden, ob sich die Daten geändert haben. Wenn sich die Daten geändert haben, ändert sich auch der Root-Hash. Wenn sich die Daten nicht geändert haben, bleibt der Root-Hash gleich. Dadurch kann die Integrität der Daten überprüft werden \parencite[S. 7-8]{Fill_BlockchainGrundlagen}. In Abbildung is der Aufbau eines Merkle-Baumes zu sehen.

\begin{figure}[H]
    \centering
    \includegraphics[width=0.9\linewidth]{images/merkle_tree.png}
    \caption{Aufbau eines Merkle-Baumes \parencite[S. 8]{Fill_BlockchainGrundlagen}}
    \label{fig:merkle_tree}
\end{figure}

\noindent Von jedem Dokument $D_{1}$, $D_{2}$, $D_{3}$ und $D_{4}$ wird ein Hash-Wert berechnet und in einem Blatt des Baumes gespeichert. Die Hash-Werte der Blätter werden dann paarweise gehasht und in den Knoten darüber gespeichert. Dieser Vorgang wird solange wiederholt, bis nur noch ein Knoten übrig ist. Dieser Knoten enthält den Root-Hash. Sollte sich ein Dokument ändern, ändert sich auch der Root-Hash. Außerdem kann bewiesen werden, dass beispielsweise $D_{2}$ Teil des Baumes ist, indem mit dem Hash-Wert von $D_{2}$ und den Hash-Werten von $D_{1}$, $D_{3}$ und $D_{4}$ versucht wird, den Root-Hash zu berechnen. Wenn der berechnete Root-Hash mit dem tatsächlichen Root-Hash übereinstimmt, ist bewiesen, dass $D_{2}$ Teil des Baumes ist \parencite[S. 9]{Fill_BlockchainGrundlagen}.

Ein weitere Technologie aus der Kryptografie, die in der Blockchain verwendet wird, sind digitale Signaturen. Digitale Signaturen werden verwendet, um die Authentizität von Daten zu gewährleisten. Wenn zwei Benutzer Alice und Bob miteinander kommunizieren wollen, kann Alice eine Nachricht an Bob senden. Jeder Benutzer hat einen privaten und einen öffentlichen Schlüssel. Der öffentliche Schlüssel darf beliebig verteilt werden, während der private Schlüssel geheim gehalten werden muss. Wenn Alice eine Nachricht an Bob senden möchte, kann sie die Nachricht mit ihrem privaten Schlüssel signieren. Bob kann dann die Signatur mit dem öffentlichen Schlüssel von Alice überprüfen. Wenn die Signatur gültig ist, kann Bob sicher sein, dass die Nachricht von Alice stammt. Wenn die Signatur wiederum ungültig ist, kann Bob sicher sein, dass die Nachricht nicht von Alice stammt \parencite[S. 9-10]{Fill_BlockchainGrundlagen}.


\subsection{Dezentrale Netzwerke}

Blockchains basieren auf einem Peer-to-Peer-Netzwerk. Ein Peer-to-Peer-Netzwerk ist ein Netzwerk, bei dem alle Teilnehmer gleichberechtigt sind. Es gibt keinen zentralen Server, der die Daten verwaltet. Stattdessen werden die Daten auf allen Teilnehmern des Netzwerkes gespeichert. Wenn ein Teilnehmer Daten an das Netzwerk senden möchte, sendet er die Daten an alle anderen Teilnehmer des Netzwerkes. Wenn ein Teilnehmer Daten vom Netzwerk empfangen möchte, empfängt er die Daten von allen anderen Teilnehmern des Netzwerkes. Dadurch ist es nicht möglich, das Netzwerk zu manipulieren, da die Daten auf allen Teilnehmern des Netzwerkes gespeichert sind. Wenn ein Teilnehmer versucht, die Daten zu manipulieren, wird dies von den anderen Teilnehmern bemerkt und die manipulierten Daten werden nicht akzeptiert \parencite[S. 10]{Fill_BlockchainGrundlagen}.

\subsection{Konsensmechanismen}

In einem dezentralen Netzwerk müssen sich die Teilnehmer auf einen gemeinsamen, für alle \textit{richtigen}, Zustand einigen. Dieser Zustand kann beispielsweise die Reihenfolge der getätigten Transaktionen sein. Wenn sich die Teilnehmer nicht auf einen gemeinsamen \textit{Konsens} einigen können, kann das Netzwerk nicht funktionieren. Die Aufgabe der Konsensmechanismen ist es also, einen gemeinsamen Zustand zu finden. Die beiden bekanntesten Konsensmechanismen sind \textit{Proof-of-Work} und \textit{Proof-of-Stake}.

\subsubsection{Proof-of-Work}

Der \textit{Proof-of-Work} (PoW) ist der Konsensmechanismus, der in der Bitcoin-Blockchain verwendet wird. Der PoW besteht aus dem bereits in Abschnitt \ref{subsec:blockchain_definition} \nameref{subsec:blockchain_definition} beschriebenen kryptografischen Puzzle. Das Lösen des Puzzles durch einen beliebigen Teilnehmer dient als Nachweis für geleistete Rechenarbeit - daher die Bezeichnung \textit{Proof-of-Work} \parencite[S. 27]{Brünnler_BlockchainKurzGut}. Ein großer Nachteil bei Proof-of-Work ist der hohe Energieverbrauch, der dem hohen Rechenaufwand zum Lösen des Puzzles geschuldet ist \parencite{Zhang_EvaluationOfEnergyConsumptionInBlockChains}. 

\subsubsection{Proof-of-Stake}

Der \textit{Proof-of-Stake} (PoS) ist ein alternativer Konsensmechanismus. Im Gegensatz zum PoW wird beim PoS kein kryptografisches Puzzle gelöst. Stattdessen wird ein Teilnehmer ausgewählt, der einen neuen Block an die Blockchain anhängen darf. Die Auswahl erfolgt zufällig, wobei die Wahrscheinlichkeit, ausgewählt zu werden, von der Anzahl der Coins abhängt, die der Teilnehmer besitzt (Stake). Wenn ein Teilnehmer beispielsweise 10\% aller Coins besitzt, hat er eine 10\%ige Chance, ausgewählt zu werden. Der Teilnehmer, der ausgewählt wurde, wird auch als \textit{Validator} bezeichnet. Der Validator kann dann einen neuen Block an die Blockchain anhängen. Wenn der Validator einen Block an die Blockchain anhängt, erhält er eine Belohnung. Wenn er allerdings einen Block an die Blockchain anhängt, der ungültig ist, verliert er einen Teil seiner Coins. Dadurch wird sichergestellt, dass von den Validatoren nur gültige Blöcke an die Blockchain anhängt werden \parencites[S. 96-97]{Kapengut_EthereumTransitionToProofOfStake}[S. 34]{Meinel_BlockchainHypeInnovation}.


\subsection{Sicherheit von Blockchain}



\textbf{\textcolor{red}{Ralph Merkle!}}
\begin{itemize}
    \item Konsensmechanismen (PoW, PoS)
    \item Warum ist Blockchain sicher?
    \item Angriffe auf Blockchain (Sybil, 51\%, etc.)
\end{itemize}

\subsection{Ethereum}
Ethereum 1.0 und 2.0
\subsection{Smart Contracts}
Smart Contracts sind Programme, die auf derBlockchain ausgeführt werden. Smart Contracts 
sind in der Programmiersprache Solidity geschrieben. Smart Contracts sind in der Blockchain
gespeichert und werden von allen Teilnehmern des Netzwerks ausgeführt. Smart Contracts sind
nicht veränderbar und können nicht gelöscht werden. Smart Contracts können nur durch andere
Smart Contracts oder durch Transaktionen aufgerufen werden. Smart Contracts können Daten
in der Blockchain speichern. Smart Contracts können auch Daten aus der Blockchain lesen.
Smart Contracts können auch Transaktionen auslösen. Smart Contracts können auch andere
Smart Contracts aufrufen.

\section{Sicherheit}
\label{sec:sicherheit_basics}
% Einleitung von Overlay Netzwerk zu Underlay Netzwerk; da die Kommunikation über IP geht, ist es wichtig, dass die Kommunikation sicher ist, weil es über mehrere Knoten geht, die nicht vertrauenswürdig sind
% Was möchte ich schützen? Vertraulichkeit, Integrität, Authentizität <- welche Technologien werden verwendet, um diese Ziele zu erreichen?

Peer-to-Peer bildet das Overlay-Netzwerk, welches über das Underlay-Netzwerk, das Internet, läuft. Da die Kommunikation über das Internet läuft, ist es wichtig, dass die Kommunikation sicher ist, da sie über mehrere Knoten geht, die nicht vertrauenswürdig sind. Mittels Kryptografie kann die Vertraulichkeit, Integrität und Authentizität der Kommunikation gewährleistet werden \Parencite[S. 7]{Hellmann_IT-Sicherheit}. 

\subsection{Vertraulichkeit}

Die Vertraulichkeit der Kommunikation wird durch Verschlüsselung gewährleistet. Es gibt zwei Arten von Verschlüsselung: \textit{symmetrische} und \textit{asymmetrische} Verschlüsselung. Bei der \textit{symmetrischen} Verschlüsselung wird nur ein Schlüssel für die Verschlüsselung und Entschlüsselung verwendet. Der Sender der Nachricht verschlüsselt die Nachricht mit einem Schlüssel und sendet die nun verschlüsselte Nachricht an den Empfänger. Der Empfänger kann die Nachricht mit dem gleichen Schlüssel entschlüsseln. Dadurch kann ein Angreifer, der die Nachricht abfängt, diese nicht entschlüsseln, da er den Schlüssel nicht kennt. Das Problem bei der symmetrischen Verschlüsselung ist, dass der Schlüssel zu Beginn der Kommunikation vom Sender an den Empfänger gelangen muss. Dies stellt eine Herausforderung dar, wenn Sender und Empfänger sich noch nicht kennen und noch nie zuvor miteinander kommuniziert haben oder noch nicht über andere Wege einen Schlüssel ausgetauscht haben. Sollte der Schlüssel bei der Übertragung über einen unsicheren Kanal abgefangen werden, kann der Angreifer die Kommunikation entschlüsseln und somit mitlesen \Parencites[S. 644]{DiffieHellman_NewDirectionsInCryptography}[S. 5-8]{Wong_KryptoPraxis}. Um dieses Problem zu lösen, wurde die \textit{asymmetrische} Verschlüsselung, welche auch unter dem Begriff \textit{Public-Key Kryptografie} bekannt ist, entwickelt. Bei der asymmetrischen Verschlüsselung wird anstatt eines Schlüssels ein Schlüsselpaar generiert, das aus einem öffentlichen und einem privaten Schlüssel besteht. Der öffentliche Schlüssel wird zum Verschlüsseln der Nachrichten verwendet und der private Schlüssel wird zum Entschlüsseln der Nachrichten verwendet. 

% #TODO: Abbildung vor asymmetrische Verschlüsselung einfügen, hilft beim Verständnis

\subsection{Integrität}

Um die Integrität der Kommunikation zu gewährleisten, werden Signaturen verwendet. Für eine Signatur wird der Hashwert einer Nachricht mit dem privaten Schlüssel des Senders verschlüsselt und an die Nachricht angehängt. 

Um den Hashwert zu erhalten, wird die Nachricht mit einer Hashfunktion gehasht. Hashfunktionen sind Funktionen, die eine Eingabe beliebiger Länge in eine Ausgabe fester Länge umwandeln. Dabei ist es wichtig, dass die Hashfunktion zwei Eigenschaften erfüllt: \textit{Einwegfunktion} und \textit{Kollisionsresistenz}. Eine Einwegfunktion ist eine Funktion, die einfach zu berechnen ist, aber schwierig zu invertieren ist. Das bedeutet, dass es einfach ist, den Hashwert einer Nachricht zu berechnen, aber schwierig ist, die Nachricht aus dem Hashwert zu berechnen. Kollisionsresistenz bedeutet, dass es schwierig ist, zwei Nachrichten zu finden, die den gleichen Hashwert haben \parencite[S. 13-15]{Brünnler_BlockchainKurzGut}.

Der Empfänger kann den verschlüsselten Hashwert mit dem öffentlichen Schlüssel des Senders entschlüsseln, um dann anschließend selbst den Hashwert der Nachricht zu berechnen. Wenn der berechnete Hashwert mit dem entschlüsselten Hashwert übereinstimmt, kann der Empfänger sicher sein, dass die Nachricht nicht verändert wurde und somit die Integrität der Nachricht gewährleistet ist. Falls der berechnete Hashwert nicht mit dem entschlüsselten Hashwert übereinstimmt, wurde die Nachricht verändert und die Integrität der Nachricht ist nicht mehr gewährleistet \Parencite[S. 73-78]{Hellmann_IT-Sicherheit}.


\subsection{Authentizität}

Für die Authentizität der Kommunikation werden ebenfalls Signaturen verwendet. Der Sender signiert die Nachricht mit seinem privaten Schlüssel und sendet die signierte Nachricht an den Empfänger. Durch die Integration der Blockchain in dieser Arbeit fungiert diese als öffentliches Verzeichnis oder auch Key-Server, in dem die öffentlichen Schlüssel der Teilnehmer gespeichert sind. Der Empfänger kann den öffentlichen Schlüssel des Senders aus der Blockchain auslesen, wodurch nachvollzogen werden kann, ob die Nachricht vom Sender stammt oder nicht.

\section{Existierende IM-Protokolle}
Es gibt bereits diverse IM-Protokolle, die P2P verwenden. Diese Protokolle werden in diesem Kapitel
vorgestellt und verglichen. Die Protokolle werden in zwei Kategorien unterteilt: Protokolle, die
einen zentralen Server verwenden und Protokolle, die P2P verwenden. Die Protokolle, die einen
zentralen Server verwenden, werden in zwei weitere Kategorien unterteilt: Protokolle, die
einen zentralen Server für die Kommunikation verwenden und Protokolle, die einen zentralen
Server für die Vermittlung verwenden. Die Protokolle, die P2P verwenden, werden in zwei weitere
Kategorien unterteilt: Protokolle, die ein hybrides P2P-Netzwerk verwenden und Protokolle, die
ein reines P2P-Netzwerk verwenden. Die Protokolle werden in den folgenden Abschnitten vorgestellt
und verglichen.

Protokolle mit zentralem Server:

\begin{itemize}
    \item XMPP
    \item Matrix
    \item Telegram
    \item Signal
    \item Threema
    \item WhatsApp
    \item IRC
    \item Skype
\end{itemize}

Protokolle mit P2P:

\begin{itemize}
    \item Tox
    \item Briar
    \item Jami
    \item Ring
    \item Ricochet
    \item Bitmessage
    \item RetroShare
\end{itemize}

XMPP:
Dieses Protokoll ist ein offener Standard für Instant Messaging und basiert auf XML. Es ist ein
Protokoll mit zentralem Server. Es gibt verschiedene XMPP-Server, die miteinander kommunizieren
können. Die Kommunikation zwischen den Servern erfolgt über das Server-zu-Server-Protokoll (S2S).
Die Kommunikation zwischen den Teilnehmern erfolgt über das Client-zu-Server-Protokoll (C2S).

Matrix:
Dieses Protokoll ist ein offener Standard für Instant Messaging und basiert auf JSON. Es ist ein
Protokoll mit zentralem Server. Es gibt verschiedene Matrix-Server, die miteinander kommunizieren
können. Die Kommunikation zwischen den Servern erfolgt über das Server-zu-Server-Protokoll (S2S).
Die Kommunikation zwischen den Teilnehmern erfolgt über das Client-zu-Server-Protokoll (C2S).

Telegram:
Dieses Protokoll ist ein proprietäres Protokoll für Instant Messaging und basiert auf JSON. Es ist
ein Protokoll mit zentralem Server. Es gibt verschiedene Telegram-Server, die miteinander
kommunizieren können. Die Kommunikation zwischen den Servern erfolgt über das Server-zu-Server-
Protokoll (S2S). Die Kommunikation zwischen den Teilnehmern erfolgt über das Client-zu-Server-
Protokoll (C2S).

    \chapter{Anforderungsanalyse}
\label{chap:anforderungsanalyse}

% Lastenheft (auch Anforderungs- oder Kundespezifikation):
% - enthält eher abstrakte, eher allgemeine Festlegungen der gewünschten Dienste
% - fundamentale Eigenschaften des Produktes
% - Beschreibung des "was" und nicht des "wie"
% - hat den Sinn, der Lösung nicht vorzugreifen (Art der Umsetzung ist hier nicht relevant)
% - sobald das Pflichtenheft existiert, ist das Lastenheft nicht mehr relevant und muss nicht weiter
%   gepflegt werden

% Pflichtenheft:
% - Entwickler beschreibt wie er die im Lastenheft dargelegten Anforderungen zu erfüllen gedenkt
% - Beschreibung bis in das kleinste Detail, wie sich die Software unter bestimmten Bedingungen verhalten soll
% - muss stets aktuell gehalten werden

Die Anforderungsanalyse dient dazu, die Grundlage für den erfolgreichen Verlauf eines Softwareprojekts zu schaffen, indem sie sicherstellt, dass die Ziele und Anforderungen des Projekts klar definiert und verstanden werden \parencite{Zakharyan_SoftwareRequirementsForMessagingService}. In dieser Arbeit wird ein Prototyp für ein Peer-to-Peer-Instant-Messaging-Protokoll entwickelt. Das Ziel dieser Arbeit ist es, die Machbarkeit eines solchen Protokolls aufzuzeigen. Um dieses Ziel zu erreichen, müssen die Anforderungen an das Protokoll klar definiert werden. Dazu werden in diesem Kapitel die funktionalen und nicht-funktionalen Anforderungen an das Protokoll beschrieben. 



\section{Funktionale Anforderungen}


%Dies sind Aussagen zu den Diensten, die das System leisten sollte, zur Reaktion des Systems auf bestimmte 
%Eingaben und zum Verhalten des Systems in bestimmten Situationen. 

Funktionale Anforderungen beziehen sich auf die spezifischen Funktionen und Aufgaben, 
die eine Software oder ein System erfüllen muss, um die Bedürfnisse und Erwartungen der Benutzer zu erfüllen.
Sie beschreiben, was das System tun soll, welche Aktionen es ausführen muss und welche 
Ergebnisse es liefern sollte. Diese Anforderungen sind entscheidend, um sicherzustellen, dass die entwickelte Software 
oder wie in diesem Fall, das entwickelte Protokoll die erwarteten Funktionen erbringt. Sie dienen 
als Grundlage für das Design, die Entwicklung, die Validierung und die Verifizierung von Software-Systemen und 
sind ein wichtiger Bestandteil des Anforderungsmanagements im Software-Engineering-Prozess.
% #TODO: find and add source 
\\

\noindent Folgende Funktionen soll das Protokoll bieten:

\begin{itemize}
    \item Registrierung des Nutzers
    \item Authentifizierung des Nutzers
    \item Verschlüsselung der Kommunikation
    \item Versenden und Empfangen von Textnachrichten
    \item Kontaktmanagement (Kontakt hinzufügen, löschen, blockieren)
    \item Handhabung von Nachrichten, wenn der Empfänger offline ist
\end{itemize}


%\subsection{Benutzeranforderungen und -erwartungen}

Benutzer erwarten die folgenden Funktionen...
%\subsection{Sicherheitsanforderungen an das Protokoll}

Sicherheitsanforderungen sind ...
\section{Nicht-funktionale Anforderungen}

%Dies sind Beschränkungen der durch das System angebotenen Dienste oder Funktionen. Das schließt 
%Zeitbeschränkungen, Beschränkungen des Entwicklungsprozesses und einzuhaltende Standards ein.

Nicht-funktionale Anforderungen sind Anforderungen, die sich nicht auf die 
spezifische Funktionalität einer Software-Anwendung beziehen, sondern auf andere Qualitätsmerkmale 
und Aspekte, die die Leistung, Zuverlässigkeit und Benutzerfreundlichkeit der Software betreffen. 
Diese Anforderungen beschreiben, wie die Software funktionieren sollte, anstatt was sie tun sollte. 
Nicht-funktionale Anforderungen sind genauso wichtig wie funktionale Anforderungen, da sie einen 
erheblichen Einfluss auf die Gesamtleistung und die Benutzerzufriedenheit haben können.
% #TODO: find and add source 
\\
Die folgenden nicht-funktionale Anforderungen soll das Protokoll bieten:

\begin{itemize}
    \item Performanz
    \item Sicherheit
    \item Zuverlässigkeit
    \item Kompatibilität
    %\item Interoperabilität
    \item Privatsphäre
    \item Skalierbarkeit
\end{itemize}
    \chapter{Entwurf und Architektur}

\section{Protokolldesign und -struktur}
Das Protokoll ist wie folgt strukturiert...

\subsection{Grundlagen des Protokolls}

Für ein Peer-to-Peer Netzwerk gibt es verschiedene Typen. Abbildung \ref{p2p_typen} zeigt 
vier Typen und ihre Unterteilung in unstrukturierte und strukturierte Netzwerke.

\begin{center}
    \captionsetup{type=figure}
    \includegraphics[width=1\linewidth]{images/peer_to_peer_typen.png}
    \captionof{figure}{Typen von Peer-to-Peer Netzwerken \parencite{Luntovskyy_ModRechnernetze}}
    \label{p2p_typen}
\end{center}

\noindent Das Protokoll dieser Arbeit fällt in die Kategorie der hybriden Peer-to-Peer Netzwerke.
Es ist sowohl strukturiert als auch unstrukturiert. Alle Teilnehmer sind in einem Netzwerk organisiert,
das aus verschiedenen Knoten besteht, wobei jeder Knoten einen Teilnehmer des Netzwerks repräsentiert.
Alle Knoten im Netzwerk sind untereinander verbunden und können Textnachrichten direkt, ohne die Verwendung 
eines Servers austauschen. Sollte sich jedoch der gewünschte Empfänger nicht im Netzwerk befinden, besteht
die Möglichkeit, dass sich der Empfänger hinter einem NAT-Gateway oder einer Firewall befindet. In diesem
Fall wird die Nachricht über einen Server, genauer gesagt über ein TCP-Relay, an den Empfänger weitergeleitet.
Der Server ist nur für die Weiterleitung der Nachricht zuständig und speichert diese nicht. Dies ist
ein Unterschied zu anderen Instant-Messaging-Protokollen, wie zum Beispiel \textcolor{red}{[Beispiel 
finden oder Satz weglassen]}, bei dem der Server die Nachrichten speichert, wenn der Empfänger offline ist. 
Das Protokoll dieser Arbeit ist also ein hybrides Peer-to-Peer-Protokoll, da es sowohl strukturierte 
als auch unstrukturierte Eigenschaften aufweist.



\subsection{Verbindung und Identität}

Um eine Verbindung zwischen zwei Teilnehmern herzustellen, muss ein Teilnehmer den öffentlichen
Schlüssel des anderen Teilnehmers kennen. Dieser wurde bei der Registrierung in die Blockchain
geschrieben. Da in der Hashtabelle Schlüssel-Wert-Paare gespeichert werden, wird
der öffentliche Schlüssel als Schlüssel und die IP-Adresse und der Port des Teilnehmers als
Wert gespeichert. Der Teilnehmer, der eine Verbindung herstellen möchte, muss den öffentlichen
Schlüssel des anderen Teilnehmers kennen. Dazu muss er den öffentlichen Schlüssel in der
verteilten Hashtabelle suchen. Dazu berechnet er aus dem öffentlichen Schlüssel einen Hashwert.
Der Hashwert wird in einem Zahlenbereich abgebildet. Der Teilnehmer sucht in der verteilten
Hashtabelle nach dem nächsten Hashwert. Der Teilnehmer, der den Hashwert in der verteilten
Hashtabelle gespeichert hat, ist der Nachbar des Teilnehmers, der die Verbindung herstellen
möchte. Der Teilnehmer, der die Verbindung herstellen möchte, sendet eine Anfrage an den
Nachbarn. Die Anfrage enthält den öffentlichen Schlüssel des Teilnehmers, der die Verbindung
herstellen möchte. Der Nachbar antwortet mit seiner IP-Adresse und seinem Port. Der Teilnehmer,
der die Verbindung herstellen möchte, kann nun eine Verbindung zum Nachbarn aufbauen. Der
Nachbar leitet die Nachrichten an den Teilnehmer weiter, der die Verbindung herstellen möchte.

\subsection{Nachrichten und Daten}

Benutzer können ausschließlich Textnachrichten miteinander austauschen. 
Diese Nachrichten können an einzelne Peers gesendet werden.
Die gesamte Kommunikation ist mit starken Verschlüsselungsalgorithmen 
verschlüsselt, um Privatsphäre und Sicherheit zu gewährleisten.

\subsection{Nachrichtenverlauf}

Benutzer haben die Möglichkeit, ihren Nachrichtenverlauf lokal zu speichern.
Der Nachrichtenverlauf ist ebenfalls verschlüsselt, um die Sicherheit zu gewährleisten.

\subsection{Sicherheit}

\begin{itemize}
    \item Integration von Verschlüsselung für Datenschutz und Sicherheit 
    \item Schutz vor Angriffen wie Man-in-the-Middle
\end{itemize}

Alle Nachrichten sind mit öffentlichen Schlüsseln verschlüsselt, um sicherzustellen, 
dass nur der beabsichtigte Empfänger die Nachrichten entschlüsseln und lesen kann.
Ein sicheres Verfahren für den Schlüsselaustausch und die Schlüsselverwaltung 
wird implementiert, um gegen Abhören und Man-in-the-Middle-Angriffe zu schützen.


\section{Technologien und Tools}
Für die Umsetzung dieses Designs wurden diese Technologien verwendet.

\section{Integration von Blockchain}
\label{sec:blockchainintegration}


\subsection{Auswahl der Blockchain}
Für die Integration der Blockchain in das Protokoll wurde zunächst eine geeignete Blockchain gesucht. Dabei wurde sich auf die beiden bekanntesten Blockchains, Bitcoin und Ethereum, beschränkt. Da Bitcoin eine reine Kryptowährung ist und keine Smart Contracts unterstützt, wurde sich für Ethereum entschieden. Ethereum ist eine Blockchain, die zwar auch eine Kryptowährung, Ether, besitzt, aber zusätzlich auch Smart Contracts unterstützt. Der Grund für die Existenz einer Währung auf der Blockchain ist, dass die Smart Contracts, die auf der Blockchain ausgeführt werden, mit Ether bezahlt werden müssen \parencite[S. 2]{Antonopoulos_MasteringEthereum}. Somit ist es nicht möglich, Smart Contracts auf der Blockchain auszuführen, ohne Ether zu besitzen. Da die Smart Contracts auf der Blockchain aber mit Ether bezahlt werden müssen, ist es notwendig, dass jeder Nutzer, der einen Smart Contract ausführen möchte, Ether besitzt.


\subsection{Registrierung}
Um das Protokoll zu nutzen, muss sich jeder Nutzer zunächst auf der Blockchain registrieren. Dazu muss ein Smart Contract auf der Blockchain ausgeführt werden, der die Registrierung des Nutzers durchführt. Dieser Smart Contract wird mit dem Benutzernamen und dem statischen öffentlichen Schlüssel aufgerufen. Der statische öffentliche Schlüssel wird bei der Registrierung festgelegt und kann nicht mehr geändert werden. Der Smart Contract erstellt einen neuen Eintrag in der Blockchain, der den Benutzernamen und den öffentlichen Schlüssel des Nutzers enthält. Der Smart Contract wird nur einmalig bei der Registrierung aufgerufen. Sollte ein Nutzer seinen Benutzernamen ändern wollen, muss er sich mit dem neuen Benutzernamen und dem öffentlichen Schlüssel eines neu erzeugten statischen Schlüsselpaars erneut registrieren.
% Der alte Benutzername wird dann aus der Blockchain gelöscht. -> wie könnte man das absichern?

\subsection{Kommunikation}
Für die Kommunikation mit anderen Teilnehmern, muss zunächst die ID des anderen Teilnehmers im Kademlia-Netzwerk bekannt sein. Dazu wird der Benutzername des anderen Teilnehmers auf der Blockchain gesucht um zu kontrollieren, ob dieser bereits registriert ist. Ist der Benutzername nicht auf der Blockchain vorhanden, ist der andere Teilnehmer nicht registriert und es kann keine Verbindung aufgebaut werden. Ist der Benutzername auf der Blockchain vorhanden, wird der dazugehörige öffentliche Schlüssel ausgelesen. Der Benutzername entspricht gleichzeitig der ID des Teilnehmers im Kademlia-Netzwerk. Somit kann der Verbindungsaufbau, der in Abschnitt \ref{subsec:verbindungsmanagement} \nameref{subsec:verbindungsmanagement} beschrieben wird, durchgeführt werden.
Wenn der Verbindungsaufbau erfolgreich war, kann die Kommunikation beginnen. Dazu wird der öffentliche Schlüssel des anderen Teilnehmers benötigt. Dieser wird ebenfalls auf der Blockchain gespeichert. Somit kann jeder Teilnehmer die öffentlichen Schlüssel der anderen Teilnehmer auf der Blockchain finden und die Nachrichten, die er erhält, mit dem öffentlichen Schlüssel des Absenders verifizieren. Somit kann sichergestellt werden, dass die Nachrichten tatsächlich vom angegebenen Absender stammen und nicht von einem anderen Teilnehmer gesendet wurden, der sich als jemand anderes ausgibt.

\subsection{Entwurf der Smart Contracts}
\label{subsec:smartcontracts}

Auf der Ethereum-Blockchain wird die hauptsächlich Programmiersprache Solidity verwendet, um Smart Contracts zu erstellen. Solidity ist eine objektorientierte Programmiersprache, die stark an JavaScript angelehnt ist \parencite[S. 131]{Antonopoulos_MasteringEthereum}. Die Smart Contracts, die für das Protokoll benötigt werden, sind in Solidity implementiert.




    \chapter{Evaluation}
\label{chap:evaluation}

Siehe \nameref{chap:anforderungsanalyse}.

Funktionale Anforderungen:

\begin{itemize}
    \item \textbf{F1}: Teilnehmer können sich anhand des Namens finden (Peer-Discovery and Routing)
    \item \textbf{F2}: Teilnehmer können sich miteinander verbinden (Aufbau, Abbau und Verbindung wiederherstellen, wenn Abbruch wegen Netzwerkfehler)
    \item \textbf{F3}: Nachrichtenformat muss definiert sein, sodass alle Teilnehmer es verstehen
    \item \textbf{F4}: Sicherheit und Verschlüsselung, um Vertraulichkeit zu gewährleisten
    \item \textbf{F5}: Protokoll muss plattformunabhängig sein, sollte standardisierte Technologien wie UDP, TCP, HTTP, JSON, etc. verwenden
    \item \textbf{F6}: Fehlerbehandlung und Wiederholungsmechanismen, Erkennung von Übertragungsfehlern
\end{itemize}

Nicht-funktionale Anforderungen:

\begin{itemize}
    \item \textbf{NF1}: Performanz: schnelle Nachrichtenübertragung, geringe Latenz
    \item \textbf{NF2}: Vertraulichkeit, Integrität und Authentizität der Daten muss gewährleistet sein
    \item \textbf{NF3}: Zuverlässigkeit: Das System muss kontinuierlich verfügbar sein, um die Kommunikation zu ermöglichen
    \item \textbf{NF4}: Kompatibilität: Ermöglichung einer nahtlosen Kommunikation zwischen Usern mit unterschiedlicher Plattform/Gerät
    \item \textbf{NF5}: Skalierbarkeit: Kommunikationseffizienz und -qualität muss auch bei steigender Anzahl an Teilnehmern ohne signifikante Einbußen gewährleistet sein
\end{itemize}



% Signaling Server könnte auch mit Anfragen geflutet werden -> der Sever könnte bei jedem checken, ob es sich um einen validen Teilnehmer handelt, was aber sehr aufwändig wäre

% wenn Kademlia mit Anfragen geflutet wird, könnte es zu einem Denial of Service kommen, da die Knoten nichtmehr erreichbar sind -> habe aber ja mit dem Signaling Server noch eine weitere Instanz, die die Anfragen entgegen nimmt und weiterleitet

Abgleich des entwickelten Protokolls mit den in der Anforderungsanalyse identifizierten funktionalen und
nicht-funktionalen Anforderungen. Welche konnte ich erfüllen und welche nicht? Falls nicht, warum
konnte ich diese Anforderung gar nicht oder nur teilweise erfüllen?

\begin{itemize}
    \item Implementierung von ICE, weil im TURN Protokoll nicht steht, wie denn die \textit{relayed addresses} zwischen den Teilnehmern ausgetauscht werden sollen.
    \item ICE hat das Manko, dass nicht definiert ist, wie das Signaling abläuft; es wird einfach vorausgesetzt, dass das funktioniert\dots
    \item Angriffe auf Kademlia und Blockchain nochmal überprüfen und Risiken aufzeigen, die ich nicht verhindert habe/konnte
    \item Kosten für Registrierung von Benutzern könnte abschreckend sein, da es viele Alternativen gibt, die kostenlos sind (und mehr Features bieten); dies bietet allerdings Schutz vor Sybil-Angriffen (habe ich auch so in der Arbeit geschrieben)
    \item Neuaushandlung des Schlüssels für jede Nachricht könnte zu viel sein
    \item warum habe ich nicht die Kademlia ID in die Blockchain geschrieben? dadurch wäre der Benutzername frei wählbar und die ID wäre trotzdem eindeutig -> zusätzliche Transaktionen auf der Blockchain, die Geld kosten
\end{itemize}


Um diesen Angriff zu erschweren, wird in dieser Arbeit die Teilnahme am Netzwerk erst durch eine vorherige Registrierung ermöglicht. Die Registrierung wird durch die Blockchain-Technologie realisiert und ist mit Kosten verbunden, die ein Angreifer aufbringen muss, um am Netzwerk teilzunehmen. Da bei einem Sybil-Angriff der Angreifer mehrere Identitäten verwendet, um das Netzwerk zu übernehmen, muss er für jede Identität die Kosten aufbringen. Dadurch wird der Angriff erschwert, da der Angreifer mehr Ressourcen benötigt, um das Netzwerk zu übernehmen.

 % Diskussion?
    \chapter{Schlussfolgerung und Ausblick}

In dieser Arbeit wurde ein Peer-to-Peer-Instant-Messaging-Protokoll entwickelt, das einen Schwerpunkt auf Sicherheit legt. Die Sicherheit wird durch die Verwendung von Ende-zu-Ende-Verschlüsselung und der Integration der Ethereum-Blockchain gewährleistet. Die Verwendung von Kademlia als Routing-Algorithmus ermöglicht eine schnelle Suche nach Teilnehmern und die Verwendung von ICE ermöglicht die Kommunikation zwischen Teilnehmern, die sich hinter NAT befinden.

Diese Arbeit hatte nicht den Anspruch eine vollständige Definition eines Protokolls zu liefern, sondern sollte einen Prototypen entwickeln, der die Machbarkeit eines solchen Protokolls aufzeigt. Deshalb wurden nicht alle Aspekte mit der eigentlich nötigen Tiefe beleuchtet. Es hat sich herausgestellt, dass die Verwendung von Peer-to-Peer-Mechanismen in Instant-Messaging-Protokollen durchaus sinnvoll, aber sehr komplex ist. Es müssen viele Aspekte beachtet werden, die in zentralisierten Protokollen nicht vorhanden sind, wie beispielsweise der Umgang mit NAT. Die Entwicklung einer Ende-zu-Ende-Verschlüsselung wurde durch die teilweise Verwendung des Signal-Protokolls vereinfacht. Die Auswahl der Algorithmen für die Schlüsselvereinbarung hat Potenzial zur Verbesserung, da kein genauer Vergleich mit anderen Algorithmen durchgeführt wurde. Die Integration der Blockchain ist jedoch eine sinnvolle Maßnahme, um die Sicherheit des Protokolls zu erhöhen, da keiner zentralen Instanz vertraut werden muss. Dies könnte die Akzeptanz des Protokolls erhöhen.

Eine prototypische Implementierung wäre ein sinnvoller nächster Schritt, um die Machbarkeit des Protokolls zu demonstrieren und eventuelle Definitionslücken zu schließen. Darüber hinaus könnten Metriken definiert und gemessen werden, um die Leistung des Protokolls zu bewerten.
    %%%%%%%%%%%%%%%%%%%%%%%%%%%%%%%%%%%%% END: Numbered Chapters %%%%%%%%%%%%%%%%%%%%%%%%%%%%%%%%%%%%


    %%%%%%%%%%%%%%%%%%%%%%%%%%%%%%%%%%%%% BEGIN: Literature %%%%%%%%%%%%%%%%%%%%%%%%%%%%%%%%%%%%%%%%%
    \printbibliography[nottype=online]
    %\printbibliography[nottype=online, title={Standards}]
    %\printbibliography[nottype=online, title={Papers}]
    \printbibliography[type=online, title={Webseiten}]
    %%%%%%%%%%%%%%%%%%%%%%%%%%%%%%%%%%%%% END: Literature %%%%%%%%%%%%%%%%%%%%%%%%%%%%%%%%%%%%%%%%%%%

\end{document}