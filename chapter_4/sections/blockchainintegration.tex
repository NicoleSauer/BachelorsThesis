\section{Sicherheit durch Blockchain}
\label{sec:blockchainintegration}

In diesem Abschnitt wird die Integration der Blockchain in das Protokoll beschrieben. Dazu werden zunächst die Möglichkeiten der Integration erläutert und ausgewählt. Darauf folgt die Beschreibung der ausgewählten Integrationsmöglichkeit und die dafür benötigten Funktionen, die die Blockchain bieten muss. Basierend darauf wird eine geeignete Blockchain ausgewählt und die Integration in das Protokoll beschrieben.

\subsection{Möglichkeiten der Integration}

Eine Blockchainintegration in das Protokoll kann auf unterschiedliche Weise erfolgen und verschiedene Vorteile bieten.

\begin{enumerate}
    \item \label{first} Eine Möglichkeit wäre die Verwendung der Blockchain zur Verwaltung von Benutzeridentitäten. Dazu wird jeder Teilnehmer auf der Blockchain registriert und erhält eine eindeutige ID. Diese ID kann dann als ID im Kademlia-Netzwerk verwendet werden. Dadurch würde die Blockchain als zentrale Instanz für die Verwaltung der Benutzeridentitäten dienen.
    \item \label{second} Eine weitere Anwendungsmöglichkeit könnte den Fokus auf Datenschutz und Anonymität legen. Hierfür könnten Smart Contracts auf der Blockchain verwendet werden, die sicherstellen, dass nur autorisierte Nutzer auf bestimmte Informationen zugreifen können. So könnte beispielsweise ein Smart Contract auf der Blockchain erstellt werden, der personenbezogene Daten wie den Namen, die Adresse oder die Telefonnummer eines Nutzers enthält. Dieser Smart Contract könnte dann so konfiguriert werden, dass nur der Nutzer selbst und bestimmte andere Nutzer, die der Nutzer autorisiert hat, auf diese Daten zugreifen können. Dadurch würde die Blockchain als zentrale Instanz für die Verwaltung der Zugriffsrechte auf personenbezogene Daten dienen.
    \item \label{three} Die Blockchain könnte auch zur Verifizierung von Nachrichtenquellen verwendet werden. Dazu könnte jeder Teilnehmer des Protokolls auf der Blockchain registriert werden und einen öffentlichen Schlüssel hinterlegen. Dieser öffentliche Schlüssel könnte dann von anderen Teilnehmern verwendet werden, um die Identität des Absenders einer Nachricht zu verifizieren. Dadurch würde die Blockchain als zentrale Instanz für die Verwaltung der öffentlichen Schlüssel dienen.
    \item \label{four} Eine Protokollierung von Nachrichten wäre ebenfalls möglich. Dazu könnte jede Nachricht auf der Blockchain protokolliert werden. Dadurch hätte man eine öffentliche und unveränderliche Historie aller Nachrichten, die über das Protokoll gesendet wurden.
\end{enumerate}

\subsection{Integration in das Protokoll}

Für die Auswahl der Integrationsmöglichkeit wurden die in Abschnitt \ref{chap:anforderungsanalyse} (\textit{\nameref{chap:anforderungsanalyse}}) definierten funktionalen und nicht-funktionalen Anforderungen herangezogen. Die meisten funktionalen und nicht-funktionalen Anforderungen werden bereits durch die Verwendung von Kademlia und ICE erfüllt. Die Blockchain eignet sich wegen ihrer Manipulationssicherheit besonders für die Sicherstellung der Authentizität der Nachrichten (siehe Punkt \ref{three}). Damit einhergehend ist auch die Verwaltung der Benutzeridentitäten (siehe Punkt \ref{first}), da der gespeicherte öffentliche Schlüssel eindeutig einem Benutzer zugeordnet werden kann und somit die Identität des Benutzers eindeutig festgelegt wird. Die übrigen Integrationsmöglichkeiten (siehe Punkt \ref{second} und \ref{four}) sind für die Anforderungen an das Protokoll nicht relevant und wurden daher nicht weiter betrachtet.


\subsection{Auswahl der Blockchain}
% # TODO: Kosten für die Ausführung meines Smart Contracts berechnen lassen und in die Arbeit schreiben
Für die Integration einer Blockchain in das Protokoll wurde zunächst eine geeignete Blockchain gesucht. Dabei wurde sich auf die beiden bekanntesten Blockchains, Bitcoin und Ethereum, beschränkt. Um eine Möglichkeit zu haben, eine Blockchain einzubinden, muss diese eine Funktion bieten, um beliebige Daten zu speichern und zu lesen. Dies kann durch Smart Contracts ermöglicht werden (siehe \ref{subsection:smart_contracts} \textit{\nameref{subsection:smart_contracts}}). Da Bitcoin eine reine Kryptowährung ist und keine Smart Contracts unterstützt, wurde sich für Ethereum entschieden. 

Ethereum ist eine Blockchain, die zwar auch eine Kryptowährung names \textit{Ether} besitzt, aber zusätzlich auch Smart Contracts unterstützt (siehe \ref{sec:ethereum_basics} \textit{\nameref{sec:ethereum_basics}}). Wenn ein Smart Contract ausgeführt wird, werden Ressourcen, wie Speicher und Rechenleistung, im Netzwerk verwendet. Die Knoten, die den Smart Contract ausführen, werden dafür mit Ether belohnt. Die Kosten für die Ausführung eines Smart Contracts werden in \textit{Gas} gemessen. Gas ist eine separate virtuelle Währung, die die Kosten für die Ausführung eines Smart Contracts angibt. Es wird Gas statt Ether verwendet, da der Ether-Preis stark schwanken kann. Der Gas-Preis hingegen ist konstant und wird in \textit{Gwei} gemessen. 1 Gwei entspricht dabei $10^{-9}$ Ether. Theoretisch wäre es möglich, den Preis für die Ausführung einer Methode innerhalb eines Smart Contracts auf $0$ zu setzen. Allerdings würde dies zum einen dazu führen, dass ein Angreifer den Smart Contract unendlich oft ausführen könnte, was das Netzwerk überlasten könnte. Zum anderen würde es dazu führen, dass der Smart Contract nicht mehr ausgeführt wird, da die Knoten im Netzwerk keine Belohnung für die Ausführung erhalten und somit keinen Anreiz haben, den Smart Contract auszuführen \parencite[S. 105-107]{Antonopoulos_MasteringEthereum}. 

Das bedeutet für die Teilnehmer des Protokolls, dass sie Ether besitzen müssen, um sich registrieren zu können. Für die zwei weiteren Funktionen, die die Blockchain im Protokoll übernimmt, ist keine Währung notwendig, da es sich um reine Lesezugriffe handelt, welche keine Kosten verursachen. Die beiden Funktionen sind die Suche nach einem Benutzernamen und die Suche nach einem öffentlichen Schlüssel anhand eines Benutzernamens. Die Blockchain wird also für die Registrierung der Teilnehmer und für die Suche nach Benutzernamen und öffentlichen Schlüsseln verwendet.


\subsection{Registrierung}
\label{subsec:contract_registrierung}
Um das Protokoll zu nutzen, muss sich jeder Nutzer zunächst auf der Blockchain registrieren. Dazu muss ein Smart Contract auf der Blockchain ausgeführt werden, der die Registrierung des Nutzers durchführt. Dieser Smart Contract wird mit dem Benutzernamen und dem statischen öffentlichen Schlüssel aufgerufen. Der statische öffentliche Schlüssel wird bei der Registrierung, bevor der Smart Contract angestoßen wird, festgelegt und kann nicht mehr geändert werden. Der Smart Contract erstellt einen neuen Eintrag in der Blockchain, der den Benutzernamen und den öffentlichen Schlüssel des Nutzers enthält. Um Dopplungen in der Blockchain zu vermeiden, wird der Benutzername nicht als Key in der Blockchain gespeichert, sondern als Value. Als Key wird die Adresse des Benutzers verwendet. Dadurch wird sichergestellt, dass jeder Benutzer nur einmal auf der Blockchain registriert werden kann. Diese Funktion wird nur einmalig bei der Registrierung aufgerufen. Sollte ein Nutzer seinen Benutzernamen ändern wollen, muss er sich mit dem neuen Benutzernamen und dem öffentlichen Schlüssel eines neu erzeugten statischen Schlüsselpaars erneut registrieren. Die Funktion im Smart Contract könnte wie folgt aussehen:

\begin{lstlisting}[language=Solidity, caption={Registrierung eines Nutzers auf der Blockchain},captionpos=b]
// Function to register a user
function registerUser(string memory _username, 
    bytes memory _publicKey) external {
        // Check if the username and public key are not empty
        require(bytes(_username).length > 0, 
            "Username cannot be empty");
        require(_publicKey.length > 0, 
            "Public key cannot be empty");

        // Check if the username is not already taken
        require(usernames[_username] == address(0), 
            "Username is already taken");

        // Check if the user is not already registered
        require(!users[msg.sender].isRegistered, "User is already registered");

        // Add the user to the users mapping with the user's 
        // username as the key
        users[msg.sender] = User(_username, _publicKey, true);

        // Store the username to address mapping
        usernames[_username] = msg.sender;

        // Emit the UserRegistered event with the user's address, 
        // username and public key
        emit UserRegistered(msg.sender, _username, _publicKey);
}
\end{lstlisting}

\noindent Für die Kommunikation mit anderen Teilnehmern, benötigt man die ID des anderen Teilnehmers im Kademlia-Netzwerk. Da die ID im Kademlia-Netzwerk dem Benutzernamen entspricht, muss der Benutzername des anderen Teilnehmers bekannt sein. Dazu wird dieser Benutzername auf der Blockchain gesucht, um zu kontrollieren, ob dieser bereits registriert ist. Hierfür wird eine Funktion im Smart Contract auf der Blockchain aufgerufen, die den Benutzernamen als Parameter erhält. Der Smart Contract sucht dann in der Blockchain nach dem Benutzernamen und gibt ihn, falls vorhanden, zurück. Sollte kein Benutzer mit diesem Benutzernamen auf der Blockchain vorhanden sein, wird ein leerer String zurückgegeben, der darauf hinweist, dass der gesuchte Benutzer nicht registriert ist und somit keine Verbindung aufgebaut werden kann. Der Rückgabewert muss von der Applikation, die dieses Protokoll nutzt, abgefangen werden. 
Die Funktion könnte wie folgt aussehen:

\begin{lstlisting}[language=Solidity, caption={Suche nach einem Benutzernamen auf der Blockchain},captionpos=b]
// Function to check if a user is registered
function isUserRegistered(string memory _username) external view returns (bool) {
    // Try to get the address associated with the username
    address userAddress = usernames[_username];
    // Check if the address is empty
    if (userAddress == address(0)) {
        // If the address is empty, return false
        return false;
    }
    // If the address is not empty, get the user details using the address
    User storage user = users[userAddress];
    // Return the user's registration status
    return (user.isRegistered);
}
\end{lstlisting}



\subsection{Öffentlicher Schlüssel}
\label{subsec:contract_oeffentlicher_schluessel}

Ist der Benutzername auf der Blockchain vorhanden und es kommt zum Verbindungsaufbau wie in Abschnitt \ref{subsec:verbindungsmanagement} (\nameref{subsec:verbindungsmanagement}) beschrieben, wird der öffentliche Schlüssel des Benutzers ausgelesen. Dafür wird die nachstehende Funktion genutzt:

\begin{lstlisting}[language=Solidity, caption={Suche nach einem öffentlichen Schlüssel auf der Blockchain},captionpos=b]
// Function to get a user's public key
function getUserPublicKey(string memory _username) external view 
    returns (bytes memory) {
        // Try to get the address associated with the username
        address userAddress = usernames[_username];

        // Check if the address is empty
        require(userAddress != address(0), 
            "User does not exist.");

        // If the address is not empty, get the user 
        // details using the address
        User storage user = users[userAddress];

        // Return the user's public key
        return (user.publicKey);
}
\end{lstlisting}

\noindent Um Redundanz im Code zu vermeiden, wäre es sinnvoll das Suchen nach der Adresse des übergebenen Benutzernamens, die Prüfung ob die Adresse leer ist und das Auslesen der Benutzerdaten aus dem Mapping in eine eigene Funktion auszulagern, welche nur von innerhalb des Smart Contracts aufgerufen werden kann. Aus Gründen des besseren Lese- und Verständnisflusses wurde dies hier nicht umgesetzt.

Durch das Speichern des öffentlichen Schlüssels auf der Blockchain kann jeder Teilnehmer die öffentlichen Schlüssel der anderen Teilnehmer finden und die Nachrichten, die er erhält, mit dem öffentlichen Schlüssel des Absenders verifizieren. Dadurch kann sichergestellt werden, dass die Nachrichten tatsächlich vom angegebenen Absender stammen und nicht von einem anderen Teilnehmer gesendet wurden, der sich als jemand anderes ausgibt.

% #TODO: maybe add JavaScript code for the function calls (only if I need more pages)

