\section{Integration von Blockchain in das Protokoll}
\label{sec:blockchainintegration}

Bei der Auswahl der Blockchain fiel die Entscheidung auf Ethereum. Ethereum ist eine quelloffene,
dezentrale Plattform, die es ermöglicht, Smart Contracts zu erstellen und auszuführen. Smart Contracts sind Programme, die auf der Blockchain ausgeführt werden und die Möglichkeit bieten, Transaktionen und Verträge zu erstellen, die automatisch ausgeführt werden, sobald bestimmte Bedingungen erfüllt sind. Mit Hilfe der Smart Contracts können Benutzer des Protokolls ihre Identität verifizieren und öffentliche Schlüssel austauschen. 


Bei der Registrierung wird ein Smart Contract angestoßen, der die Registrierung des Benutzers auf der Blockchain durchführt. Da das Kademlia-Protokoll eine ID mit einer Länge von 160 Bit erwartet, wird ein Hash aus dem Benutzernamen mit Hilfe der SHA-256-Funktion erstellt. Dieser Hash wird dann durch die Verwendung der ersten 160 Bits auf die von Kademlia erwartete Länge für IDs gekürzt und als ID für den Benutzer verwendet. Bevor der Nutzer auf der Blockchain registriert wird, muss geprüft werden, ob der Benutzername oder der Hash, welcher aus dem Benutzernamen resultiert, bereits vergeben ist. Durch die Kürzung des Hashes auf 160 Bit ist es durchaus möglich dass zwei Benutzernamen den gleichen Hashwert haben und es somit zu Kollisionen kommt. Um dies zu verhindern, wird vor der Registrierung geprüft, ob der Benutzername bereits vergeben ist. Ist dies der Fall, wird der Benutzer so lange aufgefordert einen neuen Benutzernamen zu wählen, bis ein kollisionsfreier Benutzername gefunden wurde. Der Smart Contract wird mit dem Benutzernamen, dem Hash des Benutzernamens und den öffentlichen Schlüssel des Benutzers aufgerufen und erstellt einen neuen Eintrag in der Blockchain.

Wenn nun ein Nutzer mit einem anderen Nutzer kommunizieren möchte, muss er zunächst die ID des anderen Nutzers kennen. Dazu wird der zuvor, bei der Registrierung festgelegte, Benutzername auf der Blockchain gesucht und der dazugehörige Hashwert ausgelesen. Da dieser Hashwert gleichzeitig auch die ID des Benutzers im Kademlia-Netzwerk ist, kann somit eine direkte Verbindung zwischen den beiden Nutzern hergestellt werden.

Ein weiterer Nutzen, den die Blockchain für dieses Protokoll bietet, ist, dass die Herkunft der Nachrichten verifiziert werden kann. Da die öffentlichen Schlüssel der Nutzer auf der Blockchain gespeichert sind, kann jeder Nutzer die Nachrichten, die er erhält, mit dem öffentlichen Schlüssel des Absenders verifizieren. Somit kann sichergestellt werden, dass die Nachrichten tatsächlich vom angegebenen Absender stammen und nicht von einem anderen Nutzer gesendet wurden, der sich als jemand anderes ausgibt.



