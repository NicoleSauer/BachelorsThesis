\section{Identifikation von Teilnehmern}
\label{subsec:identifikation_von_teilnehmern}

Da es in Peer-to-Peer Netzwerken keinen zentralen Server gibt, der unter anderem zur Auffindung und Identifikation anderer Teilnehmer verwendet werden kann, müssen die Teilnehmer auf andere Weise identifiziert werden. Durch die Entscheidung, das Kademlia Protokoll zu implementieren, wird die ID des Teilnehmers als Schlüssel für die Speicherung der Teilnehmerinformationen, wie IP-Adresse und Port, verwendet.
Aus der Spezifikation von Kademlia geht hervor, dass die ID einer Node, welche auch als \textit{Kademlia-ID} bezeichnet wird, 160 Bit lang sein muss und aus einer zufälligen Kennung bestehen soll \parencite[S. 2]{Maymounkov_Kademlia}. Für das hier entwickelte Protokoll wird der Benutzername des Teilnehmers als ID verwendet. Dieser muss, wie aus den funktionalen Anforderungen zu entnehmen ist(siehe \ref{subsec:registrierung}), eindeutig sein und kann vom Benutzer bei der Registrierung frei gewählt werden. Um die Anforderung an die Länge zu erfüllen, werden die Zeichen des Benutzernamens auf 20 begrenzt und bei einer Formatierung in UTF-8 ergibt sich somit eine ID mit einer Länge von 160 Bit. Sollte der Benutzername kürzer als 20 Zeichen sein, wird er mit Nullen aufgefüllt, um die geforderte Länge zu erreichen.
Durch die Beschränkung der ID auf 20 Zeichen wird die Anzahl der möglichen Teilnehmer zwar begrenzt, doch daraus ergeben sich $2^{160}$ (65-stellige Zahl) mögliche IDs, was einer Anzahl von $1.461.501.637.330.902.918.203.684.832.716.283.019.655.932.542.976$ Teilnehmern entspricht. Diese Anzahl ist so groß, dass sie in der Praxis nicht erreicht werden sollte und somit die Beschränkung der ID keine Auswirkungen auf die Funktionalität des Protokolls haben sollte.

Wenn ein Teilnehmer eine Nachricht an einen anderen Teilnehmer senden möchte, muss er dessen IP-Adresse und Port kennen. Um an diese Informationen zu gelangen, muss zuerst die ID des Teilnehmers in der Blockchain gesucht werden und anschließend eine Peer-Discovery durchgeführt werden, um die IP-Adresse und den Port des Teilnehmers zu erhalten. Eine Peer-Discovery läuft wie folgt ab:

Dezentralisierte Netzwerke wie Kademlia verwenden Distributed Hash Tables (DHTs), um effizient Daten zu speichern und abzurufen. Im Kontext von Kademlia fungiert die DHT als Speichermechanismus für Informationen über die verfügbaren Teilnehmer im Netzwerk. Jede Node speichert normalerweise Informationen über andere Nodes in ihrer Nähe basierend auf ihrer Kademlia-ID und verwendet diese DHT, um schnell auf diese Daten zugreifen zu können. Die DHT ist in diesem Fall eine Routing-Tabelle, die die Teilnehmerinformationen enthält. Die Routing-Tabelle ist in Buckets aufgeteilt, wobei jeder Bucket eine bestimmte Distanz von der eigenen ID repräsentiert. Die Routing-Tabelle enthält 160 Buckets, wobei jeder Bucket eine Distanz von $2^{i}$ zu der eigenen ID repräsentiert, wobei $i$ die Nummer des Buckets ist. Jeder Bucket enthält eine Liste von Teilnehmern, die die Distanz des Buckets repräsentieren. Die Teilnehmer in einem Bucket sind nach der Zeit sortiert, in der sie zuletzt gesehen wurden, wobei der Teilnehmer, der zuletzt gesehen wurde, an erster Stelle steht. Die Routing-Tabelle wird verwendet, um die Teilnehmer zu finden, die eine bestimmte ID repräsentieren. Der Prozess beginnt mit der Berechnung der Distanz zwischen der eigenen ID und der ID des gesuchten Teilnehmers. Dazu wird die XOR-Operation auf den beiden IDs angewendet. Das Ergebnis ist eine 160 Bit lange Zahl, die die Distanz zwischen den beiden IDs repräsentiert. Diese Distanz wird nun in 160 Blöcke aufgeteilt, wobei jeder Block eine Bitlänge von 1 Bit hat. Die Blöcke werden von links nach rechts durchlaufen und die Bits werden von links nach rechts durchlaufen. Wenn ein Bit den Wert 1 hat, wird der Block in zwei Teile aufgeteilt und der Teil, der die ID des Teilnehmers repräsentiert, wird verwendet, um die nächste Node zu finden. Wenn ein Bit den Wert 0 hat, wird der Block nicht aufgeteilt und der Teil, der die eigene ID repräsentiert, wird verwendet, um die nächste Node zu finden. Dieser Vorgang wird solange wiederholt, bis die ID des Teilnehmers gefunden wurde. Wenn der Teilnehmer gefunden wurde, wird die IP-Adresse und der Port des Teilnehmers verwendet, um eine Verbindung zu ihm herzustellen.

% #TODO: Irgendwie unterbringen?
\textbf{Die Anzahl der Einträge in einem k-Bucket ist konfigurierbar und wird durch die Konstante \texttt{K} definiert.}