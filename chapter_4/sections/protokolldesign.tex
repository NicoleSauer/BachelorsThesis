\section{Protokollstruktur}
\label{sec:protokolldesign_und_struktur}

% Das Protokoll basiert auf Peer-to-Peer. Die direkte Verbindung mittels DHTs wird
% bevorzugt, erst wenn diese nicht möglich ist, wird die Nachricht über ein TCP-Relay
% weitergeleitet.

% #TODO: hashing in Betracht gezogen, aber da er nur dazu dient eine ID zu erzeugen bzw. man einen Nutzer daran identifizieren soll, ist das eigentlich nicht notwendig zu hashen. Es ist kryptografisch nicht notwendig
% #TODO: deshalb den Benutzernamen verwenden, der ja sowieso eindeutig sein muss (aus UX-Gründen), ihn auf 20 Zeichen limitieren (utf-8 -> 160 bits/8 = 20) und gegebenenfalls mit Nullen auffüllen, falls er kürzer ist


Die Architektur des Protokolls basiert auf Peer-to-Peer. Das bedeutet, dass alle Teilnehmer gleichberechtigt sind und es keinen zentralen Server gibt, der die Kommunikation steuert. Somit erfolgt die Kommunikation direkt zwischen den Teilnehmern. Alle Teilnehmer sind in einem Netzwerk organisiert, das aus verschiedenen Knoten besteht, wobei jeder Knoten einen Teilnehmer des Netzwerks repräsentiert. Alle Knoten im Netzwerk sind untereinander verbunden und können Textnachrichten austauschen.

\subsection{Grundlagen des Protokolls}

Für ein Peer-to-Peer Netzwerk gibt es verschiedene Typen. Abbildung \ref{p2p_typen} zeigt 
vier Typen und ihre Unterteilung in unstrukturierte und strukturierte Netzwerke.

\begin{center}
    \captionsetup{type=figure}
    \includegraphics[width=1\linewidth]{images/peer_to_peer_typen.png}
    \captionof{figure}{Typen von Peer-to-Peer Netzwerken \parencite{Luntovskyy_ModRechnernetze}}
    \label{p2p_typen}
\end{center}

\noindent Das Protokoll dieser Arbeit fällt in die Kategorie der hybriden Peer-to-Peer Netzwerke.
Es ist sowohl strukturiert als auch unstrukturiert. Alle Teilnehmer sind in einem Netzwerk organisiert,
das aus verschiedenen Knoten besteht, wobei jeder Knoten einen Teilnehmer des Netzwerks repräsentiert.
Alle Knoten im Netzwerk sind untereinander verbunden und können Textnachrichten direkt, ohne die Verwendung 
eines Servers austauschen. Sollte sich jedoch der gewünschte Empfänger nicht im Netzwerk befinden, besteht
die Möglichkeit, dass sich der Empfänger hinter einem NAT-Gateway oder einer Firewall befindet. In diesem
Fall wird die Nachricht über einen Server, genauer gesagt über ein TCP-Relay, an den Empfänger weitergeleitet.
Der Server ist nur für die Weiterleitung der Nachricht zuständig und speichert diese nicht. Dies ist
ein Unterschied zu anderen Instant-Messaging-Protokollen, wie zum Beispiel \textcolor{red}{[Beispiel 
finden oder Satz weglassen]}, bei dem der Server die Nachrichten speichert, wenn der Empfänger offline ist. 
Das Protokoll dieser Arbeit ist also ein hybrides Peer-to-Peer-Protokoll, da es sowohl strukturierte 
als auch unstrukturierte Eigenschaften aufweist.



\subsection{Identifikation von Teilnehmern}
\label{subsec:identifikation_von_teilnehmern}

Da es in Peer-to-Peer Netzwerken keinen zentralen Server gibt, der unter anderem zur Auffindung und Identifikation anderer Teilnehmer verwendet werden kann, müssen die Teilnehmer auf andere Weise identifiziert werden. Durch die Entscheidung, das Kademlia Protokoll zu implementieren, wird die ID des Teilnehmers als Schlüssel für die Speicherung der Teilnehmerinformationen, wie IP-Adresse und Port, verwendet.
Aus der Spezifikation von Kademlia geht hervor, dass die ID 160 Bit lang sein muss und aus einer zufälligen Kennung bestehen soll \parencite[S. 2]{Maymounkov_Kademlia}. Für das hier entwickelte Protokoll wird der Benutzername des Teilnehmers als ID verwendet. Dieser muss, wie aus den funktionalen Anforderungen zu entnehmen (siehe \ref{subsec:registrierung}), eindeutig sein und kann vom Benutzer bei der Registrierung frei gewählt werden. Um die Anforderung an die Länge zu erfüllen, werden die Zeichen des Benutzernamens auf 20 begrenzt und bei einer Formatierung in UTF-8 ergibt sich somit eine ID mit einer Länge von 160 Bit. Sollte der Benutzername kürzer als 20 Zeichen sein, wird er mit Nullen aufgefüllt, um die geforderte Länge zu erreichen.
Durch die Beschränkung der ID auf 20 Zeichen wird die Anzahl der möglichen Teilnehmer zwar begrenzt, doch daraus ergeben sich $2^{160}$ mögliche IDs, was einer Anzahl von $1.461.501.637.330.902.918.203.684.832.716.283.019.655.932.542.976$ Teilnehmern entspricht. Diese Anzahl ist so groß, dass sie in der Praxis nicht erreicht werden sollte und somit die Beschränkung der ID keine Auswirkungen auf die Funktionalität des Protokolls haben sollte.



\subsection{Nachrichtenformat}

Struktur für Nachrichten festlegen (Header, Body, etc.).
Verschlüsselung und Authentifizierung für Sicherheit hinzufügen.
Jedes Instant Messaging-Protokoll benötigt die Definition eines Nachrichtenformats,
das die Struktur der Nachrichten festlegt, die zwischen den Teilnehmern ausgetauscht werden.
Dieses Nachrichtenformat sollte die folgenden Elemente enthalten:
\\

\noindent \textbf{Nachrichtenkopf:}
\begin{itemize}
    \item Absender-ID: Eindeutige Kennung des Absenders (Benutzername, IP-Adresse?)
    \item Empfänger-ID: Identifikation des Empfängers (Benutzername, IP-Adresse?)
    \item Zeitstempel: Zeitpunkt, zu dem die Nachricht gesendet wurde
    \item Signatur: Signatur des Absenders, um die Authentizität der Nachricht zu gewährleisten
    \item \textcolor{red}{Reihenfolge}: Nummer der Nachricht, um die Reihenfolge der Nachrichten zu bestimmen
\end{itemize}

\noindent Im Nachrichtenkopf müssen sich die Absender- und Empfänger-IP-Adressen befinden, um die Nachrichten an die richtige Adresse zu senden. Die IP-Adresse des Absenders ist erforderlich, um eine Antwort auf die Nachricht zu senden. Die IP-Adresse des Empfängers ist erforderlich, um die Nachricht an den richtigen Empfänger zu senden. Der Zeitstempel ermöglicht es den Teilnehmern, die Nachrichten in der richtigen Reihenfolge anzuordnen. Die Signatur des Absenders ist erforderlich, um die Authentizität der Nachricht zu gewährleisten. Die Signatur wird mit dem privaten Schlüssel des Absenders erstellt und kann mit dem öffentlichen Schlüssel des Absenders überprüft werden.


\noindent \textbf{Nachrichteninhalt:}
\begin{itemize}
    \item Textinhalt: Der eigentliche Text der Nachricht
    \item Formatierung: Möglicherweise könnte man einfache Formatierungsoptionen 
    unterstützen, z. B. fett, kursiv, Unterstreichungen usw.
\end{itemize}

\noindent Der Nachrichteninhalt sollte den eigentlichen Text der Nachricht enthalten.


\noindent \textbf{Zustellbestätigung:}
\begin{itemize}
    \item Eine Option für eine Bestätigung der Nachrichtenzustellung könnte 
    hilfreich sein, um sicherzustellen, dass die Nachricht erfolgreich 
    übermittelt wurde
\end{itemize}


\noindent \textbf{Unicode-Unterstützung:}
\begin{itemize}
    \item Unterstützung erforderlich, um eine breite Palette von Sprachen und
    Zeichen darstellen zu können
\end{itemize}

\noindent \textbf{Verschlüsselung und Sicherheit:}
\begin{itemize}
    \item Verschlüsselung und Authentifizierung erforderlich, um die Vertraulichkeit der Nachrichten zu gewährleisten
\end{itemize}

\subsection{Routing und Peer Discovery}
\label{subsec:routing}

Wie bereits in Abschnitt \ref{subsec:identifikation_von_teilnehmern} 
\nameref{subsec:identifikation_von_teilnehmern} erwähnt,
wird das Kademlia-Protokoll verwendet, um die Knoten im Netzwerk zu identifizieren.
Es wird auch verwendet, um die Nachrichten zwischen den Knoten zu routen.

% Wenn ein Knoten einen Wert oder einen anderen Knoten im Netzwerk finden möchte, 
% führt er iterative Suchen durch, indem er eine auf XOR basierende Distanzmetrik 
% verwendet. Er fragt Knoten in seiner Routing-Tabelle basierend auf ihrer Nähe 
% zur Zielknoten-ID ab. Diese Abfragen helfen, die Suche auf das Ziel hin zu verfeinern, 
% indem sie sich mit jedem Schritt im Knoten-ID-Raum näher bewegen. Die iterative
% Suche wird beendet, wenn der Zielknoten gefunden wurde oder wenn die Suche
% keine Knoten mehr findet, die der Zielknoten-ID näher sind als der am nächsten
% gelegene Knoten, der bereits abgefragt wurde. In diesem Fall wird die Suche
% beendet und der am nächsten gelegene Knoten zurückgegeben.

% Im Kademlia-Protokoll sind vier Funktionen definiert, die für die Suche nach
% Knoten und Werten verwendet werden. Diese Funktionen sind \texttt{FIND\_NODE},
% \texttt{FIND\_VALUE}, \texttt{PING} und \texttt{STORE}. Die Funktionen
% \texttt{FIND\_NODE} und \texttt{FIND\_VALUE} werden verwendet, um nach Knoten
% oder Werten zu suchen. Die Funktion \texttt{PING} wird verwendet, um die
% Erreichbarkeit eines Knotens zu überprüfen. Die Funktion \texttt{STORE} wird
% verwendet, um einen Wert in einem Knoten zu speichern.


Das Kademlia-Protokoll basiert auf einem Distanzmetrik-Konzept, das als \\
"Kademlia-Distanz" bekannt ist. Jeder Knoten im Netzwerk wird durch eine eindeutige ID repräsentiert, 
typischerweise als kryptografischer Hashwert, der sich aus der gehashten IP-Adresse 
des jeweiligen Knotens mittels der SHA-1 Hashfunktion ergibt. Diese IDs sind in einem 
großen binären Baum organisiert, wobei die Position eines Knotens im Baum seine 
"Kademlia-Distanz" zu anderen Knoten definiert. Diese Distanz wird durch die XOR 
(ausschließendes Oder)-Operation ihrer eindeutigen IDs bestimmt. Die XOR-Operation 
ermöglicht eine effiziente Bestimmung der Distanz zwischen den Knoten-IDs, indem 
sie auf deren Binärzahlen angewendet wird. Dieses Ergebnis repräsentiert die 
Distanz zwischen den IDs und bildet die Grundlage für das Routing und die 
Organisation im Kademlia-Netzwerk.

Kademlia verwendet ein Routingverfahren, bei dem jeder Knoten eine Routing-\\
Tabelle 
speichert, die als "K-Buckets" bezeichnet werden. Jedes K-Bucket enthält Verweise 
auf andere Knoten im Netzwerk und ist nach der Kademlia-Distanz organisiert. Ein 
K-Bucket enthält typischerweise eine begrenzte Anzahl von Einträgen und gruppiert 
Knoten mit ähnlichen IDs.

Wenn ein Knoten eine Verbindung zu einem anderen Knoten herstellen muss, verwendet 
er die Routing-Tabelle, um den am nächsten gelegenen Knoten zu finden, der die 
Ziel-ID repräsentiert. Falls dieser Knoten nicht direkt bekannt ist, wird das 
Routing iterative durchgeführt, wobei der Knoten jeweils näher an der Ziel-ID 
liegende Knoten anfragt, bis der Zielknoten gefunden wird. Durch die Verwendung 
dieses strukturierten Ansatzes ermöglicht Kademlia eine effiziente Suche und 
Kommunikation zwischen Knoten in einem P2P-Netzwerk, wobei die Skalierbarkeit 
und Robustheit des Systems erhalten bleiben. Es ist ein Schlüsselelement vieler 
P2P-Anwendungen, einschließlich Filesharing, dezentraler Datenbanken und eben auch 
Instant Messaging-Protokollen, da es die Grundlage für die direkte 
Peer-to-Peer-Kommunikation schafft.
\\

\noindent Ein Beispiel für die XOR-Berechnung zwischen gehashten Knoten-IDs könnte wie folgt 
aussehen:
Knoten A hat die gehashte ID: 0x83a2c8f7,  
Knoten B hat die gehashte ID: 0xe1b6d4a9.

\noindent Die XOR-Operation zwischen diesen IDs ergibt:
\begin{equation}
    \begin{aligned}
        \text{Knoten A:} & \quad \texttt{0x83a2c8f7} \\
        \text{Knoten B:} & \quad \texttt{0xe1b6d4a9} \\
        \text{Ergebnis:} & \quad \texttt{0x61f47c5e}
    \end{aligned}
\end{equation}
% #TODO: Berechnung noch in binär umwandeln, damit die Berechnung besser verständlich ist?


\noindent Diese Distanz repräsentiert die Maßeinheit für die Positionierung und das Routing 
im Kademlia-Netzwerk. Knoten mit einer geringeren Distanz sind näher beieinander
als Knoten mit einer größeren Distanz. Dabei hat die Distanz nichts mit der
geographischen Entfernung zu tun, sondern nur mit der Position im Kademlia-Baum.

%\subsection{Status- und Präsenzinformationen}

Definition, wie Benutzer ihren Status aktualisieren (Online, Abwesend, Offline).
Überlegungen zur Präsenzinformation für effiziente Nachrichtenzustellung.

Sobald ein Nutzer sich mit dem Netzwerk verbunden hat, wird er als online angezeigt.
Sobald er sich abmeldet, wird er als offline angezeigt.

\subsection{Verbindungsmanagement}

Mechanismen für den Aufbau und die Beendigung von Peer-Verbindungen.
Pufferung von Nachrichten für offline Benutzer.

Die Herstellung einer Verbindung zwischen zwei Teilnehmern ist ein wichtiger
Aspekt eines Instant Messaging-Protokolls. Der Benutzer, der eine Verbindung
aufbauen möchte, muss die eindeutige ID des anderen Benutzers kennen. Diese
ID wird verwendet, um eine Suche im Kademlia-Netzwerk durchzuführen, um den
Knoten zu finden, der den Benutzer repräsentiert. Wenn der Knoten gefunden
wurde, kann eine Verbindung hergestellt werden, indem eine Verbindungsanfrage
gesendet wird. Wenn der Knoten die Anfrage akzeptiert, wird eine direkte Verbindung
zwischen den beiden Knoten hergestellt. Wenn die Verbindung erfolgreich hergestellt
wurde, können die Teilnehmer Nachrichten austauschen. Sollte der gesuchte Knoten
nicht gefunden werden, wird auf ein TCP-Relay zurückgegriffen, um die Nachricht
zu übermitteln. Dieses Relay ist ein Server, der die Nachrichten zwischen den
Teilnehmern weiterleitet. Dieser Mechanismus wird verwendet, wenn die direkte
Verbindung zwischen den Teilnehmern nicht möglich ist, z. B. wenn sich die Teilnehmer
hinter einer Firewall befinden. Die Verwendung eines TCP-Relays ist jedoch nicht
wünschenswert, da es die Skalierbarkeit des Systems beeinträchtigt und die
Vertraulichkeit der Nachrichten gefährdet. Daher sollte die direkte Verbindung
zwischen den Teilnehmern bevorzugt werden.


%\subsection{Verbindung und Identität}

Um eine Verbindung zwischen zwei Teilnehmern herzustellen, muss ein Teilnehmer den öffentlichen
Schlüssel des anderen Teilnehmers kennen. Dieser wurde bei der Registrierung in die Blockchain
geschrieben. Da in der Hashtabelle Schlüssel-Wert-Paare gespeichert werden, wird
der öffentliche Schlüssel als Schlüssel und die IP-Adresse und der Port des Teilnehmers als
Wert gespeichert. Der Teilnehmer, der eine Verbindung herstellen möchte, muss den öffentlichen
Schlüssel des anderen Teilnehmers kennen. Dazu muss er den öffentlichen Schlüssel in der
verteilten Hashtabelle suchen. Dazu berechnet er aus dem öffentlichen Schlüssel einen Hashwert.
Der Hashwert wird in einem Zahlenbereich abgebildet. Der Teilnehmer sucht in der verteilten
Hashtabelle nach dem nächsten Hashwert. Der Teilnehmer, der den Hashwert in der verteilten
Hashtabelle gespeichert hat, ist der Nachbar des Teilnehmers, der die Verbindung herstellen
möchte. Der Teilnehmer, der die Verbindung herstellen möchte, sendet eine Anfrage an den
Nachbarn. Die Anfrage enthält den öffentlichen Schlüssel des Teilnehmers, der die Verbindung
herstellen möchte. Der Nachbar antwortet mit seiner IP-Adresse und seinem Port. Der Teilnehmer,
der die Verbindung herstellen möchte, kann nun eine Verbindung zum Nachbarn aufbauen. Der
Nachbar leitet die Nachrichten an den Teilnehmer weiter, der die Verbindung herstellen möchte.

%\subsection{Nachrichten und Daten}

Benutzer können ausschließlich Textnachrichten miteinander austauschen. 
Diese Nachrichten können an einzelne Peers gesendet werden.
Die gesamte Kommunikation ist mit starken Verschlüsselungsalgorithmen 
verschlüsselt, um Privatsphäre und Sicherheit zu gewährleisten.

%\subsection{Nachrichtenverlauf}

Benutzer haben die Möglichkeit, ihren Nachrichtenverlauf lokal zu speichern.
Der Nachrichtenverlauf ist ebenfalls verschlüsselt, um die Sicherheit zu gewährleisten.

\subsection{Sicherheit}

\begin{itemize}
    \item Integration von Verschlüsselung für Datenschutz und Sicherheit 
    \item Schutz vor Angriffen wie Man-in-the-Middle
\end{itemize}

Alle Nachrichten sind mit öffentlichen Schlüsseln verschlüsselt, um sicherzustellen, 
dass nur der beabsichtigte Empfänger die Nachrichten entschlüsseln und lesen kann.
Ein sicheres Verfahren für den Schlüsselaustausch und die Schlüsselverwaltung 
wird implementiert, um gegen Abhören und Man-in-the-Middle-Angriffe zu schützen.

\subsection{Fehlerbehandlung}

Definition von Fehlercodes und deren Behandlung.
Wiederholungsmechanismen für verlorene oder fehlerhafte Nachrichten.
