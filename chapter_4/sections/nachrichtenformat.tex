\section{Nachrichtenformat}
\label{sec:nachrichtenformat}

Jedes Instant-Messaging-Protokoll benötigt die Definition eines Nachrichtenformats, das die Struktur der Nachrichten festlegt, die zwischen den Teilnehmern ausgetauscht werden. Im Folgenden wird das Nachrichtenformat für das Instant-Messaging-Protokoll beschrieben.

\subsection{Verbindungsanfrage}
Um eine Verbindung zwischen zwei Teilnehmern herzustellen, muss zunächst eine Anfrage gesendet werden. Diese Anfrage enthält die folgenden Informationen:

\begin{itemize}
    \item Nachrichtenformat: immer \textit{0x01}
    \item Benutzername des Senders
    \item Benutzername des Empfängers
    \item Timestamp
    \item Signatur
\end{itemize}

\noindent Das Nachrichtenformat ist eine binäre Zahl, die angibt, dass es sich um eine Anfrage handelt. Die Benutzernamen erlauben die Zuordnung der Anfrage. Der Timestamp enthält die aktuelle Zeit, zu der die Anfrage gesendet wird. Die Signatur dient der Authentifizierung des Senders (siehe \ref{subsec:vertrauliche_kommunikation} \textit{\nameref{subsec:vertrauliche_kommunikation}}). Aus Sicherheitsgründen wird jede Anfrage signiert.


\subsection{Nachrichtenaustausch}
Nachdem eine Verbindung zwischen zwei Teilnehmern hergestellt werden konnte, kann die Nachricht vom Sender an den Empfänger gesendet werden. Die Nachricht enthält die folgenden Informationen:

\begin{itemize}
    \item Nachrichtenformat: immer \textit{0x02}
    \item Benutzername des Senders
    \item Benutzername des Empfängers
    \item Timestamp
    \item Signatur
    \item Nachrichtenlänge
    \item Nachrichteninhalt
\end{itemize}

\noindent Bei der Erstellung des Nachrichtenformats wurde sich am Vorschlag von Day et al. der Etablierung eines Standards orientiert, der in RFC $2779$ mit dem Titel \textit{Instant Messaging/Presence Protocol Requirements} definiert ist \Parencite[S. 9]{rfc2779_IMPP}. 

Die ersten fünf Felder sind identisch mit denen der Verbindungsanfrage. Hinzu kommt die Nachrichtenlänge, die die Länge des eigentlichen Nachrichteninhalts angibt und der Nachrichteninhalt, der die eigentliche Nachricht enthält, die vom Sender an den Empfänger gesendet werden soll.
