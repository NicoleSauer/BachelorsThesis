\section{Nachrichtenformat}
\label{sec:nachrichtenformat}


Jedes Instant-Messaging-Protokoll benötigt die Definition eines Nachrichtenformats, das die Struktur der Nachrichten festlegt, die zwischen den Teilnehmern ausgetauscht werden. Im Folgenden wird das Nachrichtenformat für das Instant-Messaging-Protokoll beschrieben.
Nachdem eine Verbindung zwischen zwei Teilnehmern hergestellt werden konnte, kann die Nachricht vom Sender an den Empfänger gesendet werden. Die Nachricht wird in binär serialisierter Form übertragen und enthält die folgenden Informationen:

\begin{itemize}
    \item Benutzername des Senders
    \item Benutzername des Empfängers
    \item Timestamp
    \item Signatur
    \item Nachrichtenlänge
    \item Nachrichteninhalt
\end{itemize}

\noindent Bei der Erstellung des Nachrichtenformats wurde sich an \cite[S. 9]{rfc2779_IMPP} orientiert. Die Benutzernamen erlauben die Zuordnung der Nachricht. Der Timestamp enthält die aktuelle Zeit, zu der die Nachricht gesendet wird und wird für die richtige Sortierung der Nachrichten verwendet. Die Signatur dient der Authentifizierung des Senders (siehe \ref{subsec:vertrauliche_kommunikation} \textit{\nameref{subsec:vertrauliche_kommunikation}}). Aus Sicherheitsgründen wird jede Nachricht signiert. Die Nachrichtenlänge gibt die Länge des Nachrichteninhalts an und der Nachrichteninhalt enthält die eigentliche Nachricht, die vom Sender an den Empfänger gesendet werden soll.
