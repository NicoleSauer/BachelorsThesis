\section{Nachrichtenformat}
\label{sec:nachrichtenformat}

Jedes Instant-Messaging-Protokoll benötigt die Definition eines Nachrichtenformats, das die Struktur der Nachrichten festlegt, die zwischen den Teilnehmern ausgetauscht werden. Im Folgenden wird das Nachrichtenformat für das Instant-Messaging-Protokoll beschrieben. 

Nachdem eine Verbindung zwischen zwei Teilnehmern hergestellt werden konnte, kann die Nachricht auf dem direkten Weg vom Sender an den Empfänger gesendet werden. Die Nachrichten werden in zwei Teile unterteilt: den Nachrichtenkopf und den Nachrichteninhalt. Der Nachrichtenkopf enthält die Informationen, die für die Zustellung der Nachricht erforderlich sind. Der Nachrichteninhalt enthält den eigentlichen Text der Nachricht.

\subsection{Nachrichtenkopf}
\label{subsec:nachrichtenkopf}

Der Nachrichtenkopf enthält die Informationen, die für die Zustellung der Nachricht erforderlich sind. Der Nachrichtenkopf besteht aus der Empfänger-ID, dem Zeitstempel und der Signatur.

\subsubsection{Empfänger-ID}
\label{subsubsec:empfaenger-id}


Im Nachrichtenkopf muss sich der Benutzername des Empfängers befinden, damit im Netzwerk nach dem Empfänger gesucht werden kann. Die Empfänger-ID ist der Benutzername des Empfängers. Der Benutzername ist eindeutig und wird bei der Registrierung in der Blockchain gespeichert. Somit kann jeder Teilnehmer einen anderen Teilnehmer anhand seiner ID identifizieren.

\subsubsection{Zeitstempel}
\label{subsubsec:zeitstempel}

Der Zeitstempel ermöglicht zum einen, die Anordnung der Nachrichten in der richtigen Reihenfolge und zum anderen, der Feststellung des Zeitpunkts, zu dem die Nachricht gesendet wurde. Der Zeitstempel wird in Millisekunden angegeben und ist eine 64-Bit-Ganzzahl. Die Zeit wird in Millisekunden angegeben, da die Genauigkeit in Sekunden nicht ausreichend ist. Die Genauigkeit in Millisekunden ist erforderlich, um die Reihenfolge der Nachrichten zu bestimmen. Wenn zwei Nachrichten mit demselben Zeitstempel empfangen werden, wird die Nachricht mit dem niedrigeren Zeitstempel zuerst angezeigt.

\subsubsection{Signatur}
\label{subsubsec:signatur}

Die Signatur des Absenders ist erforderlich, um die Authentizität der Nachricht zu gewährleisten. Die Signatur wird mit dem privaten Schlüssel des Absenders erstellt und kann mit dem öffentlichen Schlüssel des Absenders überprüft werden. Die Signatur wird mit dem SHA-256-Algorithmus erstellt. Der SHA-256-Algorithmus ist eine kryptographische Hashfunktion, die eine 256-Bit-Hashsumme erzeugt. Die Hashfunktion wird verwendet, um die Signatur zu erstellen, indem der Hashwert des Nachrichteninhalts mit dem privaten Schlüssel des Absenders verschlüsselt wird. Die Signatur wird mit dem öffentlichen Schlüssel des Absenders überprüft, indem der Hashwert des Nachrichteninhalts mit der Signatur des Absenders entschlüsselt wird. Wenn der Hashwert des Nachrichteninhalts mit der Signatur des Absenders entschlüsselt werden kann, ist die Signatur gültig. Die Verwendung von asymmetrischer Kryptographie ermöglicht die Authentifizierung der Benutzer und die Vertraulichkeit der Nachrichten. Die Vertraulichkeit der Nachrichten wird durch die Verschlüsselung der Nachrichten mit dem öffentlichen Schlüssel des Empfängers gewährleistet. Die Authentizität der Nachrichten wird durch die Signatur des Absenders gewährleistet.

\subsection{Nachrichteninhalt}
\label{subsec:nachrichteninhalt}

Der Nachrichteninhalt sollte den eigentlichen Text der Nachricht enthalten. Unicode-Unterstützung ist heutzutage erforderlich, um die Verwendung von Sonderzeichen und Emojis zu ermöglichen. UTF-8 ist ein variables Zeichenkodierungsformat, das für Unicode definiert wurde. Es kann Zeichen mit einer Länge von 1 bis 4 Byte kodieren \parencite[S. 4]{rfc3629_utf-8}. UTF-8 ist das bevorzugte Encoding für Instant Messaging, da es die Unterstützung für eine breite Palette von Sprachen und Zeichen bietet. UTF-32 wurde anfangs in Betracht gezogen, da es in der Lage ist jeden Unicode-Codepunkt in einem 32-Bit-Wort zu kodieren. Es ist jedoch nicht sehr effizient, da es durch die feste Länge von 4 Byte für jedes Zeichen vergleichsweise viel Speicherplatz benötigt. Auch die mögliche Verwendung von UTF-16 wurde untersucht, da es ähnlich wie UTF-8 eine variable Länge von 1 bis 4 Byte für jedes Zeichen verwendet.

\subsection{Verschlüsselung und Sicherheit}
\label{subsec:verschluesselung_und_sicherheit}

Die Authentifizierung der Benutzer erfolgt über die Verwendung von asymmetrischer Kryptographie. Jeder Benutzer besitzt ein Schlüsselpaar, das aus einem öffentlichen und einem privaten Schlüssel besteht. Der öffentliche Schlüssel wird verwendet, um Nachrichten zu verschlüsseln und die Signatur des Absenders zu überprüfen. Der private Schlüssel wird verwendet, um Nachrichten zu entschlüsseln und die Signatur zu erstellen. Die Verwendung von asymmetrischer Kryptographie ermöglicht die Authentifizierung der Benutzer und die Vertraulichkeit der Nachrichten. Die Vertraulichkeit der Nachrichten wird durch die Verschlüsselung der Nachrichten mit dem öffentlichen Schlüssel des Empfängers gewährleistet. Die Authentizität der Nachrichten wird durch die Signatur des Absenders gewährleistet.