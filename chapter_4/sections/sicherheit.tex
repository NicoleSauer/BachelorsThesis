\section{Sicherheit}
\label{subsec:sicherheit}

% #TODO: Checken, ob das nach der Umstrukturierung noch so passt
Um die Kommunikation zwischen den Teilnehmern zu schützen, finden sowohl asymmetrische als auch symmetrische Verschlüsselung Anwendung. Die asymmetrische Verschlüsselung wird für die Authentifizierung und die symmetrische Verschlüsselung für die Ende-zu-Ende-Verschlüsselung der Nachrichten verwendet. 

\subsection{Vertrauliche Kommunikation}
\label{subsec:vertrauliche_kommunikation}

Die Kommunikation zwischen den Teilnehmern soll vertraulich sein. Das bedeutet, dass die Nachrichten nur von den Teilnehmern gelesen werden können, die an der Kommunikation beteiligt sind. Um dies gewährleisten zu können, wird eine Ende-zu-Ende-Verschlüsselung verwendet. Ende-zu-Ende-Verschlüsselung ist ein Konzept, das besagt, dass die Nachrichten vom Sender verschlüsselt und erst beim Empfänger wieder entschlüsselt werden \parencite[S. 233-260]{Wong_KryptoPraxis}. Die folgende Abbildung \ref{fig:ende_zu_ende} zeigt den Ablauf einer Ende-zu-Ende-Verschlüsselung.

% Signatur mit EdDSA -> warum habe ich mich dafür entschieden?
% Key Derivation Function
% symmetrische Verschlüsselung mit AES (state of the art <- Quelle)
% Perfect Forward Secrecy erwähnen
% Ed25519 (EdDSA) -> warum habe ich mich dafür entschieden? Signatur

\begin{figure}[H]
    \centering
    \includegraphics[width=1\linewidth]{images/end2end.png}
    \caption{Ende-zu-Ende-Verschlüsselung}
    \label{fig:ende_zu_ende}
\end{figure}

\noindent In den folgenden Abschnitten werden die Blöcke aus Abbildung \ref{fig:ende_zu_ende} näher erläutert.

\subsubsection{Authentifizierter Diffie-Hellman Schlüsselaustausch}
\label{subsec:ecdh}

Zuerst müssen sich die Teilnehmer auf einen gemeinsamen Schlüssel einigen. Dies nennt man auch \textit{Schlüsselaustausch}. Für die Implementierung eines Schlüsselaustausch kamen zwei Verfahren in Frage. Zum einen der \textit{Diffie-Hellman-(DH-)}Schlüsselaustausch und zum anderen der \textit{Elliptic Curve Diffie-Hellman-(ECDH-)}Schlüsselaustausch. Grundsätzlich baut der Diffie-Hellman-Schlüsselaustausch auf das mathematische Gebiet der \textit{Gruppentheorie} auf. Der DH-Schlüsselaustausch ist ein Schlüsselaustauschverfahren, das auf dem diskreten Logarithmusproblem basiert. Das diskrete Logarithmusproblem ist ein Problem aus der Kryptographie, das besagt, dass es schwierig ist, den Exponenten $x$ zu berechnen, wenn nur die Basis $g$ und das Ergebnis $y$ bekannt sind. Der ECDH-Schlüsselaustausch hingegen basiert auf elliptischen Kurven. David Wong empfiehlt in seinem Buch \textit{Kryptografie in der Praxis} ECDH zu verwenden, da die Schlüssel kleiner sind und noch keine starken Angriffe gegen das Verfahren gefunden wurden \Parencite[S. 101-125]{Wong_KryptoPraxis}.

Aus diesem Grund wurde sich für den ECDH-Schlüsselaustausch entschieden. Hierfür muss eine elliptische Kurve definiert werden. Wong zählt hierfür zwei Kurven auf, die von den meisten Anwendungen verwendet werden: \textit{P-256} und \textit{Curve25519}. Da die Erzeugung der Kurve P-256 unklar ist und damit die Vertrauenswürdigkeit darunter leidet, wurde sich für Curve25519 entschieden \Parencite[S. 121]{Wong_KryptoPraxis}. Den ECDH-Schlüsselaustausch mit Curve25519 nennt man auch X25519 und wird in \cite{rfc_ietf_curve25519} spezifiziert.

Solch ein Schlüsselaustauschverfahren für sich genommen, hat den Nachteil, dass ein aktiver Angreifer die Kommunikation zwischen den Teilnehmern abhören und manipulieren kann. Um dies zu verhindern, wird ein \textit{authentifizierter} Schlüsselaustausch verwendet. Dieser setzt sich zusammen aus dem Schlüsselaustauschverfahren und einer Signatur. Die Signatur wird mit dem privaten statischen Schlüssel des Senders erstellt und kann mit dem öffentlichen statischen Schlüssel des Senders verifiziert werden. Die Wahl eines Signaturverfahrens ist nicht trivial. Wong zählt zwei moderne Verfahren auf: \textit{Elliptic Curve Digital Signature-Algorithmus (ECDSA)} und \textit{Edwards-curve Digital Signature Algorithm (EdDSA)}. Dabei erwähnt er auch Vertrauensbedenken gegenüber ECDSA, weshalb sich für EdDSA entschieden wurde. EdDSA ist ein Signaturverfahren, das auf elliptischen Kurven basiert, weshalb auch hier eine Kurve ausgewählt werden muss. Wong beschreibt, dass in der Praxis meist die Kurve \textit{Edwards25519} verwendet wird. Diese Kombination nennt man auch Ed25519 und wird in \cite{rfc_EdDSA} spezifiziert \Parencite[S. 160-172]{Wong_KryptoPraxis}.

Im Block \textit{Authentifizierter Diffie-Hellman Schlüsselaustausch} aus Abbildung \ref{fig:ende_zu_ende} erzeugt Alice ein neues X25519-Schlüsselpaar und signiert ihren öffentlichen Schlüssel unter Verwendung des privaten statischen Ed25519-Schlüssels. Dieser signierte öffentliche Schlüssel wird an Bob gesendet. Bob erzeugt ebenfalls ein X25519-Schlüsselpaar und signiert seinen öffentlichen Schlüssel mit seinem privaten statischen Ed25519-Schlüssel. Dieser signierte öffentliche Schlüssel wird an Alice gesendet. Alice und Bob können nun das gemeinsame Geheimnis berechnen (siehe \ref{fig:schluesselvereinbarung} \textit{\nameref{fig:schluesselvereinbarung}}).

\subsubsection{Berechnung des gemeinsamen Schlüssels}

Um aus diesem Geheimnis ein möglichst gleichverteiltes Geheimnis zu erzeugen, wird eine \textit{Schlüsselableitungsfunktion} (englisch: \textit{Key Derivation Function}) verwendet. Eine Schlüsselableitungsfunktion ist eine Funktion, die aus einem Eingabeschlüssel (hier das Geheimnis) einen Ausgabeschlüssel erzeugt. Dieser wird dann für die eigentliche Verschlüsselung der Nachricht verwendet. Laut Wong kann dafür eine einfache Hashfunktion (siehe \ref{subsec:vertraulichkeit_basics} \textit{\nameref{subsec:vertraulichkeit_basics}}) verwendet werden, wenn die Ausgabelänge für den Ausgabeschlüssel passend ist \Parencite[S. 194]{Wong_KryptoPraxis}. In diesem Fall wird die \textit{SHA-256} Hashfunktion verwendet, da diese eine Ausgabelänge von $256$ Bit hat \Parencite{rfc6234_SHA2}.


\subsubsection{Symmetrische Verschlüsselung}





%\noindent Beim Aufbau einer Verbindung wird ein Diffie-Hellman-Schlüsselaustausch durchgeführt. Dieser Schlüsselaustausch wird verwendet, um einen symmetrischen Schlüssel für die Ende-zu-Ende-Verschlüsselung zu erzeugen. Der Diffie-Hellman-Schlüsselaustausch wird in Abbildung \ref{fig:ende_zu_ende} dargestellt. 

%Zu Beginn erzeugt Alice ein flüchtiges Schlüsselpaar. Flüchtige Schlüssel sind Schlüssel, die nur für eine Verbindung verwendet werden. Nach der Verwendung werden sie verworfen \parencite{NIST_ephemeralKey}. Diesen Schlüssel signiert sie mit ihrem privaten statischen Schlüssel und sendet ihn an Bob. Bob kann die Signatur mit dem öffentlichen statischen Schlüssel von Alice verifizieren, da dieser bei Alice´ Registrierung in die Blockchain geschrieben wurde und mit Hilfe einer Funktion aus der Blockchain gelesen werden kann. Bob erzeugt nun ebenfalls ein flüchtiges Schlüsselpaar und signiert den flüchtigen öffentlichen Schlüssel mit seinem privaten statischen Schlüssel. Diesen signierten Schlüssel sendet er an Alice. Auch Alice kann nun die Signatur durch das Hinterlegen des öffentlichen statischen Schlüssels von Bob in der Blockchain prüfen. Der nächste Schritt ist die Berechnung des gemeinsamen Geheimnis. Dazu wird der flüchtige öffentliche Schlüssel des anderen Teilnehmers mit dem eigenen privaten flüchtigen Schlüssel multipliziert (?). Das Ergebnis ist das gemeinsame Geheimnis. Dieses wird nun mit einer \textit{Schlüsselableitungsfunktion} (englisch: \textit{Key Derivation Function}) in einen symmetrischen Schlüssel umgewandelt, mit dem die Nachrichten verschlüsselt werden. Alice kann nun eine, mit dem symmetrischen Schlüssel verschlüsselte, Nachricht an Bob senden und Bob ist in der Lage diese Nachricht mit dem symmetrischen Schlüssel zu entschlüsseln. 

