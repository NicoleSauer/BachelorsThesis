\section{Sicherheit}
\label{subsec:sicherheit}

Um die Kommunikation zwischen den Teilnehmern zu schützen, finden sowohl asymmetrische als auch symmetrische Verschlüsselung Anwendung. Die asymmetrische Verschlüsselung wird für die Authentifizierung und die symmetrische Verschlüsselung für die Ende-zu-Ende-Verschlüsselung der Nachrichten verwendet. Die Implementierung der asymmetrischen Verschlüsselung basiert auf Public-Key-Kryptographie. In dem von Whitfield Diffie und Martin Hellman 1976 veröffentlichten Paper \textit{New Directions in Cryptography} \parencite{DiffieHellman_NewDirectionsInCryptography} wurde die Public-Key-Kryptographie erstmals beschrieben. Darin wird die Problematik der symmetrischen Verschlüsselung beschrieben, dass ein Schlüssel für die Verschlüsselung und Entschlüsselung verwendet wird und dieser Schlüssel zu Beginn der Kommunikation über einen unsicheren Kanal zwischen den Teilnehmern ausgetauscht werden muss. Die Lösung dieses Problems ist die Public-Key-Kryptographie.