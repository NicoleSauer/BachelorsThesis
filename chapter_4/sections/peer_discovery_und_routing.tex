 \section{Peer-Discovery und Routing}
\label{subsec:identifikation_von_teilnehmern}

Da es in Peer-to-Peer-Netzwerken keinen zentralen Server gibt, der unter anderem zur Auffindung und Identifikation anderer Teilnehmer verwendet werden kann, müssen die Teilnehmer auf andere Weise identifiziert werden. Da sich im vorangegangenen Abschnitt für die Implementierung von Kademlia entschieden wurde, wird in diesem Abschnitt beschrieben, wie die Teilnehmer identifiziert werden. Um einen Teilnehmer zu identifizieren, wird eine ID benötigt. Hierfür wird der Benutzername des Teilnehmers verwendet, welchen er bei der Registrierung frei wählen kann. 

% Die Anzahl der in den Buckets gespeicherten Knoten muss zu Beginn festgelegt werden. In dieser Arbeit wird festgelegt, dass jeder Bucket 10 Knoten speichern kann. Die Anzahl der Buckets selbst, wird nicht festgelegt, da diese dynamisch ist und von Kademlia übernommen wird. Auch wird festgelegt, dass als Kademlia-ID der Benutzername des Teilnehmers verwendet wird. Anhand dieser ID können Teilnehmerinformationen, wie IP-Adresse und Port, im Netzwerk gesucht werden.

Aus der Spezifikation von Kademlia geht hervor, dass die ID einer Node, welche auch als \textit{Kademlia-ID} bezeichnet wird, aus einer zufälligen Kennung bestehen soll \parencite[S. 2]{Maymounkov_Kademlia}. Für das hier entwickelte Protokoll wird, wie bereits oben festgelegt, der Benutzername als ID verwendet. Das führt dazu, dass der Benutzername eindeutig sein muss, da sonst mehrere Teilnehmer die gleiche ID haben könnten. Um den Benutzer später identifizieren zu können, wird bei der Registrierung ein statisches Schlüsselpaar generiert. Der öffentliche Schlüssel wird zusammen mit dem Benutzernamen in der Blockchain gespeichert (siehe \ref{subsec:contract_registrierung} \textit{\nameref{subsec:contract_registrierung}}). Der private Schlüssel wird lokal auf dem Gerät des Teilnehmers gespeichert. 

Außerdem wird eine ID mit einer Länge von 160 Bit erwartet. Daher muss der Benutzername auf diese Länge gebracht werden. Dafür wird eine Zeichenbegrenzung von 20 Zeichen festgelegt, was bei einer UTF-8 Codierung einer Bitlänge von 160 Bit entspricht \parencite{rfc3629_utf-8}. Falls der Benutzername kürzer als 20 Zeichen ist, wird er mit Nullen aufgefüllt. Durch die Beschränkung der ID auf 20 Zeichen wird die Anzahl der Teilnehmer zwar begrenzt, jedoch ist die Anzahl der möglichen Teilnehmer immer noch sehr groß. Es können in der Theorie $2^{160}$ Teilnehmer am Netzwerk teilnehmen, wobei nicht jedes Zeichen in UTF-8 sinnvoll für die Verwendung in einem Benutzername ist. Somit ist die Anzahl der möglichen Teilnehmer in der Praxis geringer, aber immer noch groß genug, um einem Instant-Messaging-Protokoll zu genügen. In der Praxis sollte diese Zahl nicht erreicht werden, wodurch sich die Entscheidung, die Länge des Benutzernamens zu beschränken, nicht negativ auf die Funktionalität des Protokolls auswirken sollte.
In der Node werden dann Schlüssel-Wert-Paare gespeichert, wobei der Schlüssel die ID des Teilnehmers ist und der Wert die IP-Adresse und der Port. Es wird festgelegt, dass immer Port $49152$ verwendet wird, da sich dieser Port im Bereich der dynamischen Ports befindet und somit nicht für andere Anwendungen reserviert ist \parencite[S. 20]{rfc6335_IANA_Ports}. 

Wenn ein Teilnehmer eine Nachricht an einen anderen Teilnehmer senden möchte, muss er dessen IP-Adresse kennen. Um an diese Information zu gelangen, muss zuerst die ID des Teilnehmers in der Blockchain gesucht werden und anschließend eine Peer-Discovery durchgeführt werden, um die IP-Adresse des Teilnehmers zu erhalten. Eine Peer-Discovery läuft wie folgt ab:

\begin{enumerate}
    \item Alice sendet eine \textit{FIND\_NODE}-Nachricht an den Knoten in ihrer Routing-Tabelle, der der Kademlia-ID von Bob am nächsten ist. 
    \item Der Knoten sucht in seiner Routing-Tabelle nach der Kademlia-ID von Bob. Falls er den Knoten findet, antwortet er mit den gefundenen Informationen in Form eines Tripletts aus Kademlia-ID, IP-Adresse und Port. Falls dieser Knoten den gesuchten Knoten nicht findet, antwortet er mit mehreren Triplets, die dem Knoten mit der Kademlia-ID von Bob am nächsten sind.
    \item Alice erhält die Antwort und sucht in der Antwort nach der Kademlia-ID von Bob. Falls sie gefunden wird, speichert Alice die IP-Adresse und den Port von Bob in ihrer Routing-Tabelle ab. Falls sie nicht gefunden wird, wiederholt Alice Schritt 2 solange, bis sie keine Antworten mehr erhält, die IDs beinhalten, die der Kademlia-ID von Bob näher sind als die ID des Knotens, der die \textit{FIND\_NODE}-Nachricht erhalten hat.
\end{enumerate}

% #TODO: falls mehr zum pingen benötigt wird, siehe Kademlia vs. Chord Paper
\noindent Um das Netzwerk immer aktuell zu halten, wird in regelmäßigen Abständen ein \textit{PING} an alle Teilnehmer gesendet, die in den Routing-Tabellen der Nodes gespeichert sind. Falls ein Teilnehmer nicht antwortet, wird er aus der Routing-Tabelle entfernt.
