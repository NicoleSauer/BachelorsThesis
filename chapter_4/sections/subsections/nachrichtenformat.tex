\subsection{Nachrichtenformat}

Struktur für Nachrichten festlegen (Header, Body, etc.).
Verschlüsselung und Authentifizierung für Sicherheit hinzufügen.

\noindent Nachrichtenkopf:
\begin{itemize}
    \item Absender-ID: Eindeutige Kennung des Absenders
    \item Empfänger-ID: Identifikation des Empfängers
    \item Zeitstempel: Zeitpunkt, zu dem die Nachricht gesendet wurde
    \item Nachrichten-ID: Eindeutige Kennung der Nachricht für die Identifikation 
    und Verfolgung
\end{itemize}


\noindent Nachrichteninhalt:
\begin{itemize}
    \item Textinhalt: Der eigentliche Text der Nachricht.
    \item Formatierung: Möglicherweise könntest du einfache Formatierungsoptionen 
    unterstützen, z. B. fett, kursiv, Unterstreichungen usw.
\end{itemize}


\noindent Zusätzliche Tags:
\begin{itemize}
    \item Priorität: Wenn Nachrichten eine Priorität haben sollen, könnte 
    man Tags oder Flags einführen, um die Wichtigkeit zu markieren 
    (z. B. "Dringend", "Normal", "Niedrige Priorität").
\end{itemize}


\noindent Zustellbestätigung:
\begin{itemize}
    \item Eine Option für eine Bestätigung der Nachrichtenzustellung könnte 
    hilfreich sein, um sicherzustellen, dass die Nachricht erfolgreich 
    übermittelt wurde.
\end{itemize}


\noindent Unicode-Unterstützung:
\begin{itemize}
    \item Unterstützung erforderlich, um eine breite Palette von Sprachen und
    Zeichen darstellen zu können
\end{itemize}

\noindent Verschlüsselung und Sicherheit:
\begin{itemize}
    \item Verschlüsselung und Authentifizierung erforderlich, um die 
    Vertraulichkeit der Nachrichten zu gewährleisten
\end{itemize}