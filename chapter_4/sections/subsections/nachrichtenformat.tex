\subsection{Nachrichtenformat}

Struktur für Nachrichten festlegen (Header, Body, etc.).
Verschlüsselung und Authentifizierung für Sicherheit hinzufügen.
Jedes Instant Messaging-Protokoll benötigt die Definition eines Nachrichtenformats,
das die Struktur der Nachrichten festlegt, die zwischen den Teilnehmern ausgetauscht werden.
Dieses Nachrichtenformat sollte die folgenden Elemente enthalten:

\noindent Nachrichtenkopf:
\begin{itemize}
    \item Absender-ID: Eindeutige Kennung des Absenders (Benutzername? Public Key?)
    \item Empfänger-ID: Identifikation des Empfängers (Benutzername? Public Key?)
    \item Zeitstempel: Zeitpunkt, zu dem die Nachricht gesendet wurde
    \item Nachrichten-ID: Eindeutige Kennung der Nachricht für die Identifikation 
    und Verfolgung
\end{itemize}

Im Nachrichtenkopf müssen sich die Absender- und Empfänger-IDs befinden,
um die Kommunikation zwischen den Teilnehmern zu ermöglichen.
Die Nachrichten-ID ist erforderlich, um die Nachrichten zu identifizieren und
die Zustellung zu verfolgen. Der Zeitstempel ist hilfreich, um die Nachrichten
zu sortieren und die Reihenfolge der Nachrichten zu bestimmen.


\noindent Nachrichteninhalt:
\begin{itemize}
    \item Textinhalt: Der eigentliche Text der Nachricht
    \item Formatierung: Möglicherweise könnte man einfache Formatierungsoptionen 
    unterstützen, z. B. fett, kursiv, Unterstreichungen usw.
\end{itemize}

Der Nachrichteninhalt sollte den eigentlichen Text der Nachricht enthalten.


\noindent Zusätzliche Tags:
\begin{itemize}
    \item Priorität: Wenn Nachrichten eine Priorität haben sollen, könnte 
    man Tags oder Flags einführen, um die Wichtigkeit zu markieren 
    (z. B. "Dringend", "Normal", "Niedrige Priorität")
\end{itemize}


\noindent Zustellbestätigung:
\begin{itemize}
    \item Eine Option für eine Bestätigung der Nachrichtenzustellung könnte 
    hilfreich sein, um sicherzustellen, dass die Nachricht erfolgreich 
    übermittelt wurde
\end{itemize}


\noindent Unicode-Unterstützung:
\begin{itemize}
    \item Unterstützung erforderlich, um eine breite Palette von Sprachen und
    Zeichen darstellen zu können
\end{itemize}

\noindent Verschlüsselung und Sicherheit:
\begin{itemize}
    \item Verschlüsselung und Authentifizierung erforderlich, um 
    die Vertraulichkeit der Nachrichten zu gewährleisten
\end{itemize}