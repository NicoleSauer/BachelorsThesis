\subsection{Nachrichtenformat}

Jedes Instant-Messaging-Protokoll benötigt die Definition eines Nachrichtenformats, das die Struktur der Nachrichten festlegt, die zwischen den Teilnehmern ausgetauscht werden. Im Folgenden wird das Nachrichtenformat für das Instant-Messaging-Protokoll beschrieben. 

Nachdem die IP-Adresse und Port des Empfängers ermittelt wurden, kann die Nachricht auf dem direkten Weg an den Empfänger gesendet werden. Die Nachrichten werden in zwei Teile unterteilt: den Nachrichtenkopf und den Nachrichteninhalt. Der Nachrichtenkopf enthält die Informationen, die für die Zustellung der Nachricht erforderlich sind. Der Nachrichteninhalt enthält den eigentlichen Text der Nachricht. 
\\

\noindent \textbf{Nachrichtenkopf:}
\begin{itemize}
    \item Empfänger-ID: Identifikation des Empfängers = Benutzername
    \item Zeitstempel: Zeitpunkt, zu dem die Nachricht gesendet wurde
    \item Signatur: Signatur des Absenders, um die Authentizität der Nachricht zu gewährleisten
\end{itemize}

\noindent Im Nachrichtenkopf muss sich der Benutzername des Empfängers befinden, damit im Netzwerk nach dem Empfänger gesucht werden kann. Der Zeitstempel ermöglicht zum einen, die Anordnung der Nachrichten in der richtigen Reihenfolge und zum anderen, der Feststellung des Zeitpunkts, zu dem die Nachricht gesendet wurde. Die Signatur des Absenders ist erforderlich, um die Authentizität der Nachricht zu gewährleisten. Die Signatur wird mit dem privaten Schlüssel des Absenders erstellt und kann mit dem öffentlichen Schlüssel des Absenders überprüft werden.
\\


\noindent \textbf{Nachrichteninhalt:}
\begin{itemize}
    \item Nachrichtentext: Der eigentliche Text der Nachricht
    \item Encoding in UTF-8
\end{itemize}

\noindent Der Nachrichteninhalt sollte den eigentlichen Text der Nachricht enthalten. Unicode-Unterstützung ist heutzutage erforderlich, um die Verwendung von Sonderzeichen und Emojis zu ermöglichen. UTF-8 ist ein variables Zeichenkodierungsformat, das für Unicode definiert wurde. Es kann Zeichen mit einer Länge von 1 bis 4 Byte kodieren \parencite[S. 4]{rfc3629_utf-8}. UTF-8 ist das bevorzugte Encoding für Instant Messaging, da es die Unterstützung für eine breite Palette von Sprachen und Zeichen bietet. UTF-32 wurde anfangs in Betracht gezogen, da es in der Lage ist jeden Unicode-Codepunkt in einem 32-Bit-Wort zu kodieren. Es ist jedoch nicht sehr effizient, da es durch die feste Länge von 4 Byte für jedes Zeichen vergleichsweise viel Speicherplatz benötigt. Auch die mögliche Verwendung von UTF-16 wurde untersucht, da es eine ähnlich wie UTF-8 eine variable Länge von 1 bis 4 Byte für jedes Zeichen verwendet.  
\\


\noindent \textbf{Verschlüsselung und Sicherheit:}
\begin{itemize}
    \item Verschlüsselung und Authentifizierung erforderlich, um die Vertraulichkeit der Nachrichten zu gewährleisten
\end{itemize}

\noindent Die Authentifizierung der Benutzer erfolgt über die Verwendung von asymmetrischer Kryptographie. Jeder Benutzer besitzt ein Schlüsselpaar, das aus einem öffentlichen und einem privaten Schlüssel besteht. Der öffentliche Schlüssel wird verwendet, um Nachrichten zu verschlüsseln und die Signatur des Absenders zu überprüfen. Der private Schlüssel wird verwendet, um Nachrichten zu entschlüsseln und die Signatur zu erstellen. Die Verwendung von asymmetrischer Kryptographie ermöglicht die Authentifizierung der Benutzer und die Vertraulichkeit der Nachrichten. Die Vertraulichkeit der Nachrichten wird durch die Verschlüsselung der Nachrichten mit dem öffentlichen Schlüssel des Empfängers gewährleistet. Die Authentizität der Nachrichten wird durch die Signatur des Absenders gewährleistet. 