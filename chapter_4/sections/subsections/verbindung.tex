\subsection{Verbindung und Identität}

Um eine Verbindung zwischen zwei Teilnehmern herzustellen, muss ein Teilnehmer den öffentlichen
Schlüssel des anderen Teilnehmers kennen. Dieser wurde bei der Registrierung in die Blockchain
geschrieben. Da in der Hashtabelle Schlüssel-Wert-Paare gespeichert werden, wird
der öffentliche Schlüssel als Schlüssel und die IP-Adresse und der Port des Teilnehmers als
Wert gespeichert. Der Teilnehmer, der eine Verbindung herstellen möchte, muss den öffentlichen
Schlüssel des anderen Teilnehmers kennen. Dazu muss er den öffentlichen Schlüssel in der
verteilten Hashtabelle suchen. Dazu berechnet er aus dem öffentlichen Schlüssel einen Hashwert.
Der Hashwert wird in einem Zahlenbereich abgebildet. Der Teilnehmer sucht in der verteilten
Hashtabelle nach dem nächsten Hashwert. Der Teilnehmer, der den Hashwert in der verteilten
Hashtabelle gespeichert hat, ist der Nachbar des Teilnehmers, der die Verbindung herstellen
möchte. Der Teilnehmer, der die Verbindung herstellen möchte, sendet eine Anfrage an den
Nachbarn. Die Anfrage enthält den öffentlichen Schlüssel des Teilnehmers, der die Verbindung
herstellen möchte. Der Nachbar antwortet mit seiner IP-Adresse und seinem Port. Der Teilnehmer,
der die Verbindung herstellen möchte, kann nun eine Verbindung zum Nachbarn aufbauen. Der
Nachbar leitet die Nachrichten an den Teilnehmer weiter, der die Verbindung herstellen möchte.