\subsection{Verbindung und Identität}

Um eine Verbindung zwischen zwei Teilnehmern herzustellen, muss der Client des
Absenders die IP-Adresse und den Port des Empfängers kennen. Diese Informationen
werden benötigt, um eine Verbindung über das Internet herzustellen. In einem Peer
to Peer-Netzwerk ist es jedoch nicht möglich, dass jeder Teilnehmer die IP-Adresse
und den Port jedes anderen Teilnehmers kennt. Deshalb kommen hier die verteilten
Hash-Tabellen zum Einsatz. Jeder Teilnehmer speichert in einer verteilten Hash-Tabelle
seine IP-Adresse und seinen Port. Die verteilte Hash-Tabelle ist ein verteiltes
System, das die IP-Adressen und Ports aller Teilnehmer speichert. Die verteilte
Hash-Tabelle ist in mehrere Buckets unterteilt. Jeder Bucket enthält die IP-Adressen
und Ports der Teilnehmer, die sich in einem bestimmten Bereich befinden. Die
Buckets sind nach der Distanz der IP-Adressen und Ports der Teilnehmer zum eigenen
IP-Adresse und Port sortiert. Die Buckets sind in der Regel so konfiguriert, dass
jeder Bucket maximal 8 Teilnehmer enthalten kann. Wenn ein Teilnehmer eine Verbindung
zu einem anderen Teilnehmer herstellen möchte, sucht er in der verteilten Hash-Tabelle
nach der IP-Adresse und dem Port des Empfängers. Wenn der Empfänger in der verteilten
Hash-Tabelle gefunden wird, kann eine Verbindung hergestellt werden. Wenn der Empfänger
nicht in der verteilten Hash-Tabelle gefunden wird, wird die Nachricht über ein
TCP-Relay an den Empfänger weitergeleitet. Wenn aber auch hier der Empfänger nicht
gefunden wird, wird die Nachricht lokal auf dem Gerät des Absenders gespeichert,
bis der Empfänger online ist. Wenn der Empfänger online ist, wird die Nachricht
zugestellt. Die direkte Verbindung zwischen den Teilnehmern wird bevorzugt, erst wenn
diese nicht möglich ist, wird die Nachricht über ein TCP-Relay weitergeleitet.