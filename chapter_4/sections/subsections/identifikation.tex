\subsection{Identifikation von Teilnehmern}
\label{subsec:identifikation_von_teilnehmern}


\begin{itemize}
    \item Jeder Benutzer im Netzwerk hat eine eindeutige ID.
    \item Adressierung kann über Benutzer-IDs oder Schlüsselpaare erfolgen.
    \item Die ID wird als 32-Byte-Array gespeichert.
\end{itemize}

[Jeder Benutzer im Netzwerk hat eine eindeutige ID. Adressierung kann über 
Benutzer-IDs oder Schlüsselpaare erfolgen.]

\noindent Da es in Peer-to-Peer-Netzwerken keinen zentralen Server gibt, der die
Kommunikation steuert, müssen die Teilnehmer auf andere Weise identifiziert
werden. Hierfür können verschiedene Methoden zur
Identifizierung von Teilnehmern verwendet werden. Für dieses Protokoll
fiel die Wahl auf das Kademlia-Protokoll, das in vielen Peer-to-Peer-Netzwerken
verwendet wird. Das Kademlia-Protokoll verwendet eine ID mit einer Länge von 160
Bit, die durch einen \textcolor{red}{SHA-1-Hash} des öffentlichen Schlüssels des 
Teilnehmers erzeugt wird. Durch die Verwendung einer Hashfunktion wird die ID auf eine
festgelegte Länge begrenzt.
