\subsection{Identifikation von Teilnehmern}
\label{subsec:identifikation_von_teilnehmern}


\noindent Da es in Peer-to-Peer-Netzwerken keinen zentralen Server gibt, der unter anderem zur Auffindung anderer Teilnehmer verwendet werden kann, müssen die Teilnehmer auf andere Weise identifiziert werden. Hierfür können verschiedene Methoden verwendet werden. Für dieses Protokoll fiel die Wahl auf die Integration des Kademlia-Protokolls, das in vielen Peer-to-Peer-Netzwerken Anwendung findet \textcolor{red}{(Beispiele nennen)}. Um nun Teilnehmer identifizieren zu können, verwendet Kademlia eine ID, die jedem Teilnehmer zugeordnet wird. Diese ID muss laut der Spezifikation 160 Bit lang sein \Parencite[S. 2]{Maymounkov_Kademlia}. Weiter ist zu entnehmen, dass angenommen wird, dass die ID beim Netzwerkbeitritt aus einer zufälligen Kennung gewählt wird. Für das hier entwickelte Protokoll wird der Benutzername des Teilnehmers als ID verwendet. Dieser muss eindeutig sein und kann vom Benutzer frei gewählt werden. Um die Anforderung an die Länge zu erfüllen, werden die Zeichen des Benutzernamens auf 20 begrenzt und bei einer Formatierung in UTF-8 ergibt sich somit eine ID mit einer Länge von 160 Bit. Sollte der Benutzername kürzer als 20 Zeichen sein, wird er mit Nullen aufgefüllt.
