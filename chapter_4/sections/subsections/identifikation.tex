\subsection{Identifikation von Teilnehmern}
\label{subsec:identifikation_von_teilnehmern}

Da es in Peer-to-Peer Netzwerken keinen zentralen Server gibt, der unter anderem zur Auffindung und Identifikation anderer Teilnehmer verwendet werden kann, müssen die Teilnehmer auf andere Weise identifiziert werden. Durch die Entscheidung, das Kademlia Protokoll zu implementieren, wird die ID des Teilnehmers als Schlüssel für die Speicherung der Teilnehmerinformationen, wie IP-Adresse und Port, verwendet.
Aus der Spezifikation von Kademlia geht hervor, dass die ID 160 Bit lang sein muss und aus einer zufälligen Kennung bestehen soll \parencite[S. 2]{Maymounkov_Kademlia}. Für das hier entwickelte Protokoll wird der Benutzername des Teilnehmers als ID verwendet. Dieser muss, wie aus den funktionalen Anforderungen zu entnehmen (siehe \ref{subsec:registrierung}), eindeutig sein und kann vom Benutzer bei der Registrierung frei gewählt werden. Um die Anforderung an die Länge zu erfüllen, werden die Zeichen des Benutzernamens auf 20 begrenzt und bei einer Formatierung in UTF-8 ergibt sich somit eine ID mit einer Länge von 160 Bit. Sollte der Benutzername kürzer als 20 Zeichen sein, wird er mit Nullen aufgefüllt, um die geforderte Länge zu erreichen.
Durch die Beschränkung der ID auf 20 Zeichen wird die Anzahl der möglichen Teilnehmer zwar begrenzt, doch daraus ergeben sich $2^{160}$ mögliche IDs, was einer Anzahl von $1.461.501.637.330.902.918.203.684.832.716.283.019.655.932.542.976$ Teilnehmern entspricht. Diese Anzahl ist so groß, dass sie in der Praxis nicht erreicht werden sollte und somit die Beschränkung der ID keine Auswirkungen auf die Funktionalität des Protokolls haben sollte.

