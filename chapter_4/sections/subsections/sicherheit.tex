\subsection{Sicherheit}


Bei der Registrierung wird ein statisches Schlüsselpaar generiert. Der private Schlüssel wird lokal gespeichert und der öffentliche Schlüssel wird in der Blockchain hinterlegt. Bei einem Verbindungsaufbau wird vom Sender ein neues Schlüsselpaar generiert, welches diesmal aus flüchtigen Schlüsseln besteht. Der flüchtige öffentliche Schlüssel wird mit dem statischen privaten Schlüssel signiert und an den Empfänger gesendet. Der Empfänger kann die Signatur mit dem öffentlichen Schlüssel des Senders verifizieren, da dieser in der Blockchain hinterlegt ist. Anschließend wird ein gemeinsamer geheimer Schlüssel aus dem flüchtigen privaten Schlüssel und dem öffentlichen Schlüssel des Empfängers mittels Key-Derivation-Funktion berechnet und die flüchtigen Schlüssel werden verworfen. Der berechnete flüchtige symmetrische Schlüssel wird für die Verschlüsselung der Nachrichten verwendet. \\
Durch die Anwendung von flüchtigen Schlüsseln wird die Sicherheit erhöht, da diese nur für eine kurze Zeit existieren und somit ein Angreifer nur für einen kurzen Zeitraum Zugriff auf diese hat.
Mit Hilfe dieser Methode wird eine Ende-zu-Ende-Verschlüsselung ermöglicht, da nur der Sender und der Empfänger den gemeinsamen geheimen Schlüssel kennen. Selbst wenn die Verbindung über ein Relay läuft, kann dieses die Nachricht nicht entschlüsseln, da es den gemeinsamen geheimen Schlüssel nicht kennt.
