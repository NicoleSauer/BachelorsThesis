\subsection{Routing}


Durch die in Abschnitt \ref{subsec:identifikation_von_teilnehmern} erwähnte 
Verwendung des Kademlia-Protokolls, wird das Routing und die Weiterleitung von
Nachrichten durch die DHTs übernommen.

Wenn ein Knoten einen Wert oder einen anderen Knoten im Netzwerk finden möchte, 
führt er iterative Suchen durch, indem er eine auf XOR basierende Distanzmetrik 
verwendet. Er fragt Knoten in seiner Routing-Tabelle basierend auf ihrer Nähe 
zur Zielknoten-ID ab. Diese Abfragen helfen, die Suche auf das Ziel hin zu verfeinern, 
indem sie sich mit jedem Schritt im Knoten-ID-Raum näher bewegen. Die iterative
Suche wird beendet, wenn der Zielknoten gefunden wurde oder wenn die Suche
keine Knoten mehr findet, die der Zielknoten-ID näher sind als der am nächsten
gelegene Knoten, der bereits abgefragt wurde. In diesem Fall wird die Suche
beendet und der am nächsten gelegene Knoten zurückgegeben.

Im Kademlia-Protokoll sind vier Funktionen definiert, die für die Suche nach
Knoten und Werten verwendet werden. Diese Funktionen sind \texttt{FIND\_NODE},
\texttt{FIND\_VALUE}, \texttt{PING} und \texttt{STORE}. Die Funktionen
\texttt{FIND\_NODE} und \texttt{FIND\_VALUE} werden verwendet, um nach Knoten
oder Werten zu suchen. Die Funktion \texttt{PING} wird verwendet, um die
Erreichbarkeit eines Knotens zu überprüfen. Die Funktion \texttt{STORE} wird
verwendet, um einen Wert in einem Knoten zu speichern.


