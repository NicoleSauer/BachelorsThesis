\subsection{Routing und Peer Discovery}
\label{subsec:routing}

Wie bereits in Abschnitt \ref{subsec:identifikation_von_teilnehmern} 
\nameref{subsec:identifikation_von_teilnehmern} erwähnt,
wird das Kademlia-Protokoll verwendet, um die Knoten im Netzwerk zu identifizieren.
Es wird auch verwendet, um die Nachrichten zwischen den Knoten zu routen.

% Wenn ein Knoten einen Wert oder einen anderen Knoten im Netzwerk finden möchte, 
% führt er iterative Suchen durch, indem er eine auf XOR basierende Distanzmetrik 
% verwendet. Er fragt Knoten in seiner Routing-Tabelle basierend auf ihrer Nähe 
% zur Zielknoten-ID ab. Diese Abfragen helfen, die Suche auf das Ziel hin zu verfeinern, 
% indem sie sich mit jedem Schritt im Knoten-ID-Raum näher bewegen. Die iterative
% Suche wird beendet, wenn der Zielknoten gefunden wurde oder wenn die Suche
% keine Knoten mehr findet, die der Zielknoten-ID näher sind als der am nächsten
% gelegene Knoten, der bereits abgefragt wurde. In diesem Fall wird die Suche
% beendet und der am nächsten gelegene Knoten zurückgegeben.

% Im Kademlia-Protokoll sind vier Funktionen definiert, die für die Suche nach
% Knoten und Werten verwendet werden. Diese Funktionen sind \texttt{FIND\_NODE},
% \texttt{FIND\_VALUE}, \texttt{PING} und \texttt{STORE}. Die Funktionen
% \texttt{FIND\_NODE} und \texttt{FIND\_VALUE} werden verwendet, um nach Knoten
% oder Werten zu suchen. Die Funktion \texttt{PING} wird verwendet, um die
% Erreichbarkeit eines Knotens zu überprüfen. Die Funktion \texttt{STORE} wird
% verwendet, um einen Wert in einem Knoten zu speichern.


Das Kademlia-Protokoll basiert auf einem Distanzmetrik-Konzept, das als \\
"Kademlia-Distanz" bekannt ist. Jeder Knoten im Netzwerk wird durch eine eindeutige ID repräsentiert, 
typischerweise als kryptografischer Hashwert, der sich aus der gehashten IP-Adresse 
des jeweiligen Knotens mittels der SHA-1 Hashfunktion ergibt. Diese IDs sind in einem 
großen binären Baum organisiert, wobei die Position eines Knotens im Baum seine 
"Kademlia-Distanz" zu anderen Knoten definiert. Diese Distanz wird durch die XOR 
(ausschließendes Oder)-Operation ihrer eindeutigen IDs bestimmt. Die XOR-Operation 
ermöglicht eine effiziente Bestimmung der Distanz zwischen den Knoten-IDs, indem 
sie auf deren Binärzahlen angewendet wird. Dieses Ergebnis repräsentiert die 
Distanz zwischen den IDs und bildet die Grundlage für das Routing und die 
Organisation im Kademlia-Netzwerk.

Kademlia verwendet ein Routingverfahren, bei dem jeder Knoten eine Routing-\\
Tabelle 
speichert, die als "K-Buckets" bezeichnet werden. Jedes K-Bucket enthält Verweise 
auf andere Knoten im Netzwerk und ist nach der Kademlia-Distanz organisiert. Ein 
K-Bucket enthält typischerweise eine begrenzte Anzahl von Einträgen und gruppiert 
Knoten mit ähnlichen IDs.

Wenn ein Knoten eine Verbindung zu einem anderen Knoten herstellen muss, verwendet 
er die Routing-Tabelle, um den am nächsten gelegenen Knoten zu finden, der die 
Ziel-ID repräsentiert. Falls dieser Knoten nicht direkt bekannt ist, wird das 
Routing iterative durchgeführt, wobei der Knoten jeweils näher an der Ziel-ID 
liegende Knoten anfragt, bis der Zielknoten gefunden wird. Durch die Verwendung 
dieses strukturierten Ansatzes ermöglicht Kademlia eine effiziente Suche und 
Kommunikation zwischen Knoten in einem P2P-Netzwerk, wobei die Skalierbarkeit 
und Robustheit des Systems erhalten bleiben. Es ist ein Schlüsselelement vieler 
P2P-Anwendungen, einschließlich Filesharing, dezentraler Datenbanken und eben auch 
Instant Messaging-Protokollen, da es die Grundlage für die direkte 
Peer-to-Peer-Kommunikation schafft.
\\

\noindent Ein Beispiel für die XOR-Berechnung zwischen gehashten Knoten-IDs könnte wie folgt 
aussehen:
Knoten A hat die gehashte ID: 0x83a2c8f7,  
Knoten B hat die gehashte ID: 0xe1b6d4a9.

\noindent Die XOR-Operation zwischen diesen IDs ergibt:
\begin{equation}
    \begin{aligned}
        \text{Knoten A:} & \quad \texttt{0x83a2c8f7} \\
        \text{Knoten B:} & \quad \texttt{0xe1b6d4a9} \\
        \text{Ergebnis:} & \quad \texttt{0x61f47c5e}
    \end{aligned}
\end{equation}
% #TODO: Berechnung noch in binär umwandeln, damit die Berechnung besser verständlich ist?


\noindent Diese Distanz repräsentiert die Maßeinheit für die Positionierung und das Routing 
im Kademlia-Netzwerk. Knoten mit einer geringeren Distanz sind näher beieinander
als Knoten mit einer größeren Distanz. Dabei hat die Distanz nichts mit der
geographischen Entfernung zu tun, sondern nur mit der Position im Kademlia-Baum.