\chapter{Architektur des Protokolls}
\label{chap:entwurf_und_architektur}

% #TODO: hashing in Betracht gezogen, aber da er nur dazu dient eine ID zu erzeugen bzw. man einen Nutzer daran identifizieren soll, ist das eigentlich nicht notwendig zu hashen. Es ist kryptografisch nicht notwendig
% #TODO: deshalb den Benutzernamen verwenden, der ja sowieso eindeutig sein muss (aus UX-Gründen), ihn auf 20 Zeichen limitieren (utf-8 -> 160 bits/8 = 20) und gegebenenfalls mit Nullen auffüllen, falls er kürzer ist

% #TODO: Vielleicht noch den Sicherheitsaspekt der Blockchain erklären, dass die IDs nicht gefälscht werden können, da sie in der Blockchain gespeichert sind und somit nicht manipuliert werden können; wenn die Nachrichten mit dem public key, der in der Blockchain liegt, signiert werden, bietet auch das eine Sicherheit, dass die Nachrichten nicht gefälscht werden können

% #TODO: neu schreiben, wenn der komplette Inhalt des Kapitels feststeht 
Die Architektur des Protokolls basiert auf Peer-to-Peer. Das bedeutet, dass alle Teilnehmer gleichberechtigt sind und es keinen zentralen Server gibt, der die Kommunikation steuert. Somit erfolgt die Kommunikation direkt zwischen den Teilnehmern. Alle Teilnehmer sind in einem Netzwerk organisiert, das aus verschiedenen Knoten besteht, wobei jeder Knoten einen Teilnehmer des Netzwerks repräsentiert. Die direkte Verbindung wird bevorzugt, erst wenn diese nicht möglich ist, wird die Nachricht über ein Relay weitergeleitet. Die Blockchain findet ihren Einsatz bei der Identifikation und Authentifizierung der Teilnehmer. Jeder Teilnehmer hat eine eindeutige ID, die aus dem Benutzernamen besteht und bei der Registrierung in der Blockchain gespeichert wird, und somit für alle Teilnehmer einsehbar ist. Somit kann jeder Teilnehmer einen anderen Teilnehmer anhand seiner ID identifizieren. Die Authentifizierung erfolgt ebenfalls über die Blockchain, indem der öffentliche Schlüssel des Teilnehmers in der Blockchain gesucht wird. Wenn der öffentliche Schlüssel gefunden wird und der Benutzername mit der ID übereinstimmt, ist der Teilnehmer authentifiziert. 

\section{Grundlagen des Protokolls}
\label{sec:grundlagen_des_protokolls}


Um die Peer-to-Peer Funktionalität für das Protokoll dieser Arbeit zu gewährleisten, wird ein mehrschichtiges Verfahren verwendet.

Für den effizienten Aufbau einer Direktverbindung zwischen zwei Teilnehmern kamen Chord und Kademlia in die engere Auswahl, welche beide lange Gegenstand intensiver Forschung waren, sowohl in der Industrie als auch in der akademischen Welt \parencite[S. 808]{MedranoChavez_ChordKademliaHighChurnScenarios}. 
Das Chord-Protokoll und das Kademlia-Protokoll sind zwei grundlegend verschiedene Ansätze zur Organisation von Peer-to-Peer-Netzwerken. Beide Protokolle sind strukturiert und bieten eine effiziente Ressourcenlokalisierung, aber sie unterscheiden sich in ihrer Routing-Struktur und der Art und Weise, wie sie die Knoteninformationen verwalten.

\subsubsection{Chord}
Chord basiert auf einer Ringstruktur (siehe Abbildung \ref{chord_ring}), bei der die Knoten in einem Ring angeordnet sind und jeder Knoten für einen bestimmten Schlüsselbereich verantwortlich ist. Die Verbindungen zwischen den Knoten sind durch ihren Platz im Ring definiert, wobei jeder Knoten eine Verbindung zu seinem nächsten Nachbarn im Uhrzeigersinn hat. Ein Knoten besitzt zwei Informationsmengen: eine \textit{Successor-Liste} und eine \textit{Finger-Tabelle}. Die Successor-Liste enthält die Knoten, die direkt nach dem Knoten im Uhrzeigersinn im Ring kommen. Die Anzahl der dort enthaltenen Knoten hängt davon ab, wie viele Knoten im Netzwerk insgesamt vorhanden sind. Die Finger-Tabelle enthält die Knoten, die für die Schlüsselbereiche verantwortlich sind, die durch eine Berechnung auf der ID des Knotens basieren \Parencite[S. 810-811]{MedranoChavez_ChordKademliaHighChurnScenarios}.

\begin{center}
    \captionsetup{type=figure}
    \includegraphics[width=1\linewidth]{images/chord_ring_altered.png}
    \captionof{figure}{Visualisierung der Ringstruktur von Chord, in Anlehnung an \cite[S. 811]{MedranoChavez_ChordKademliaHighChurnScenarios}}
    \label{chord_ring}
\end{center}

\noindent Wenn das Chord-Netzwerk eine Suchanfrage erhält, gibt es zwei Strategien, um die Anfrage zu bearbeiten. In der ersten Strategie wird die Anfrage sequentiell von Knoten zu Knoten weitergeleitet, bis der Knoten gefunden wird, der für den Schlüsselbereich verantwortlich ist, in dem sich der gesuchte Schlüssel befindet. Für diese Suchstrategie ergibt sich daher eine Komplexität von $\mathcal{O}(n)$, wobei $n$ die Anzahl der Knoten im Netzwerk ist. Die zweite Strategie verwendet die Finger-Tabelle, um die Anzahl der Knoten zu reduzieren, die die Anfrage weiterleiten. Diese Strategie hat eine Komplexität von $\mathcal{O}(\log n)$, wobei $n$ die Anzahl der Knoten im Netzwerk ist \parencite[S. 810-811]{MedranoChavez_ChordKademliaHighChurnScenarios}. 

In Abbildung \ref{chord_ring} ist zu erkennen, dass Knoten $59$ eine Suchanfrage für den Knoten mit der ID $20$ beginnt. Unter Verwendung der Finger-Tabelle von Knoten $59$ wird die Anfrage an den Knoten gesendet, der am nächsten an Knoten $20$ liegt. In diesem Fall ist dies Knoten $16$. Knoten $16$ wiederum leitet die Anfrage an den Knoten weiter, der ebenfalls am nächsten an Knoten $20$ liegt, was Knoten $23$ ist. Knoten $23$ ist für den Schlüsselbereich verantwortlich, in dem sich der gesuchte Schlüssel befindet, und sendet daher die Antwort an Knoten $59$ zurück. Durch die Verwendung dieser Strategie wurde Knoten $20$ in nur zwei Schritten gefunden, anstatt in fünft Schritten, wenn die Anfrage sequentiell weitergeleitet worden wäre.    

\subsubsection{Kademlia}
Im Gegensatz dazu verwendet Kademlia eine K-Bucket-Struktur, die in Abbildung \ref{kademlia_tree} zu sehen ist, um eine effiziente Verwaltung von Knoteninformationen zu ermöglichen. Die K-Buckets enthalten eine Liste von Knoten für verschiedene Schlüsselbereiche basierend auf ihrer Nähe, die durch XOR-Distanzen der IDs berechnet wird. Die Verbindungen zwischen den Knoten sind asymmetrisch, und jeder Knoten speichert Informationen über andere Knoten in seinen K-Buckets. Bei der Suche nach einem bestimmten Schlüssel erfolgt das Routing durch die XOR-Entfernung, wodurch die nächsten Knoten für diesen Schlüssel gefunden werden.

\begin{center}
    \captionsetup{type=figure}
    \includegraphics[width=1\linewidth]{images/kademlia_tree_altered.png}
    \captionof{figure}{Visualisierung der Baumstruktur von Kademlia, in Anlehnung an \cite[S. 812]{MedranoChavez_ChordKademliaHighChurnScenarios}}
    \label{kademlia_tree}
\end{center}

\noindent In Abbildung \ref{kademlia_tree} ist zu sehen, wie der Knoten $1011$ ein Suche nach Knoten $0010$ startet. Der Knoten $1011$ sucht in seinen vier K-Buckets nach dem nächsten Knoten, der am Schlüssel $0010$ liegt. Im ersten Bucket sind Knoten mit dem Präfix $0xxx$ enthalten. In Bucket zwei sind Knoten mit dem Präfix $11xx$, in Bucket drei Knoten mit dem Präfix $100x$ und in Bucket vier Knoten mit dem Präfix $101x$. Da der Schlüssel $0010$ mit dem Präfix $00$ beginnt, wird der nächste Knoten für diesen Schlüssel im ersten Bucket gesucht. Knoten $0011$ wird als nächster Knoten für den Schlüssel $0010$ gefunden. Knoten $1011$ sendet die Anfrage also an Knoten $0011$, der wiederum den nächsten Knoten für den Schlüssel $0010$ sucht. Da dieser Knoten den Schlüssel $0010$ in seinem K-Bucket besitzt, sendet er die Antwort auf die Suchanfrage an Knoten $1011$ zurück. Es werden zwei Weiterleitungen durchgeführt, um den Zielknoten zu finden, woraus sich eine Komplexität von $\mathcal{O}(\log n)$ ergibt, wobei $n$ die Anzahl der Knoten im Netzwerk ist \parencite[S. 812]{MedranoChavez_ChordKademliaHighChurnScenarios}.    


Da bei einem Instant-Messaging-Protokoll häufig Teilnehmer das Netzwerk verlassen und neue Teilnehmer dem Netzwerk beitreten, ist es wichtig, dass das Protokoll mit hoher Fluktuation umgehen kann. Diese Fluktuation von Nodes wird als Churn (engl. Abwanderung) bezeichnet. In einer Studie von Medrano-Chávez et al. \parencite{MedranoChavez_ChordKademliaHighChurnScenarios}, welche im hybriden Journal \textit{Peer-to-Peer Networking and Applications} veröffentlicht wurde, wurde die Leistung von Chord und Kademlia in Bezug auf Netzwerkfluktuation untersucht. Die Ergebnisse zeigen, dass Kademlia bei hoher Fluktuation besser abschneidet als Chord. Aus diesem Grund wird Kademlia in diesem Protokoll als Grundlage für das Auffinden von Teilnehmern und das Routing verwendet.

% #TODO: Warum Kademlia und was die Sicherheitsaspekte angeht, erst in Architektur aufgreifen.
    % Signaling Server könnte auch mit Anfragen geflutet werden -> der Sever könnte bei jedem checken, ob es sich um einen validen Teilnehmer handelt, was aber sehr aufwändig wäre
    % wenn Kademlia mit Anfragen geflutet wird, könnte es zu einem Denial of Service kommen, da die Knoten nicht mehr erreichbar sind -> habe aber ja mit dem Signaling Server noch eine weitere Instanz, die die Anfragen entgegen nimmt und weiterleitet


% #TODO: Overlay Netzwerk erklären? Kademlia ist ein Overlay Netzwerk
\begin{itemize}
    \item Kademlia vs. Angriffe
    \item Angriffe gegen ICE gucken und erklären
\end{itemize}

\noindent Sollte der Aufbau einer Direktverbindung mittels Kademlia nicht möglich sein, wird das Interactive Connectivity Establishment (ICE) Protokoll verwendet, um eine Verbindung zwischen zwei Teilnehmern herzustellen. ICE ist ein Framework, das mehrere Techniken kombiniert, um eine Verbindung zwischen zwei Endpunkten herzustellen, die sich hinter NATs befinden. Genaue Details zur Implementierung von ICE folgt in Kapitel \ref{subsec:verbindungsmanagement} \textit{\nameref{subsec:verbindungsmanagement}}.

\noindent Um die Problematik mit NATs zu lösen, wird das ICE-Protokoll verwendet. ICE ist ein Framework, das mehrere Techniken kombiniert, um eine Verbindung zwischen zwei Endpunkten herzustellen, die sich hinter NATs befinden. Genaue Details zur Implementierung von ICE folgt in Kapitel \ref{subsec:verbindungsmanagement} \textit{\nameref{subsec:verbindungsmanagement}}.




\section{Identifikation von Teilnehmern}
\label{subsec:identifikation_von_teilnehmern}

Da es in Peer-to-Peer-Netzwerken keinen zentralen Server gibt, der unter anderem zur Auffindung und Identifikation anderer Teilnehmer verwendet werden kann, müssen die Teilnehmer auf andere Weise identifiziert werden. Durch die Entscheidung, das Kademlia Protokoll zu implementieren, wird die ID des Teilnehmers als Schlüssel für die Speicherung der Teilnehmerinformationen, wie IP-Adresse und Port, verwendet.

Aus der Spezifikation von Kademlia geht hervor, dass die ID einer Node, welche auch als \textit{Kademlia-ID} bezeichnet wird, 160 Bit lang sein muss und aus einer zufälligen Kennung bestehen soll \parencite[S. 2]{Maymounkov_Kademlia}. Für das hier entwickelte Protokoll wird der Benutzername des Teilnehmers als ID verwendet. Da Kademlia eine ID mit einer Länge von 160 Bit erwartet, muss der Benutzername auf diese Länge gebracht werden. Dafür wird eine Zeichenbegrenzung von 20 Zeichen festgelegt, was bei einer UTF-8 Codierung einer Bitlänge von 160 Bit entspricht \parencite{rfc3629_utf-8}. Falls der Benutzername kürzer als 20 Zeichen ist, wird er mit Nullen aufgefüllt.
Der Benutzername muss, wie aus den funktionalen Anforderungen (siehe \ref{subsec:registrierung}) zu entnehmen ist, eindeutig sein und kann vom Benutzer bei der Registrierung frei gewählt werden. Außerdem wird bei der Registrierung ein privater und ein öffentlicher Schlüssel erzeugt. Der öffentliche Schlüssel wird zusammen mit dem Benutzernamen in der Blockchain gespeichert. Der private Schlüssel wird lokal auf dem Gerät des Teilnehmers gespeichert. Durch die Beschränkung der ID auf 20 Zeichen wird die Anzahl der Teilnehmer zwar begrenzt, jedoch ist die Anzahl der möglichen Teilnehmer immer noch sehr groß. Es können in der Theorie $2^{160}$ Teilnehmer am Netzwerk teilnehmen, wobei nicht jedes Zeichen in UTF-8 sinnvoll für die Verwendung in einem Benutzername ist. Somit ist die Anzahl der möglichen Teilnehmer in der Praxis geringer.

Diese Anzahl ist so groß, dass sie in der Praxis nicht erreicht werden sollte und somit die Beschränkung der ID keine Auswirkungen auf die Funktionalität des Protokolls haben sollte.
In der Node werden dann Schlüssel-Wert-Paare gespeichert, wobei der Schlüssel die ID des Teilnehmers ist und der Wert die IP-Adresse und der Port. Es wird festgelegt, dass immer Port $49152$ verwendet wird, da sich dieser Port im Bereich der dynamischen Ports befindet und somit nicht für andere Anwendungen reserviert ist \parencite[S. 20]{rfc6335_IANA_Ports}. 

Wenn ein Teilnehmer eine Nachricht an einen anderen Teilnehmer senden möchte, muss er dessen IP-Adresse kennen. Um an diese Information zu gelangen, muss zuerst die ID des Teilnehmers in der Blockchain gesucht werden und anschließend eine Peer-Discovery durchgeführt werden, um die IP-Adresse des Teilnehmers zu erhalten. Eine Peer-Discovery läuft wie folgt ab:

\begin{enumerate}
    \item Alice hasht den Benutzernamen von Bob mit der SHA-1 Hashfunktion und erhält damit die Kademlia-ID von Bob.
    \item Alice sendet eine \textit{FIND\_NODE}-Nachricht an den Knoten in ihrer Routing-Tabelle, der der Kademlia-ID von Bob am nächsten ist. 
    \item Der Knoten sucht in seiner Routing-Tabelle nach der Kademlia-ID von Bob. Falls er den Knoten findet, antwortet er mit den gefundenen Informationen in Form eines Tripletts aus Kademlia-ID, IP-Adresse und Port. Falls der Knoten den Knoten nicht findet, antwortet er mit mehreren Triplets, die dem Knoten mit der Kademlia-ID von Bob am nächsten sind.
    \item Alice erhält die Antwort und sucht in der Antwort nach der Kademlia-ID von Bob. Falls sie gefunden wird, speichert Alice die IP-Adresse und den Port von Bob in ihrer Routing-Tabelle ab. Falls sie nicht gefunden wird, wiederholt Alice Schritt 2 solange, bis sie keine Antworten mehr erhält, die IDs beinhalten, die der Kademlia-ID von Bob näher sind als die ID des Knotens, der die \textit{FIND\_NODE}-Nachricht erhalten hat.
\end{enumerate}


\noindent Um das Netzwerk immer aktuell zu halten, wird in regelmäßigen Abständen ein \textit{PING} an alle Teilnehmer gesendet, die in den Routing-Tabellen der Nodes gespeichert sind. Falls ein Teilnehmer nicht antwortet, wird er aus der Routing-Tabelle entfernt.


\section{Nachrichtenformat}
\label{sec:nachrichtenformat}

Jedes Instant-Messaging-Protokoll benötigt die Definition eines Nachrichtenformats, das die Struktur der Nachrichten festlegt, die zwischen den Teilnehmern ausgetauscht werden. Im Folgenden wird das Nachrichtenformat für das Instant-Messaging-Protokoll beschrieben.

Nachdem eine Verbindung zwischen zwei Teilnehmern hergestellt werden konnte, kann die Nachricht vom Sender an den Empfänger gesendet werden. Die Nachricht wird in binär serialisierter Form übertragen und enthält die folgenden Informationen:

\begin{itemize}
    \item Benutzername des Senders
    \item Benutzername des Empfängers
    \item Timestamp
    \item Signatur
    \item Nachrichtenlänge
    \item Nachrichteninhalt
\end{itemize}

\noindent Bei der Erstellung des Nachrichtenformats wurde sich am Vorschlag von Day et al. der Etablierung eines Standards orientiert, der in RFC $2779$ mit dem Titel \textit{Instant Messaging/Presence Protocol Requirements} definiert ist \Parencite[S. 9]{rfc2779_IMPP}. 

Die Benutzernamen erlauben die Zuordnung der Nachricht. Der Timestamp enthält die aktuelle Zeit, zu der die Nachricht gesendet wird und wird für die richtige Sortierung der Nachrichten verwendet. Die Signatur dient der Authentifizierung des Senders (siehe \ref{subsec:vertrauliche_kommunikation} \textit{\nameref{subsec:vertrauliche_kommunikation}}). Aus Sicherheitsgründen wird jede Nachricht signiert. Die Nachrichtenlänge gibt die Länge des Nachrichteninhalts an und der Nachrichteninhalt enthält die eigentliche Nachricht, die vom Sender an den Empfänger gesendet werden soll.


\section{Verbindungsmanagement}
\label{subsec:routing}

\begin{itemize}
    \item Verbindungsaufbau
    \item Nachrichtenübertragung
    \item Verbindungsabbau
\end{itemize}

% Nachdem die IP-Adresse und Port des Empfängers ermittelt wurden, kann die Nachricht auf dem direkten Weg an den Empfänger gesendet werden
% Für die Kommunikation wird das ICE Protokoll verwendet
% #TODO: ICE Protokoll verwenden, um die Verbindung aufzubauen, wenn die direkte Verbindung nicht möglich ist
% #TODO: Wie verbinde ich Kademlia und ICE? -> Kademlia für die Peer Discovery und ICE für den Verbindungsaufbau?
% #TODO: Zustellbestätigung?
% #TODO: Was passiert, wenn der Empfänger nicht erreichbar ist? -> Nachricht wird nicht zugestellt, da keine Speicherung der Nachrichten vorgesehen ist
% #TODO: Erwähnen, dass zuerst nur TURN verwendet werden sollte, da es einfacher zu implementieren ist, sich dann aber für ICE entschieden wurde, da es insgesamt eine bessere Lösung ist

\noindent Das Verbindungsmanagement ist ein wichtiger Bestandteil des Protokolldesigns, da es die Grundlage für die Kommunikation zwischen den Teilnehmern bildet. Es ist dafür verantwortlich, dass die Nachrichtenübertragung zwischen den Teilnehmern funktioniert. Dazu gehört der Verbindungsaufbau, die Nachrichtenübertragung und der Verbindungsabbau.

\subsection{Verbindungsaufbau}

Nachdem das Kademlia-Netzwerk erfolgreich die IP-Adresse und den Port des Zielgeräts ermittelt hat, beginnt der Prozess der Vorbereitung für die Verbindungseinrichtung zwischen den beiden Teilnehmern. Dies beinhaltet das Sammeln verschiedener potenzieller Verbindungsadressen (auch \textit{Kandidatenadressen} genannt \parencite[S. 8]{rfc8445_ICE}), die für eine zuverlässige Kommunikation benötigt werden, insbesondere wenn einer oder beide Teilnehmer sich hinter Netzwerkadressübersetzungen (kurz: NATs) oder Firewalls befinden. Zuallererst werden die lokalen oder \textit{Host}-Adressen der beteiligten Geräte berücksichtigt. Diese Adressen repräsentieren die standardmäßigen IP-Adressen und Ports der Geräte innerhalb ihres lokalen Netzwerks. Sie dienen als potenzielle direkte Verbindungswege zwischen den Geräten, falls sie sich im gleichen Netzwerk oder Subnetz befinden. Dank der Peer-Discovery wurde die IP-Adresse und der Port des Zielgeräts bereits ermittelt und kann somit 
als Host-Adresse verwendet werden.

Zusätzlich zu den lokalen Adressen werden reflexive Adressen über STUN (Session Traversal Utilities for NAT) ermittelt. STUN ermöglicht es einem Gerät, seine eigene externe IP-Adresse und den entsprechenden Port zu identifizieren, wie sie von einem NAT-Gerät reflektiert werden \parencite[S. 4]{rfc8489_STUN}. Diese reflexiven Adressen stellen die externe Sichtbarkeit des Geräts aus der Perspektive des NAT-Geräts dar und helfen bei der Umgehung von NAT-Beschränkungen für den direkten Verbindungsaufbau. Sollte es jedoch aufgrund von restriktiven NAT-Konfigurationen nicht möglich sein, eine direkte Verbindung aufzubauen, kommen Relay-Adressen durch TURN (Traversal Using Relays around NAT) ins Spiel.

Durch TURN werden Relay-Server genutzt, um den Datenverkehr zwischen den Geräten zu vermitteln \parencite[S. 10 f.]{rfc8656_TURN}. Diese Weiterleitungsadressen dienen als alternative Verbindungsmethode, indem sie den Datenverkehr über den Relay-Server leiten und so die Hindernisse von restriktiven NATs oder Firewalls umgehen. Die Verwendung eines Relay-Servers ist jedoch nicht wünschenswert, da er zusätzliche Kosten und Latenz verursacht.

Die kombinierte Nutzung dieser verschiedenen Arten von potenziellen Verbindungsadressen – von lokalen, reflexiven bis hin zu Relay-Adressen – ermöglicht eine Vielzahl von Optionen für die Verbindungseinrichtung zwischen den Geräten. Diese Vielfalt an Adressen gewährleistet, dass selbst in komplexen Netzwerkszenarien wie NATs oder Firewalls verschiedene Wege für eine zuverlässige Kommunikation vorhanden sind. 
Die nächste Phase nach der Sammlung dieser Adressen umfasst die Durchführung von Konnektivitätsprüfungen und die Auswahl der am besten geeigneten Verbindungswege für eine erfolgreiche Kommunikation zwischen den beteiligten Geräten.


\textbf{\textcolor{red}{TODO: UML-Diagramm für Verbindungsaufbau erstellen}}


\subsection{Konnektivitätsprüfung}

Nach der Sammlung potenzieller Verbindungsadressen werden Konnektivitätsprüfungen durchgeführt. Diese Prüfungen dienen dazu, die Eignung und Zuverlässigkeit der gesammelten Verbindungswege zwischen den Geräten zu bewerten, um die bestmögliche Verbindung für eine erfolgreiche Kommunikation zu identifizieren. Die Reihenfolge der Konnektivitätsprüfungen wird durch einen Prioritätsalgorithmus bestimmt, der die Verbindungsadressen nach ihrer Priorität ordnet. Aus den Guidelines von ICE geht hervor, dass die Priorität einer Verbindungsadresse durch die folgende Formel berechnet wird \parencite[S. 22]{rfc8445_ICE}:

\begin{equation}
    \label{eq:ice_priority}
    \text{priority} = \text{2}^{24} \cdot \text{(type preference)} + \text{2}^{8} \cdot \text{(local preference)} + \text{2}^{0} \cdot \text{(256 - component ID)}
\end{equation}

\noindent Die Werte für die Typenpräferenz, die in der Dokumentation von ICE für die Berechnung der Priorität empfohlen werden, sind die folgenden: $126$ für Host-Adressen, $100$ für Peer-reflexive Adressen und $0$ für Relay-Adressen. Der Wert $0$ bedeutet nicht, dass Relay-Adressen nicht verwendet werden sollten, sondern dass sie die niedrigste Priorität haben. 

Für die lokalen Präferenzen wird ein Wert von $65535$ empfohlen, um die Verwendung von lokalen Adressen zu priorisieren. Und der dritte und letzte Wert der Formel ist die Komponenten-ID, die dazu dient, verschiedene Datenströme oder Komponenten innerhalb eines einzelnen Kandidaten zu unterscheiden. Als Beispiel dient hier WebRTC, das mehrere Komponenten für verschiedene Arten von Datenströmen, wie beispielsweise Audio, Video oder Text, verwendet. Da in dem Protokoll dieser Arbeit nur ein Datenstrom verwendet wird - und zwar Text - wird der Komponenten-ID der Wert $1$ zugewiesen.

Die Priorität einer Verbindungsadresse wird durch die Summe der drei Werte berechnet. Die Priorität wird dann verwendet, um die Reihenfolge der Konnektivitätsprüfungen zu bestimmen. Die Verbindungsadressen mit der höchsten Priorität werden zuerst getestet, da sie die besten Verbindungswege darstellen.
\\
\\
Während der Konnektivitätsprüfung werden die gesammelten und priorisierten Kandidatenadressen getestet, um deren Erreichbarkeit, Stabilität und Latenzzeit zu bewerten. Dieser Prozess beinhaltet den Versuch, Verbindungen aufzubauen und Datenverkehr über verschiedene potenzielle Wege zu senden und zu empfangen.
Durch das Senden von Probe-Paketen über jede potenzielle Verbindungsadresse wird geprüft, ob die Kommunikation erfolgreich erfolgen kann. Dabei werden die drei verschiedenen Arten von Adressen (lokale, reflexive und Relay-Adressen) verwendet, um verschiedene Möglichkeiten zu testen, wie die Geräte miteinander kommunizieren können. Die Probe-Pakete werden über UDP (User Datagram Protocol) gesendet, da es ein verbindungsloses Protokoll ist und somit keine Verbindung aufgebaut werden muss \parencite[S. 1]{rfc768_UDP}. Dies ermöglicht es, die Erreichbarkeit der Verbindungsadressen zu testen, ohne eine Verbindung aufzubauen. Die Probe-Pakete werden an die Verbindungsadressen gesendet und die Antwort wird überwacht. Wenn eine Antwort empfangen wird, wird die Verbindungsadresse als erreichbar angesehen. Wenn keine Antwort empfangen wird, wird die Verbindungsadresse als nicht erreichbar angesehen. Die Probe-Pakete werden in regelmäßigen Abständen gesendet, um die Stabilität der Verbindungsadressen zu testen. Wenn die Probe-Pakete über einen längeren Zeitraum nicht beantwortet werden, wird die Verbindungsadresse als nicht stabil angesehen. Die Latenzzeit wird durch die Zeit gemessen, die benötigt wird, um eine Antwort auf ein Probe-Paket zu erhalten. Die Latenzzeit wird verwendet, um die Verbindungsadressen nach ihrer Latenzzeit zu priorisieren. Je niedriger die Latenzzeit, desto höher die Priorität der Verbindungsadresse. Die Latenzzeit wird auch verwendet, um die Stabilität der Verbindungsadressen zu bewerten. Wenn die Latenzzeit über einen längeren Zeitraum zu hoch ist, wird die Verbindungsadresse als nicht stabil angesehen. 

Ein Vorteil bei der Verwendung von ICE ist, dass es die Möglichkeit bietet, die Konnektivitätsprüfungen mit der Peer-Discovery zu kombinieren. Da die Peer-Discovery bereits die IP-Adresse und den Port des Zielgeräts ermittelt hat, kann dieser Prozess genutzt werden, um die Konnektivitätsprüfungen durchzuführen. Dadurch wird die Anzahl der Nachrichten, die zwischen den Geräten ausgetauscht werden müssen, reduziert, was zu einer effizienteren Kommunikation führt. Ein weiterer Vorteil ist die Flexibilität von ICE, die es ermöglicht, sich an sich ändernde Netzwerkbedingungen anzupassen. Falls während der Kommunikation Netzwerkparameter sich ändern - beispielsweise durch einen Wechsel zwischen Wi-Fi und Mobilfunknetzwerken - kann ICE dynamisch neue Kandidatenadressen identifizieren und die Verbindungen anpassen, ohne die laufende Kommunikation zu unterbrechen.

\textbf{\textcolor{red}{TODO: Vielleicht hier auch ein Diagramm?}}

\subsection{Nachrichtenübertragung}

Nachdem die Konnektivitätsprüfungen abgeschlossen sind, wird die am besten geeignete Verbindungsadresse für die Kommunikation ausgewählt. Die Auswahl der Verbindungsadresse erfolgt durch den Prioritätsalgorithmus, der die Verbindungsadressen nach ihrer Priorität ordnet. Die Verbindungsadresse mit der höchsten Priorität wird als die am besten geeignete Verbindungsadresse ausgewählt. Wenn die am besten geeignete Verbindungsadresse eine lokale Adresse ist, wird eine direkte Verbindung zwischen den Geräten aufgebaut. Wenn die am besten geeignete Verbindungsadresse eine reflexive Adresse ist, wird eine direkte Verbindung zwischen den Geräten aufgebaut, indem die reflexive Adresse als Zieladresse verwendet wird. Wenn wiederum die am besten geeignete Verbindungsadresse eine Relay-Adresse ist, wird eine Verbindung über den Relay-Server aufgebaut, indem die Relay-Adresse als Zieladresse verwendet wird. Der Nachrichtenaustausch erfolgt über die ausgewählte Verbindungsadresse. 

Wobei hier mit \textit{Verbindung aufbauen} gemeint ist, dass die Nachrichten über die Verbindungsadresse gesendet und empfangen werden können. Eine wirkliche Verbindung wird nicht aufgebaut, da die Nachrichten mittels UDP gesendet werden, und UDP ist verbindungslos. Das bedeutet, dass keine Verbindung aufgebaut werden muss, um die Nachrichten zu senden und zu empfangen. Die Nachrichten werden über die Verbindungsadresse gesendet und empfangen. Ein Nachteil bei der Verwendung von UDP ist, dass die Nachrichten nicht zuverlässig zugestellt werden. Was bedeutet, dass Nachrichten verloren gehen können, ohne dass der Absender oder Empfänger davon erfährt. Das Transportprotokoll TCP (Transmission Control Protocol) hingegen ist zuverlässig, da es eine Verbindung aufbaut und sicherstellt, dass die Nachrichten erfolgreich zugestellt werden \parencite[S. 36]{rfc9293_TCP}. Im Falle von Paketverlusten ist auf der gleichen Seite definiert, dass TCP in der Lage ist, die verlorenen Pakete zu erkennen und erneut zu senden.

Doch TCP hat auch Nachteile: ein Nachteil ist, dass es eine Verbindung aufbauen muss, bevor die Nachrichten gesendet werden können. Dies kann zu einer höheren Latenz führen. Ein weiterer Nachteil ist, dass die Verbindung aufrechterhalten werden muss, um die Nachrichten zu senden und zu empfangen. Dies führt zu einem höheren Ressourcenverbrauch, da die Verbindung aufrechterhalten werden muss, auch wenn keine Nachrichten gesendet werden. Auch bei NATs hat TCP Nachteile. NATs schließen die Verbindungen nach einer gewissen Zeit, wenn keine Daten übertragen werden. Dies kann dazu führen, dass die Verbindung geschlossen werden, bevor die Nachrichten gesendet werden können.Bei Instant Messaging zählt die Geschwindigkeit, mit der die Nachrichten gesendet und empfangen werden, mehr als die Zuverlässigkeit der Nachrichten.
Daher ist es besser, die Nachrichten schnell zu senden und zu empfangen, auch wenn dies bedeutet, dass die Nachrichten nicht zuverlässig zugestellt werden. Und auch in Verbindung mit NATs stellt sich UDP als die bessere Wahl heraus. Aus diesen Gründen wurde UDP als Transportprotokoll für die Nachrichtenübertragung gewählt.


\subsection{Verbindungsabbau}

Da es wie bereits erwähnt keine Verbindung gibt, die aufgebaut werden muss, um die Nachrichten zu senden und zu empfangen, gibt es auch keinen Verbindungsabbau. Die Nachrichtenübertragung kann jederzeit gestoppt werden, indem einfach keine Nachrichten mehr gesendet werden. 


%\subsection{Status- und Präsenzinformationen}

Definition, wie Benutzer ihren Status aktualisieren (Online, Abwesend, Offline).
Überlegungen zur Präsenzinformation für effiziente Nachrichtenzustellung.

Sobald ein Nutzer sich mit dem Netzwerk verbunden hat, wird er als online angezeigt.
Sobald er sich abmeldet, wird er als offline angezeigt.

%\subsection{Verbindung und Identität}

Um eine Verbindung zwischen zwei Teilnehmern herzustellen, muss ein Teilnehmer den öffentlichen
Schlüssel des anderen Teilnehmers kennen. Dieser wurde bei der Registrierung in die Blockchain
geschrieben. Da in der Hashtabelle Schlüssel-Wert-Paare gespeichert werden, wird
der öffentliche Schlüssel als Schlüssel und die IP-Adresse und der Port des Teilnehmers als
Wert gespeichert. Der Teilnehmer, der eine Verbindung herstellen möchte, muss den öffentlichen
Schlüssel des anderen Teilnehmers kennen. Dazu muss er den öffentlichen Schlüssel in der
verteilten Hashtabelle suchen. Dazu berechnet er aus dem öffentlichen Schlüssel einen Hashwert.
Der Hashwert wird in einem Zahlenbereich abgebildet. Der Teilnehmer sucht in der verteilten
Hashtabelle nach dem nächsten Hashwert. Der Teilnehmer, der den Hashwert in der verteilten
Hashtabelle gespeichert hat, ist der Nachbar des Teilnehmers, der die Verbindung herstellen
möchte. Der Teilnehmer, der die Verbindung herstellen möchte, sendet eine Anfrage an den
Nachbarn. Die Anfrage enthält den öffentlichen Schlüssel des Teilnehmers, der die Verbindung
herstellen möchte. Der Nachbar antwortet mit seiner IP-Adresse und seinem Port. Der Teilnehmer,
der die Verbindung herstellen möchte, kann nun eine Verbindung zum Nachbarn aufbauen. Der
Nachbar leitet die Nachrichten an den Teilnehmer weiter, der die Verbindung herstellen möchte.

%\subsection{Nachrichten und Daten}

Benutzer können ausschließlich Textnachrichten miteinander austauschen. 
Diese Nachrichten können an einzelne Peers gesendet werden.
Die gesamte Kommunikation ist mit starken Verschlüsselungsalgorithmen 
verschlüsselt, um Privatsphäre und Sicherheit zu gewährleisten.

%\subsection{Nachrichtenverlauf}

Benutzer haben die Möglichkeit, ihren Nachrichtenverlauf lokal zu speichern.
Der Nachrichtenverlauf ist ebenfalls verschlüsselt, um die Sicherheit zu gewährleisten.

\section{Sicherheit}
\label{subsec:sicherheit}

% #TODO: Checken, ob das nach der Umstrukturierung noch so passt
Um die Kommunikation zwischen den Teilnehmern zu schützen, finden sowohl asymmetrische als auch symmetrische Verschlüsselung Anwendung. Die asymmetrische Verschlüsselung wird für die Authentifizierung und die symmetrische Verschlüsselung für die Ende-zu-Ende-Verschlüsselung der Nachrichten verwendet. 

\subsection{Vertrauliche Kommunikation}
\label{subsec:vertrauliche_kommunikation}

Die Kommunikation zwischen den Teilnehmern soll vertraulich sein. Das bedeutet, dass die Nachrichten nur von den Teilnehmern gelesen werden können. Um dies zu gewährleisten, wird die Ende-zu-Ende-Verschlüsselung verwendet. Bei der Ende-zu-Ende-Verschlüsselung wird die Nachricht vom Sender verschlüsselt und erst beim Empfänger wieder entschlüsselt. Dadurch kann ein Angreifer, der die Nachricht abfängt, diese nicht lesen, da er den Schlüssel nicht kennt. Im Detail 

%\section{Fehlerbehandlung}

Fehlerbehandlung und -vermeidung sind wichtige Aspekte eines Instant Messaging-Protokolls. Fehler können zu einer Unterbrechung der Verbindung zwischen den Teilnehmern führen. Um dies zu verhindern, werden verschiedene Mechanismen implementiert, die die Verbindung zwischen den Teilnehmern aufrechterhalten.

Netzwerkfehler können zu einer Unterbrechung der Verbindung zwischen den Teilnehmern führen. Um dies zu verhindern, wird eine Pufferung der Nachrichten implementiert. Wenn ein Teilnehmer eine Nachricht sendet, wird diese in einem Puffer gespeichert, bis der Empfänger die Nachricht empfangen hat. Wenn der Empfänger die Nachricht nicht empfangen kann, wird sie im Puffer gespeichert, bis der Empfänger wieder online ist. Wenn der Empfänger wieder online ist, wird die Nachricht erneut gesendet. Wenn der Empfänger die Nachricht empfangen hat, wird sie aus dem Puffer gelöscht. Wenn der Empfänger die Nachricht nicht empfangen kann, wird sie nach einer bestimmten Zeit aus dem Puffer gelöscht. Die Pufferung der Nachrichten ermöglicht es, die Verbindung zwischen den Teilnehmern aufrechtzuerhalten, auch wenn einer der Teilnehmer offline ist. Dies ist ein wichtiger Aspekt für ein Instant Messaging-Protokoll, da die Teilnehmer nicht immer online sind. 
\\
\\
Firewall und NAT können ebenfalls zu Verbindungsproblemen führen. Um dies zu verhindern, wird ein TCP-Relay verwendet, um die Nachrichten zwischen den Teilnehmern weiterzuleiten. Dieser Mechanismus wird verwendet, wenn die direkte Verbindung zwischen den Teilnehmern nicht möglich ist, z. B. wenn sich die Teilnehmer hinter einer Firewall befinden. Die Verwendung eines TCP-Relays ist jedoch nicht wünschenswert, da es die Skalierbarkeit des Systems beeinträchtigt und die Vertraulichkeit der Nachrichten gefährdet. Daher sollte die direkte Verbindung zwischen den Teilnehmern bevorzugt werden.
\\
\\
Peer-spezifische Fehler:
\\
\\
In einem Peer-to-Peer-Netzwerk kann es häufig vorkommen, dass der gewünschte Teilnehmer nicht online und somit nicht Teil des Netzwerks ist. In diesem Fall wird eine Fehlermeldung an den Absender zurückgegeben, dass der gewünschte Teilnehmer nicht gefunden werden konnte. Wie diese Fehlermeldung implementiert wird, ist dem Entwickler überlassen. Ein weiterer Fehler könnte im Zusammenhang mit Authentifizierung entstehen. Wenn ein Teilnehmer eine Nachricht an einen anderen Teilnehmer senden möchte, muss er sich zuerst authentifizieren. Die Authentifizierung erfolgt über die Blockchain, indem der öffentliche Schlüssel des Teilnehmers mit der ID des Teilnehmers verglichen wird. Wenn die IDs übereinstimmen, ist der Teilnehmer authentifiziert und die Nachricht kann gesendet werden. Wenn die IDs nicht übereinstimmen, ist der Teilnehmer nicht authentifiziert und die Nachricht wird nicht gesendet. In diesem Fall wird eine Fehlermeldung an den Absender zurückgegeben, dass der Teilnehmer nicht authentifiziert ist. Wie diese Fehlermeldung implementiert wird, ist dem Entwickler überlassen.
\\
\\
Nachrichtenbezogene Fehler:
\\
\\
Es besteht die Möglichkeit, dass Teile von Nachrichten oder auch ganze Nachrichten verloren gehen können. Da Kademlia über UDP läuft, hat es nicht die Möglichkeiten, die beispielsweise TCP bietet. TCP kann bei Paketverlust die Pakete erneut senden, UDP kann dies nicht. Zusätzlich können Nachrichten beschädigt werden (\textcolor{red}{Prüfsumme?}).

%\section{Protokolldesign und -struktur}
Das Protokoll ist wie folgt strukturiert...

\subsection{Grundlagen des Protokolls}

Für ein Peer-to-Peer Netzwerk gibt es verschiedene Typen. Abbildung \ref{p2p_typen} zeigt 
vier Typen und ihre Unterteilung in unstrukturierte und strukturierte Netzwerke.

\begin{center}
    \captionsetup{type=figure}
    \includegraphics[width=1\linewidth]{images/peer_to_peer_typen.png}
    \captionof{figure}{Typen von Peer-to-Peer Netzwerken \parencite{Luntovskyy_ModRechnernetze}}
    \label{p2p_typen}
\end{center}

\noindent Das Protokoll dieser Arbeit fällt in die Kategorie der hybriden Peer-to-Peer Netzwerke.
Es ist sowohl strukturiert als auch unstrukturiert. Alle Teilnehmer sind in einem Netzwerk organisiert,
das aus verschiedenen Knoten besteht, wobei jeder Knoten einen Teilnehmer des Netzwerks repräsentiert.
Alle Knoten im Netzwerk sind untereinander verbunden und können Textnachrichten direkt, ohne die Verwendung 
eines Servers austauschen. Sollte sich jedoch der gewünschte Empfänger nicht im Netzwerk befinden, besteht
die Möglichkeit, dass sich der Empfänger hinter einem NAT-Gateway oder einer Firewall befindet. In diesem
Fall wird die Nachricht über einen Server, genauer gesagt über ein TCP-Relay, an den Empfänger weitergeleitet.
Der Server ist nur für die Weiterleitung der Nachricht zuständig und speichert diese nicht. Dies ist
ein Unterschied zu anderen Instant-Messaging-Protokollen, wie zum Beispiel \textcolor{red}{[Beispiel 
finden oder Satz weglassen]}, bei dem der Server die Nachrichten speichert, wenn der Empfänger offline ist. 
Das Protokoll dieser Arbeit ist also ein hybrides Peer-to-Peer-Protokoll, da es sowohl strukturierte 
als auch unstrukturierte Eigenschaften aufweist.



\subsection{Verbindung und Identität}

Um eine Verbindung zwischen zwei Teilnehmern herzustellen, muss ein Teilnehmer den öffentlichen
Schlüssel des anderen Teilnehmers kennen. Dieser wurde bei der Registrierung in die Blockchain
geschrieben. Da in der Hashtabelle Schlüssel-Wert-Paare gespeichert werden, wird
der öffentliche Schlüssel als Schlüssel und die IP-Adresse und der Port des Teilnehmers als
Wert gespeichert. Der Teilnehmer, der eine Verbindung herstellen möchte, muss den öffentlichen
Schlüssel des anderen Teilnehmers kennen. Dazu muss er den öffentlichen Schlüssel in der
verteilten Hashtabelle suchen. Dazu berechnet er aus dem öffentlichen Schlüssel einen Hashwert.
Der Hashwert wird in einem Zahlenbereich abgebildet. Der Teilnehmer sucht in der verteilten
Hashtabelle nach dem nächsten Hashwert. Der Teilnehmer, der den Hashwert in der verteilten
Hashtabelle gespeichert hat, ist der Nachbar des Teilnehmers, der die Verbindung herstellen
möchte. Der Teilnehmer, der die Verbindung herstellen möchte, sendet eine Anfrage an den
Nachbarn. Die Anfrage enthält den öffentlichen Schlüssel des Teilnehmers, der die Verbindung
herstellen möchte. Der Nachbar antwortet mit seiner IP-Adresse und seinem Port. Der Teilnehmer,
der die Verbindung herstellen möchte, kann nun eine Verbindung zum Nachbarn aufbauen. Der
Nachbar leitet die Nachrichten an den Teilnehmer weiter, der die Verbindung herstellen möchte.

\subsection{Nachrichten und Daten}

Benutzer können ausschließlich Textnachrichten miteinander austauschen. 
Diese Nachrichten können an einzelne Peers gesendet werden.
Die gesamte Kommunikation ist mit starken Verschlüsselungsalgorithmen 
verschlüsselt, um Privatsphäre und Sicherheit zu gewährleisten.

\subsection{Nachrichtenverlauf}

Benutzer haben die Möglichkeit, ihren Nachrichtenverlauf lokal zu speichern.
Der Nachrichtenverlauf ist ebenfalls verschlüsselt, um die Sicherheit zu gewährleisten.

\subsection{Sicherheit}

\begin{itemize}
    \item Integration von Verschlüsselung für Datenschutz und Sicherheit 
    \item Schutz vor Angriffen wie Man-in-the-Middle
\end{itemize}

Alle Nachrichten sind mit öffentlichen Schlüsseln verschlüsselt, um sicherzustellen, 
dass nur der beabsichtigte Empfänger die Nachrichten entschlüsseln und lesen kann.
Ein sicheres Verfahren für den Schlüsselaustausch und die Schlüsselverwaltung 
wird implementiert, um gegen Abhören und Man-in-the-Middle-Angriffe zu schützen.


%\section{Technologien und Tools}
Für die Umsetzung dieses Designs wurden diese Technologien verwendet.

%\section{Integration von Blockchain}
\label{sec:blockchainintegration}


\subsection{Auswahl der Blockchain}
Für die Integration der Blockchain in das Protokoll wurde zunächst eine geeignete Blockchain gesucht. Dabei wurde sich auf die beiden bekanntesten Blockchains, Bitcoin und Ethereum, beschränkt. Da Bitcoin eine reine Kryptowährung ist und keine Smart Contracts unterstützt, wurde sich für Ethereum entschieden. Ethereum ist eine Blockchain, die zwar auch eine Kryptowährung, Ether, besitzt, aber zusätzlich auch Smart Contracts unterstützt. Der Grund für die Existenz einer Währung auf der Blockchain ist, dass die Smart Contracts, die auf der Blockchain ausgeführt werden, mit Ether bezahlt werden müssen \parencite[S. 2]{Antonopoulos_MasteringEthereum}. Somit ist es nicht möglich, Smart Contracts auf der Blockchain auszuführen, ohne Ether zu besitzen. Da die Smart Contracts auf der Blockchain aber mit Ether bezahlt werden müssen, ist es notwendig, dass jeder Nutzer, der einen Smart Contract ausführen möchte, Ether besitzt.


\subsection{Registrierung}
Um das Protokoll zu nutzen, muss sich jeder Nutzer zunächst auf der Blockchain registrieren. Dazu muss ein Smart Contract auf der Blockchain ausgeführt werden, der die Registrierung des Nutzers durchführt. Dieser Smart Contract wird mit dem Benutzernamen und dem statischen öffentlichen Schlüssel aufgerufen. Der statische öffentliche Schlüssel wird bei der Registrierung festgelegt und kann nicht mehr geändert werden. Der Smart Contract erstellt einen neuen Eintrag in der Blockchain, der den Benutzernamen und den öffentlichen Schlüssel des Nutzers enthält. Der Smart Contract wird nur einmalig bei der Registrierung aufgerufen. Sollte ein Nutzer seinen Benutzernamen ändern wollen, muss er sich mit dem neuen Benutzernamen und dem öffentlichen Schlüssel eines neu erzeugten statischen Schlüsselpaars erneut registrieren.
% Der alte Benutzername wird dann aus der Blockchain gelöscht. -> wie könnte man das absichern?

\subsection{Kommunikation}
Für die Kommunikation mit anderen Teilnehmern, muss zunächst die ID des anderen Teilnehmers im Kademlia-Netzwerk bekannt sein. Dazu wird der Benutzername des anderen Teilnehmers auf der Blockchain gesucht um zu kontrollieren, ob dieser bereits registriert ist. Ist der Benutzername nicht auf der Blockchain vorhanden, ist der andere Teilnehmer nicht registriert und es kann keine Verbindung aufgebaut werden. Ist der Benutzername auf der Blockchain vorhanden, wird der dazugehörige öffentliche Schlüssel ausgelesen. Der Benutzername entspricht gleichzeitig der ID des Teilnehmers im Kademlia-Netzwerk. Somit kann der Verbindungsaufbau, der in Abschnitt \ref{subsec:verbindungsmanagement} \nameref{subsec:verbindungsmanagement} beschrieben wird, durchgeführt werden.
Wenn der Verbindungsaufbau erfolgreich war, kann die Kommunikation beginnen. Dazu wird der öffentliche Schlüssel des anderen Teilnehmers benötigt. Dieser wird ebenfalls auf der Blockchain gespeichert. Somit kann jeder Teilnehmer die öffentlichen Schlüssel der anderen Teilnehmer auf der Blockchain finden und die Nachrichten, die er erhält, mit dem öffentlichen Schlüssel des Absenders verifizieren. Somit kann sichergestellt werden, dass die Nachrichten tatsächlich vom angegebenen Absender stammen und nicht von einem anderen Teilnehmer gesendet wurden, der sich als jemand anderes ausgibt.

\subsection{Entwurf der Smart Contracts}
\label{subsec:smartcontracts}

Auf der Ethereum-Blockchain wird die hauptsächlich Programmiersprache Solidity verwendet, um Smart Contracts zu erstellen. Solidity ist eine objektorientierte Programmiersprache, die stark an JavaScript angelehnt ist \parencite[S. 131]{Antonopoulos_MasteringEthereum}. Die Smart Contracts, die für das Protokoll benötigt werden, sind in Solidity implementiert.


