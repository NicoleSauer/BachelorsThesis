\chapter*{Kurzfassung}

Diese Bachelorarbeit präsentiert die Entwicklungsarbeit an einem innovativen Peer-to-Peer-Instant-Messaging-Protokoll, das auf einer Kombination von Kademlia und Interactive Connectivity Establishment (ICE) basiert. Der Fokus liegt auf der Schaffung eines dezentralen, robusten Peer-to-Peer Kommunikationsprotokolls, das unabhängig von zentralen Servern agiert.

Die Authentifizierung der Teilnehmer erfolgt über öffentliche Schlüssel, die sicher in der Ethereum Blockchain hinterlegt sind. Diese Blockchain-basierte Authentifizierung gewährleistet eine vertrauenswürdige Identitätsüberprüfung, wobei die dezentrale Natur von Ethereum eine hohe Sicherheit und Manipulationssicherheit bietet.

Ein zentraler Beitrag dieser Arbeit ist die Integration von Kademlia, einem verteilten Peer-to-Peer Routing-Algorithmus, und ICE, einem Mechanismus zur Überquerung von Netzwerkgrenzen. Diese Kombination ermöglicht eine effiziente und zuverlässige P2P-Kommunikation, wodurch die Abhängigkeit von zentralen Servern minimiert wird. Die Implementierung wurde darauf ausgerichtet, eine hohe Skalierbarkeit und Robustheit des Systems sicherzustellen.

Die präsentierte Lösung bietet nicht nur eine sichere Peer-to-Peer Kommunikationsumgebung, sondern betont auch die Bedeutung der Blockchain-Technologie für die Authentifizierung in dezentralen Systemen. Das entwickelte Protokoll demonstriert, wie eine Kombination von Ethereum, Kademlia und ICE die Grundlage für eine vertrauenswürdige und effiziente Peer-to-Peer-Instant-Messaging-Infrastruktur schaffen kann.