\section{Nicht-funktionale Anforderungen}

%Dies sind Beschränkungen der durch das System angebotenen Dienste oder Funktionen. Das schließt 
%Zeitbeschränkungen, Beschränkungen des Entwicklungsprozesses und einzuhaltende Standards ein.

Nicht-funktionale Anforderungen sind Anforderungen, die sich nicht auf eine 
spezifische Funktionalität einer Software-Anwendung beziehen, sondern auf die Gesamtstruktur und damit auf
Qualitätsmerkmale und Aspekte, wie beispielsweise die Leistung, Zuverlässigkeit, Antwortzeit und Benutzerfreundlichkeit 
der Software. 
Diese Anforderungen beschreiben, \textit{wie} die Software funktionieren sollte, anstatt \textit{was} sie tun sollte. 
Nicht-funktionale Anforderungen sind genauso wichtig wie funktionale Anforderungen, da sie einen 
erheblichen Einfluss auf die Gesamtleistung und die Benutzerzufriedenheit haben können
\parencite[S. 126-130]{AnfAnalyse_Sommerville}.
% #TODO: find and add source 
\\
Die folgenden nicht-funktionale Anforderungen soll das Protokoll bieten:

\begin{itemize}
    \item Performanz
    \item Sicherheit
    \item Zuverlässigkeit
    \item Kompatibilität
    %\item Interoperabilität
    \item Privatsphäre
    \item Skalierbarkeit
\end{itemize}

\noindent Die folgenden Abschnitte beschreiben die nicht-funktionalen Anforderungen an das Protokoll.