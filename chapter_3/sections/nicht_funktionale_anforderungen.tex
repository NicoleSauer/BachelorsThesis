\section{Nicht-funktionale Anforderungen}


Nicht-funktionale Anforderungen sind Anforderungen, die sich nicht auf eine 
spezifische Funktionalität einer Software-Anwendung beziehen, sondern auf die Gesamtstruktur und damit auf
Qualitätsmerkmale und Aspekte, wie beispielsweise die Leistung, Zuverlässigkeit, Reaktionszeit und Benutzerfreundlichkeit 
der Software. 
Diese Anforderungen beschreiben, \textit{wie} die Software funktionieren sollte, anstatt \textit{was} sie tun sollte. 
Nicht-funktionale Anforderungen sind genauso wichtig wie funktionale Anforderungen, da sie einen 
erheblichen Einfluss auf die Gesamtleistung und die Benutzerzufriedenheit haben können
\parencite[S. 126-130]{Sommerville_AnfAnalyse}.

Die folgenden nicht-funktionale Anforderungen sollte das Protokoll bieten:

\begin{itemize}
    \item Performanz
    \item Sicherheit
    \item Zuverlässigkeit
    \item Kompatibilität
    \item Skalierbarkeit
\end{itemize}

\noindent Die folgenden Abschnitte beschreiben die nicht-funktionalen Anforderungen an das Protokoll.


\subsection{Performanz}

Die Leistungsanforderungen an ein Peer-to-Peer Instant-Messaging-Protokoll beziehen sich auf die Effizienz und Geschwindigkeit der Nachrichtenübertragung, um sicherzustellen, dass Benutzer eine reaktionsschnelle und nahtlose Kommunikation erfahren. Die Leistungsfähigkeit des Protokolls beeinflusst direkt die Benutzerzufriedenheit und die Gesamterfahrung der Instant-Messaging-Anwendung, die das Protokoll verwendet. 

\subsection{Sicherheit}

Das entwickelte Protokoll möglichst sicher sein, um die Vertraulichkeit, Integrität und Authentizität der übertragenen Nachrichten und der Benutzerdaten gewährleisten zu können. Ein sicherheitsorientiertes Protokoll minimiert das Risiko von unbefugtem Zugriff, Datenmanipulation oder anderen Bedrohungen, die die Vertraulichkeit und Integrität der Kommunikation gefährden könnten.


\subsection{Zuverlässigkeit}

Das Protokoll soll eine hohe Zuverlässigkeit bieten. Dazu gehört die Fähigkeit, Nachrichten
zuverlässig zu übertragen. Die Zuverlässigkeit ist ein wichtiger Aspekt, da die Benutzer eine
zuverlässige Kommunikation erwarten.

\subsection{Kompatibilität}

% Das Protokoll soll mit Ethernet, Wi-Fi, IP, TCP, TLS, und weiteren Peer-to-Peer Instant Messaging 
% Protokollen kompatibel sein.


Das Protokoll soll mit Ethernet, Wi-Fi, IP, TCP, TLS, und weiteren Peer-to-Peer Instant Messaging
Protokollen kompatibel sein. Ethernet und Wi-Fi sind Standards für kabelgebundene und kabellose
Netzwerke. IP ist ein Standard für die Adressierung und das Routing von Paketen in einem Netzwerk.
TCP ist ein Standard für die zuverlässige Übertragung von Daten in einem Netzwerk. TLS ist ein
Standard für die sichere Übertragung von Daten in einem Netzwerk. Die Kompatibilität ist ein
wichtiger Aspekt, da die Benutzer eine einfache Kommunikation erwarten.


\subsection{Skalierbarkeit}

Die Skalierbarkeit eines Peer-to-Peer Instant-Messaging-Protokolls ist entscheidend, um sicherzustellen, dass die Kommunikationseffizienz und -qualität auch bei steigender Benutzerzahl erhalten bleibt. Skalierbarkeit bezieht sich auf die Fähigkeit des Protokolls, mit einem zunehmenden Benutzerwachstum umzugehen, ohne dabei signifikante Einbußen in Bezug auf Leistung und Reaktionsfähigkeit zu erleiden.

