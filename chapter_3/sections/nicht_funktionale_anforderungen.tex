\section{Nicht-funktionale Anforderungen}

%Dies sind Beschränkungen der durch das System angebotenen Dienste oder Funktionen. Das schließt 
%Zeitbeschränkungen, Beschränkungen des Entwicklungsprozesses und einzuhaltende Standards ein.

Nicht-funktionale Anforderungen sind Anforderungen, die sich nicht auf die 
spezifische Funktionalität einer Software-Anwendung beziehen, sondern auf andere Qualitätsmerkmale 
und Aspekte, die die Leistung, Zuverlässigkeit und Benutzerfreundlichkeit der Software betreffen. 
Diese Anforderungen beschreiben, wie die Software funktionieren sollte, anstatt was sie tun sollte. 
Nicht-funktionale Anforderungen sind genauso wichtig wie funktionale Anforderungen, da sie einen 
erheblichen Einfluss auf die Gesamtleistung und die Benutzerzufriedenheit haben können.