\section{Nicht-funktionale Anforderungen}


Nicht-funktionale Anforderungen sind Anforderungen, die sich nicht auf eine 
spezifische Funktionalität einer Software-Anwendung beziehen, sondern auf die Gesamtstruktur und damit auf
Qualitätsmerkmale und Aspekte, wie beispielsweise die Leistung, Zuverlässigkeit, Reaktionszeit und Benutzerfreundlichkeit 
der Software. 
Diese Anforderungen beschreiben, \textit{wie} die Software funktionieren sollte, anstatt \textit{was} sie tun sollte. 
Nicht-funktionale Anforderungen sind genauso wichtig wie funktionale Anforderungen, da sie einen 
erheblichen Einfluss auf die Gesamtleistung und die Benutzerzufriedenheit haben können
\parencite[S. 126-130]{Sommerville_AnfAnalyse}.

Die folgenden nicht-funktionale Anforderungen sollte das Protokoll bieten:

\begin{itemize}
    \item Performanz
    \item Sicherheit
    \item Zuverlässigkeit
    \item Kompatibilität
    %\item Interoperabilität
    \item Skalierbarkeit
\end{itemize}

\noindent Die folgenden Abschnitte beschreiben die nicht-funktionalen Anforderungen an das Protokoll.


\subsection{Performanz}

Das Protokoll soll eine hohe Performanz bieten. Dazu gehört eine geringe Latenzzeit, eine hohe
Durchsatzrate und eine hohe Skalierbarkeit. Die Latenzzeit ist die Zeit, die zwischen dem Senden einer Nachricht und dem
Empfangen der Nachricht durch den Empfänger vergeht. Die Durchsatzrate ist die Anzahl der Nachrichten, die pro Zeiteinheit
übertragen werden können. Die Skalierbarkeit ist die Fähigkeit des Systems, die Leistung bei steigender Anzahl von Benutzern
aufrechtzuerhalten. Die Performanz ist ein wichtiger Aspekt, da die Benutzer eine schnelle und zuverlässige Kommunikation
erwarten.

\subsection{Sicherheit}

Das Protokoll soll eine hohe Sicherheit bieten. Dazu gehört die Vertraulichkeit, 
Integrität und Authentizität der Nachrichten. Die Vertraulichkeit ist die Eigenschaft, dass nur der
beabsichtigte Empfänger die Nachricht lesen kann. Die Integrität ist die Eigenschaft, dass 
die Nachricht nicht verändert wurde. Die Authentizität ist die Eigenschaft, dass der Absender 
der Nachricht der ist, der er vorgibt zu sein. Die Sicherheit ist ein wichtiger Aspekt, da die
Benutzer eine sichere Kommunikation erwarten.


\subsection{Zuverlässigkeit}

Die Zuverlässigkeit des zu entwickelnden Instant-Messaging-Protokolls bezieht sich darauf, dass das System kontinuierlich und konsistent arbeitet, selbst unter verschiedenen Lastszenarien. Eine zuverlässige Kommunikation ist essentiell, um sicherzustellen, dass Nachrichten korrekt zugestellt werden und Benutzer jederzeit auf den Instant-Messaging-Dienst zugreifen können.
\subsection{Kompatibilität}

% Das Protokoll soll mit Ethernet, Wi-Fi, IP, TCP, TLS, und weiteren Peer-to-Peer Instant Messaging 
% Protokollen kompatibel sein.


Das Protokoll soll mit Ethernet, Wi-Fi, IP, TCP, TLS, und weiteren Peer-to-Peer Instant Messaging
Protokollen kompatibel sein. Ethernet und Wi-Fi sind Standards für kabelgebundene und kabellose
Netzwerke. IP ist ein Standard für die Adressierung und das Routing von Paketen in einem Netzwerk.
TCP ist ein Standard für die zuverlässige Übertragung von Daten in einem Netzwerk. TLS ist ein
Standard für die sichere Übertragung von Daten in einem Netzwerk. Die Kompatibilität ist ein
wichtiger Aspekt, da die Benutzer eine einfache Kommunikation erwarten.


%\subsection{Interoperabilität}

Die Interoperabilität eines Instant-Messaging-Protokolls bezeichnet die Fähigkeit, effektiv mit anderen Messaging-Systemen zu kommunizieren. Ein interoperables Protokoll ermöglicht den Austausch von Nachrichten zwischen verschiedenen Instant-Messaging-Diensten, indem es offene Standards und Schnittstellen nutzt. Dies fördert die nahtlose Kommunikation zwischen Benutzern, unabhängig davon, welchen Messaging-Dienst sie verwenden, und erleichtert die Integration in unterschiedliche Anwendungen und Plattformen. Eine robuste Interoperabilität ist entscheidend für die Erweiterung der Reichweite und Benutzerfreundlichkeit des Instant-Messaging-Protokolls.

\subsection{Skalierbarkeit}

Die Skalierbarkeit eines Peer-to-Peer Instant-Messaging-Protokolls ist entscheidend, um sicherzustellen, dass die Kommunikationseffizienz und -qualität auch bei steigender Benutzerzahl erhalten bleibt. Skalierbarkeit bezieht sich auf die Fähigkeit des Protokolls, mit einem zunehmenden Benutzerwachstum umzugehen, ohne dabei signifikante Einbußen in Bezug auf Leistung und Reaktionsfähigkeit zu erleiden.

