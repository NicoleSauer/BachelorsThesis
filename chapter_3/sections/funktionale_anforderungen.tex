\section{Funktionale Anforderungen}


%Dies sind Aussagen zu den Diensten, die das System leisten sollte, zur Reaktion des Systems auf bestimmte 
%Eingaben und zum Verhalten des Systems in bestimmten Situationen. 

Funktionale Anforderungen beziehen sich auf die spezifischen Funktionen und Aufgaben, 
die eine Software oder ein System erfüllen muss, um die Bedürfnisse und Erwartungen der Benutzer zu erfüllen.
Sie beschreiben, was das System tun soll, welche Aktionen es ausführen muss und welche 
Ergebnisse es liefern sollte. Diese Anforderungen sind entscheidend, um sicherzustellen, dass die entwickelte Software 
oder wie in diesem Fall, das entwickelte Protokoll die erwarteten Funktionen erbringt. Sie dienen 
als Grundlage für das Design, die Entwicklung, die Validierung und die Verifizierung von Software-Systemen und 
sind ein wichtiger Bestandteil des Anforderungsmanagements im Software-Engineering-Prozess.
% #TODO: find and add source 
\\

\noindent Folgende Funktionen soll das Protokoll bieten:

\begin{itemize}
    \item Registrierung des Nutzers
    \item Authentifizierung des Nutzers
    \item Verschlüsselung der Kommunikation
    \item Versenden und Empfangen von Textnachrichten
    \item Kontaktmanagement (Kontakt hinzufügen, löschen, blockieren)
    \item Handhabung von Nachrichten, wenn der Empfänger offline ist
\end{itemize}


%\subsection{Benutzeranforderungen und -erwartungen}

Benutzer erwarten die folgenden Funktionen...
%\subsection{Sicherheitsanforderungen an das Protokoll}

Sicherheitsanforderungen sind ...