\section{Funktionale Anforderungen}


%Dies sind Aussagen zu den Diensten, die das System leisten sollte, zur Reaktion des Systems auf bestimmte 
%Eingaben und zum Verhalten des Systems in bestimmten Situationen. 

Funktionale Anforderungen beziehen sich auf die spezifischen Funktionen und Aufgaben, 
die eine Software oder ein System erfüllen muss, um die Bedürfnisse und Erwartungen der Benutzer zu erfüllen.
Diese Anforderungen beschreiben, was das System tun soll, welche Aktionen es ausführen muss und welche 
Ergebnisse es liefern sollte.
\\

\noindent Folgende Funktionen soll das Protokoll bieten:

\begin{itemize}
    \item Registrierung des Nutzers
    \item Authentifizierung des Nutzers
    \item Verschlüsselung der Kommunikation
    \item Versenden von Textnachrichten
    \item Nachrichtenaustausch
    \item Kontaktmanagement (Kontakt hinzufügen, löschen, blockieren)
    \item Handhabung von Nachrichten, wenn der Empfänger offline ist
\end{itemize}


%\subsection{Benutzeranforderungen und -erwartungen}

Benutzer erwarten die folgenden Funktionen...
%\subsection{Sicherheitsanforderungen an das Protokoll}

Sicherheitsanforderungen sind ...