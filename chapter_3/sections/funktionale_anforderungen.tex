\section{Funktionale Anforderungen}


%Dies sind Aussagen zu den Diensten, die das System leisten sollte, zur Reaktion des Systems auf bestimmte 
%Eingaben und zum Verhalten des Systems in bestimmten Situationen. 

Funktionale Anforderungen beziehen sich auf die spezifischen Funktionen und Aufgaben, 
die eine Software oder ein System erfüllen muss, um die Bedürfnisse und Erwartungen der Benutzer zu erfüllen.
Sie beschreiben, was das System tun soll, welche Aktionen es ausführen muss und welche 
Ergebnisse es liefern sollte. Mitunter werden sie auch dazu verwendet, um festzuhalten, was das System
nicht können soll.
All diese Anforderungen sind entscheidend, um sicherzustellen, dass die entwickelte Software 
oder wie in diesem Fall, das entwickelte Protokoll die erwarteten Funktionen erbringt. Sie dienen 
als Grundlage für das Design, die Entwicklung, die Validierung und die Verifizierung von Software-Systemen und 
sind ein wichtiger Bestandteil des Anforderungsmanagements im Software-Engineering-Prozess 
\parencite[S.124-126]{AnfAnalyse_Sommerville}.

\noindent Um die Anforderungen an das zu entwickelnde Protokoll zu definieren, wird es in die folgenden
Funktionen unterteilt:

\begin{itemize}
    \item Registrierung
    \item Authentifizierung
    \item Verschlüsselung der Kommunikation
    \item Versenden und Empfangen von Textnachrichten
    \item Kontaktmanagement (Kontakt hinzufügen, löschen, blockieren)
    \item Handhabung von Nachrichten, wenn der Empfänger offline ist
\end{itemize}

\noindent Die folgenden Abschnitte beschreiben die funktionalen Anforderungen an das Protokoll.
\\

\subsection{Registrierung}

% In meiner Bachelorarbeit entwickle ich ein Peer-to-Peer-Instant-Messaging-Protokoll. 
% Um für eine gewisse Sicherheit zu sorgen, möchte ich neuen Usern eine ID und einen Public Key zuordnen 
% und diese mittels Smart Contract auf die Ethereum-Blockchain schreiben.

Der erste Schritt ist die Registrierung eines neuen Benutzers. Hierfür muss der Benutzer einen Benutzernamen
und ein Passwort angeben. Der Benutzername muss nicht eindeutig sein, da von der Anwendung eine ID generiert wird, 
mit deren Hilfe der Nutzer identifiziert werden kann.
Das Passwort muss mindestens zehn Zeichen lang sein und mindestens einen Großbuchstaben, einen Kleinbuchstaben,
eine Zahl und ein Sonderzeichen enthalten. Diese Anforderungen sollen sicherstellen, dass das Passwort
nicht leicht zu erraten ist und somit die Sicherheit des Benutzers gewährleistet ist.
Nachdem der Benutzername und das Passwort eingegeben wurden, wird ein Schlüsselpaar auf 
Basis des Passworts erzeugt.
% #TODO: mit welchem Algorithmus? -> Argon2 https://cryptobook.nakov.com/mac-and-key-derivation/argon2


% #TODO: könnte man auch einen Passkey verwenden? Wie funktionieren die?

\subsection{Authentifizierung}

Die Authentifizierung ist der zweite Schritt, um das Protokoll nutzen zu können. 
Dabei wird der Benutzername und das Passwort, die bei der Registrierung festgelegt wurden,
benötigt. Das Protokoll überprüft, ob der Benutzername und das Passwort korrekt sind.
Wenn die Authentifizierung erfolgreich ist, wird ein Token generiert, der für die
Kommunikation mit dem Server benötigt wird. Dieser Token wird an den Client zurückgegeben.
\\
\\
\noindent Die Authentifizierung ist notwendig, um sicherzustellen, dass nur registrierte Benutzer
das Protokoll nutzen können. Außerdem wird durch die Authentifizierung sichergestellt, dass
die Nachrichten, die ein Benutzer sendet, auch wirklich von diesem Benutzer stammen.
\subsection{Verschlüsselung der Kommunikation}

% Wenn Public Key des anderen bekannt ist, gilt das als authentifizierter Schlüsselaustausch. Bietet Schutz 
% vor Man in the Middle Attacken, da bekannt ist, wem der Public Key gehört. Da in meinem Fall beide
% authentifiziert sind, ist es ein wechselseitiger authentifizierter Schlüsselaustausch.

Für die Verschlüsselung der Kommunikation soll das Public-Key-Verfahren verwendet werden.
Das Public-Key-Verfahren ist ein asymmetrisches Verschlüsselungsverfahren, das zwei Schlüssel verwendet,
einen öffentlichen und einen privaten Schlüssel. Das soll gewährleisten, dass die Nachrichten nur von
dem Benutzer gelesen werden können, für den sie bestimmt sind und erkannt werden kann, ob die Nachricht
manipuliert wurde. Zudem erhöht es die Sicherheit des Schlüsselaustauschs, da es praktisch unmöglich ist,
den privaten Schlüssel aus dem öffentlichen Schlüssel zu berechnen.


% Der öffentliche Schlüssel wird verwendet, um Nachrichten
% zu verschlüsseln, und der private Schlüssel wird verwendet, um Nachrichten zu entschlüsseln. Der öffentliche
% Schlüssel wird an alle Benutzer verteilt, während der private Schlüssel nur dem Benutzer bekannt ist.

%#TODO: DH bzw ECDH muss in den Grundlagen erklärt werden

% Für die Verschlüsselung wird der Elliptic Curve Diffie-Hellman-Schlüsselaustausch verwendet.
% Der Elliptic Curve Diffie-Hellman-Schlüsselaustausch ist ein Schlüsselaustauschprotokoll, das auf dem
% Diffie-Hellman-Schlüsselaustausch basiert. Der Diffie-Hellman-Schlüsselaustausch ist ein Schlüsselaustauschprotokoll,
% das verwendet wird, um einen gemeinsamen geheimen Schlüssel zwischen zwei Parteien zu erzeugen. Der
% gemeinsame geheime Schlüssel wird verwendet, um die Nachrichten zu verschlüsseln und zu entschlüsseln.
% Der Elliptic Curve Diffie-Hellman-Schlüsselaustausch verwendet elliptische Kurven, um die Sicherheit des
% Schlüsselaustauschs zu erhöhen. Die elliptische Kurve ist eine mathematische Funktion, die verwendet wird,
% um die Punkte zu berechnen, die für die Verschlüsselung und Entschlüsselung der Nachrichten verwendet werden.
% Der Elliptic Curve Diffie-Hellman-Schlüsselaustausch ist sicher, da es praktisch unmöglich ist, den
% gemeinsamen geheimen Schlüssel aus den öffentlichen Schlüsseln zu berechnen.
 

%\subsection{Benutzeranforderungen und -erwartungen}

Benutzer erwarten die folgenden Funktionen...
%\subsection{Sicherheitsanforderungen an das Protokoll}

Sicherheitsanforderungen sind ...