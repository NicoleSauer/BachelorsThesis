\subsection{Handhabung von Offline-Nachrichten}

Eine direkte Peer to Peer Kommunikation ist nur möglich, wenn keiner der beiden Benutzer hinter einem 
Network Address Translation (NAT) Gateway sitzt. Da dies allerdings in der Regel nicht der Fall ist,
muss eine andere Lösung gefunden werden. Eine Möglichkeit ist die Verwendung eines Relay Servers.
Der Relay Server ist ein Server, der die Nachrichten zwischen den Benutzern weiterleitet, wenn eine 
reine Peer to Peer Kommunikation nicht möglich ist. Wenn ein Benutzer offline ist, kann der Relay 
Server die Nachricht nicht an den Benutzer weiterleiten.
Deshalb muss das Protokoll eine Möglichkeit bieten, Nachrichten auch an Teilnehmer zu senden, die
offline sind. In diesem Fall könnte die Anwendung die Nachrichten lokal speichern und sie senden, 
sobald der Empfänger wieder online ist oder immer wieder versuchen, die Nachricht zu senden, bis 
sie erfolgreich gesendet wurde.

\noindent Die folgenden Funktionen sind Teil der Handhabung von Offline-Nachrichten:

\begin{itemize}
    \item Nachrichten speichern, wenn der Empfänger offline ist
    \item Nachrichten senden, wenn der Empfänger wieder online ist
    \item Nachrichten speichern, wenn der Sender offline ist
    \item Nachrichten senden, wenn der Sender wieder online ist
\end{itemize}
