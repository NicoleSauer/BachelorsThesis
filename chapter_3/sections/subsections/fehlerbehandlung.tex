\subsection{Fehlerbehandlung und Wiederholungsmechanismen}

Die Fehlerbehandlung und die Implementierung von Wiederholungsmechanismen stellen entscheidende Komponenten im Verlauf einer Peer-to-Peer Instant-Messaging-Kommuni-\\kation dar. Ein zentraler Aspekt der Fehlerbehandlung ist die Erkennung von Übertragungsfehlern. Das Protokoll sollte in der Lage sein, Fehlerzustände während des Nachrichtenaustauschs zu identifizieren. Dies können beispielsweise fehlerhafte Pakete, verlorene Verbindungen oder andere unvorhergesehene Probleme sein. Die Fehlererkennung ermöglicht es, schnell auf Probleme zu reagieren und entsprechende Maßnahmen einzuleiten. Die Wiederholungsmechanismen sind eng mit der Fehlerbehandlung verbunden und dienen dazu, sicherzustellen, dass fehlgeschlagene Übertragungen erneut versucht werden. Dies könnte durch automatisches erneutes Senden von Nachrichten oder das Auslösen spezifischer Protokollmechanismen geschehen. Die Wiederholungsmechanismen sind darauf ausgerichtet, die Zustellung von Nachrichten trotz vorübergehender Probleme im Netzwerk zu gewährleisten.