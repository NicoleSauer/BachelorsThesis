\subsection{Kontaktmanagement}

Das Kontaktmanagement ist ein wichtiger Bestandteil des Protokolls. Es ermöglicht den Benutzern, ihre Kontakte
zu verwalten. Die folgenden Funktionen sind Teil des Kontaktmanagements:

\begin{itemize}
    \item Kontakt hinzufügen
    \item Kontakt löschen
    \item Kontakt blockieren
    \item Kontakt entsperren
    \item Kontaktliste anzeigen
    \item Kontakte suchen
\end{itemize}

\noindent Jeder Benutzer besitzt eine Kontaktliste und eine Blockierliste. Die Kontaktliste enthält alle Kontakte,
die nicht blockiert sind. Die Blockierliste enthält alle Kontakte, die blockiert sind.
Um einen neuen Kontakt hinzuzufügen, muss der Benutzer den Benutzernamen des Kontakts eingeben. 
Der Benutzername wird dann in der Blockchain gesucht und wenn er gefunden wurde, wird der Kontakt zur 
Kontaktliste hinzugefügt. Der Benutzer kann nun die Kommunikation mit dem Kontakt aufnehmen.
Um einen Kontakt zu löschen, wird der Kontakt aus der Kontaktliste entfernt. Um einen Kontakt zu blockieren,
wird der Kontakt aus der Kontaktliste entfernt und in die Blockierliste verschoben. Um einen Kontakt zu entsperren,
wird der Kontakt aus der Blockierliste entfernt und zurück in die Kontaktliste verschoben.
Um einen Kontakt zu suchen, muss der Benutzer den Benutzernamen des Kontakts eingeben. Der Benutzername
wird dann in der Kontaktliste gesucht. Wenn der Benutzername gefunden wurde, wird der Kontakt angezeigt.
