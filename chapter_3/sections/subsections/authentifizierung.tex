\subsection{Authentifizierung}


Bei der Authentifizierung wird die Identität des Benutzers überprüft.
Dabei sollen die bei der Registrierung festgelegten Zugangsdaten verwendet werden. Die Authentifizierung 
soll sicherstellen, dass der Benutzer der ist, für den er sich ausgibt.
Bei einem Verbindungsaufbau wird mittels des öffentlichen Schlüssels des Nachrichtenempfängers sichergestellt,
dass die Nachrichten an den richtigen Empfänger gesendet werden. Der Nachrichtenempfänger wiederum kann
mittels des öffentlich einsehbaren öffentlichen Schlüssels des Nachrichtenabsenders sicherstellen, dass die
Nachrichten vom richtigen Absender stammen.
Die Authentifizierung ist ein wichtiger Schritt, um die Sicherheit des Protokolls zu gewährleisten.



% Die Authentifizierung ist der zweite Schritt nach der Registrierung eines neuen Benutzers. 
% Dabei wird der Benutzername und das Passwort, die bei der Registrierung festgelegt wurden,
% benötigt. Das Protokoll überprüft, ob der Benutzername und das Passwort korrekt sind. %#TODO: wie?
% Wenn die Registrierung erfolgreich war, wurde anschließend ein Schlüsselpaar auf Basis des Passworts erzeugt.
%#TODO: was macht die Registrierung erfolgreich? -> Smart Contract auf Blockchain geschrieben?

%#TODO: muss Public und Private Key nicht auch noch erklärt und gegebenenfalls übersetzt werden?
% Das Schlüsselpaar besteht aus einem öffentlichen Schlüssel (engl. Public Key) und einem privaten Schlüssel 
% (engl. Private Key). Der Public Key wird zusammen mit
% dem Benutzernamen auf die Blockchain geschrieben. Der Private Key wird lokal auf dem
% Gerät des Benutzers gespeichert. Der Private Key wird benötigt, um Nachrichten zu verschlüsseln und
% zu entschlüsseln. Der Public Key wird benötigt, um Nachrichten zu verschlüsseln. Der Public Key wird
% zusammen mit dem Benutzernamen und dem Passwort auf die Blockchain geschrieben, damit andere Benutzer
% den Public Key finden können, um Nachrichten an den Benutzer zu verschlüsseln. Der Public Key wird
% auch benötigt, um die Identität des Benutzers zu verifizieren. Wenn ein Benutzer eine Nachricht an
% einen anderen Benutzer sendet, wird der Public Key des Empfängers verwendet, um die Nachricht zu
% verschlüsseln. Wenn der Empfänger die Nachricht empfängt, verwendet er seinen Private Key, um die
% Nachricht zu entschlüsseln. Wenn der Private Key des Empfängers mit dem Public Key des Absenders
% übereinstimmt, kann der Empfänger sicher sein, dass die Nachricht vom Absender stammt und nicht
% manipuliert wurde.
