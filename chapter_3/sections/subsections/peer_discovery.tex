\subsection{Peer-Discovery und Routing}
\label{subsec:peer_discovery}

Das Protokoll muss es den Benutzern ermöglichen, sich gegenseitig zu finden, um miteinander kommunizieren zu können. Als Grundlage hierfür soll das Internet dienen. Das bedeutet, dass das Protokoll auf IP-Adressen basieren muss. Dazu muss es eine Möglichkeit geben, die IP-Adresse eines Benutzers zu ermitteln, wenn der Benutzername des Ziels bekannt ist. Wenn die IP-Adresse eines Benutzers bekannt ist, muss das Protokoll in der Lage sein, eine Verbindung zu diesem Benutzer herzustellen. Dies ist notwendig, um die Dezentralität des Protokolls zu gewährleisten. Sollte ein Art von zentralem Server benötigt werden, muss die Verschlüsselung der Nachrichten so implementiert werden, dass dem Server nicht vertraut werden muss und dieser damit nicht in der Lage ist, die Nachrichten zu entschlüsseln und zu lesen.