\subsection{Performanz}

Das Protokoll soll eine hohe Performanz bieten. Dazu gehört eine geringe Latenzzeit, eine hohe
Durchsatzrate und eine hohe Skalierbarkeit. Die Latenzzeit ist die Zeit, die zwischen dem Senden einer Nachricht und dem
Empfangen der Nachricht durch den Empfänger vergeht. Die Durchsatzrate ist die Anzahl der Nachrichten, die pro Zeiteinheit
übertragen werden können. Die Skalierbarkeit ist die Fähigkeit des Systems, die Leistung bei steigender Anzahl von Benutzern
aufrechtzuerhalten. Die Performanz ist ein wichtiger Aspekt, da die Benutzer eine schnelle und zuverlässige Kommunikation
erwarten.
