\subsection{Verschlüsselung der Kommunikation}

% Wenn Public Key des anderen bekannt ist, gilt das als authentifizierter Schlüsselaustausch. Bietet Schutz 
% vor Man in the Middle Attacken, da bekannt ist, wem der Public Key gehört. Da in meinem Fall beide
% authentifiziert sind, ist es ein wechselseitiger authentifizierter Schlüsselaustausch.

Für die Verschlüsselung der Kommunikation soll das Public-Key-Verfahren verwendet werden.
Das Public-Key-Verfahren ist ein asymmetrisches Verschlüsselungsverfahren, das zwei Schlüssel verwendet,
einen öffentlichen und einen privaten Schlüssel \parencite[S. 11-13]{Wong_KryptoPraxis}. Das soll gewährleisten, dass die Nachrichten nur von
dem Benutzer gelesen werden können, für den sie bestimmt sind und erkannt werden kann, ob die Nachricht
manipuliert wurde. Zudem erhöht es die Sicherheit des Schlüsselaustauschs, da es praktisch unmöglich ist,
den privaten Schlüssel aus dem öffentlichen Schlüssel zu berechnen.


% Der öffentliche Schlüssel wird verwendet, um Nachrichten
% zu verschlüsseln, und der private Schlüssel wird verwendet, um Nachrichten zu entschlüsseln. Der öffentliche
% Schlüssel wird an alle Benutzer verteilt, während der private Schlüssel nur dem Benutzer bekannt ist.

%#TODO: DH bzw ECDH muss in den Grundlagen erklärt werden

% Für die Verschlüsselung wird der Elliptic Curve Diffie-Hellman-Schlüsselaustausch verwendet.
% Der Elliptic Curve Diffie-Hellman-Schlüsselaustausch ist ein Schlüsselaustauschprotokoll, das auf dem
% Diffie-Hellman-Schlüsselaustausch basiert. Der Diffie-Hellman-Schlüsselaustausch ist ein Schlüsselaustauschprotokoll,
% das verwendet wird, um einen gemeinsamen geheimen Schlüssel zwischen zwei Parteien zu erzeugen. Der
% gemeinsame geheime Schlüssel wird verwendet, um die Nachrichten zu verschlüsseln und zu entschlüsseln.
% Der Elliptic Curve Diffie-Hellman-Schlüsselaustausch verwendet elliptische Kurven, um die Sicherheit des
% Schlüsselaustauschs zu erhöhen. Die elliptische Kurve ist eine mathematische Funktion, die verwendet wird,
% um die Punkte zu berechnen, die für die Verschlüsselung und Entschlüsselung der Nachrichten verwendet werden.
% Der Elliptic Curve Diffie-Hellman-Schlüsselaustausch ist sicher, da es praktisch unmöglich ist, den
% gemeinsamen geheimen Schlüssel aus den öffentlichen Schlüsseln zu berechnen.