\subsection{Registrierung}

% In meiner Bachelorarbeit entwickle ich ein Peer-to-Peer-Instant-Messaging-Protokoll. 
% Um für eine gewisse Sicherheit zu sorgen, möchte ich neuen Usern eine ID und einen Public Key zuordnen 
% und diese mittels Smart Contract auf die Ethereum-Blockchain schreiben.

% Benutzername MUSS eindeutig sein, denn sonst kann der Benutzer von anderen nicht gefunden werden und die
% User Experience beim Suchen von Teilnehmern leidet massiv.

Der erste Schritt ist die Registrierung eines neuen Benutzers. Hierfür muss der Benutzer einen Benutzernamen
und ein Passwort angeben. Der Benutzername muss eindeutig sein, da er das Identifikationsmerkmal 
des Benutzers gegenüber sich selbst und anderen Teilnehmern darstellt.
Das Passwort muss mindestens zehn Zeichen lang sein und mindestens einen Großbuchstaben, einen Kleinbuchstaben,
eine Zahl und ein Sonderzeichen enthalten. Diese Anforderungen sollen sicherstellen, dass das Passwort
nicht leicht zu erraten ist und somit die Sicherheit des Benutzers gewährleistet ist.
\textcolor{red}{Nachdem der Benutzername und das Passwort eingegeben wurden, wird ein Schlüsselpaar auf 
Basis des Passworts erzeugt und der öffentliche Schlüssel wird zusammen mit dem
Benutzernamen auf die Blockchain geschrieben. -> Vielleicht das Passwort nicht direkt verwenden, sondern
einen Hashwert davon?}
% #TODO: mit welchem Algorithmus? -> Argon2 https://cryptobook.nakov.com/mac-and-key-derivation/argon2


% #TODO: könnte man auch einen Passkey verwenden? Wie funktionieren die?
