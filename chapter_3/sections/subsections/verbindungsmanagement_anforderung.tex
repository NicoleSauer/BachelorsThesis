\subsection{Verbindungsmanagement}
\label{subsec:verbindungsmanagement_req}

Das Verbindungsmanagement in einem Peer-to-Peer Instant-Messaging-Protokoll umfasst verschiedene Aspekte, die für eine zuverlässige, stabile und sichere Kommunikation zwischen den Peers essentiell sind.

Zunächst spielt der Verbindungsaufbau eine wichtige Rolle. Dieser Mechanismus ermöglicht es den Peers, miteinander in Verbindung zu treten. Durch die Verwendung von IP-Adressen und Ports oder anderen Identifikationsmechanismen wird sichergestellt, dass die Kommunikation initiiert werden kann. Dieser Prozess sollte sicher und authentifiziert ablaufen, um die Integrität des Netzwerks zu gewährleisten. Die Überwachung der Verbindungsstabilität ist ein weiterer wichtiger Aspekt des Verbindungsmanagements. Durch regelmäßige Überprüfungen sollte sichergestellt werden, dass die Verbindung zwischen den Peers aktiv bleibt und eventuelle Probleme frühzeitig erkannt werden können. Ein weiterer Schlüsselaspekt ist die Verbindungsbeendigung. Ein ordnungsgemäßer Mechanismus zur Beendigung von Verbindungen ist wichtig, um Ressourcen freizugeben und mögliche Sicherheitsrisiken zu minimieren. Die Verbindungsbeendigung kann durch Benutzeraktionen wie das Abmelden ausgelöst werden oder aufgrund von Fehlern im Netzwerk auftreten. Im Falle vorübergehender Unterbrechungen, beispielsweise durch Netzwerkausfälle, sollte das Verbindungsmanagement Mechanismen zur automatischen Wiederherstellung von Verbindungen bereitstellen. Die Verbindungsauthentifizierung hingegen stellt sicher, dass die Kommunikation nur zwischen vertrauenswürdigen Parteien stattfindet. Dieser Sicherheitsaspekt ist entscheidend, um unautorisierte Zugriffe zu verhindern und die Vertraulichkeit der übertragenen Daten zu gewährleisten.