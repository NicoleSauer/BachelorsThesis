\subsection{Plattformunabhängigkeit}

Das Protokoll sollte auf verschiedenen Betriebssystemen und Gerätetypen nahtlos funktionieren. Um Plattformunabhängigkeit zu gewährleisten, sollte das Protokoll auf offenen, standardisierten Technologien basieren. Hierzu gehören beispielsweise Netzwerkprotokolle wie TCP, UDP oder IP. Die Verwendung plattformübergreifender Standards stellt sicher, dass die Kernfunktionalitäten des Protokolls von den meisten Betriebssystemen unterstützt werden. Die Implementierung der Protokollspezifikationen sollte in verschiedenen Programmiersprachen möglich sein. Dies ermöglicht es, das Protokoll in verschiedenen Anwendungen zu verwenden. Diese Unabhängigkeit ist ein wichtiger Aspekt, um die Verbreitung des Protokolls zu fördern und die Interoperabilität zwischen verschiedenen Anwendungen zu gewährleisten.