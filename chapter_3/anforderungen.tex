\chapter{Anforderungsanalyse}
\label{chap:anforderungsanalyse}

% Lastenheft (auch Anforderungs- oder Kundespezifikation):
% - enthält eher abstrakte, eher allgemeine Festlegungen der gewünschten Dienste
% - fundamentale Eigenschaften des Produktes
% - Beschreibung des "was" und nicht des "wie"
% - hat den Sinn, der Lösung nicht vorzugreifen (Art der Umsetzung ist hier nicht relevant)
% - sobald das Pflichtenheft existiert, ist das Lastenheft nicht mehr relevant und muss nicht weiter
%   gepflegt werden

% Pflichtenheft:
% - Entwickler beschreibt wie er die im Lastenheft dargelegten Anforderungen zu erfüllen gedenkt
% - Beschreibung bis in das kleinste Detail, wie sich die Software unter bestimmten Bedingungen verhalten soll
% - muss stets aktuell gehalten werden

Die Anforderungsanalyse dient dazu, die Grundlage für den erfolgreichen Verlauf eines Softwareprojekts zu schaffen, indem sie sicherstellt, dass die Ziele und Anforderungen des Projekts klar definiert und verstanden werden \parencite{Zakharyan_SoftwareRequirementsForMessagingService}. In dieser Arbeit wird ein Prototyp für ein Peer-to-Peer-Instant-Messaging-Protokoll entwickelt. Das Ziel dieser Arbeit ist es, die Machbarkeit eines solchen Protokolls aufzuzeigen. Um dieses Ziel zu erreichen, müssen die Anforderungen an das Protokoll klar definiert werden. Dazu werden in diesem Kapitel die funktionalen und nicht-funktionalen Anforderungen an das Protokoll beschrieben. 



\section{Funktionale Anforderungen}


%Dies sind Aussagen zu den Diensten, die das System leisten sollte, zur Reaktion des Systems auf bestimmte 
%Eingaben und zum Verhalten des Systems in bestimmten Situationen. 

Funktionale Anforderungen beziehen sich auf die spezifischen Funktionen und Aufgaben, 
die eine Software oder ein System erfüllen muss, um die Bedürfnisse und Erwartungen der Benutzer zu erfüllen.
Sie beschreiben, was das System tun soll, welche Aktionen es ausführen muss und welche 
Ergebnisse es liefern sollte. Diese Anforderungen sind entscheidend, um sicherzustellen, dass die entwickelte Software 
oder wie in diesem Fall, das entwickelte Protokoll die erwarteten Funktionen erbringt. Sie dienen 
als Grundlage für das Design, die Entwicklung, die Validierung und die Verifizierung von Software-Systemen und 
sind ein wichtiger Bestandteil des Anforderungsmanagements im Software-Engineering-Prozess.
% #TODO: find and add source 
\\

\noindent Folgende Funktionen soll das Protokoll bieten:

\begin{itemize}
    \item Registrierung des Nutzers
    \item Authentifizierung des Nutzers
    \item Verschlüsselung der Kommunikation
    \item Versenden und Empfangen von Textnachrichten
    \item Kontaktmanagement (Kontakt hinzufügen, löschen, blockieren)
    \item Handhabung von Nachrichten, wenn der Empfänger offline ist
\end{itemize}


%\subsection{Benutzeranforderungen und -erwartungen}

Benutzer erwarten die folgenden Funktionen...
%\subsection{Sicherheitsanforderungen an das Protokoll}

Sicherheitsanforderungen sind ...
\section{Nicht-funktionale Anforderungen}

%Dies sind Beschränkungen der durch das System angebotenen Dienste oder Funktionen. Das schließt 
%Zeitbeschränkungen, Beschränkungen des Entwicklungsprozesses und einzuhaltende Standards ein.

Nicht-funktionale Anforderungen sind Anforderungen, die sich nicht auf die 
spezifische Funktionalität einer Software-Anwendung beziehen, sondern auf andere Qualitätsmerkmale 
und Aspekte, die die Leistung, Zuverlässigkeit und Benutzerfreundlichkeit der Software betreffen. 
Diese Anforderungen beschreiben, wie die Software funktionieren sollte, anstatt was sie tun sollte. 
Nicht-funktionale Anforderungen sind genauso wichtig wie funktionale Anforderungen, da sie einen 
erheblichen Einfluss auf die Gesamtleistung und die Benutzerzufriedenheit haben können.
% #TODO: find and add source 
\\
Die folgenden nicht-funktionale Anforderungen soll das Protokoll bieten:

\begin{itemize}
    \item Performanz
    \item Sicherheit
    \item Zuverlässigkeit
    \item Kompatibilität
    %\item Interoperabilität
    \item Privatsphäre
    \item Skalierbarkeit
\end{itemize}