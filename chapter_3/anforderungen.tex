\chapter{Anforderungsanalyse}

% Lastenheft (auch Anforderungs- oder Kundespezifikation):
% - enthält eher abstrakte, eher allgemeine Festlegungen der gewünschten Dienste
% - fundamentale Eigenschaften des Produktes
% - Beschreibung des "was" und nicht des "wie"
% - hat den Sinn, der Lösung nicht vorzugreifen (Art der Umsetzung ist hier nicht relevant)
% - sobald das Pflichtenheft existiert, ist das Lastenheft nicht mehr relevant und muss nicht weiter
%   gepflegt werden

% Pflichtenheft:
% - Entwickler beschreibt wie er die im Lastenheft dargelegten Anforderungen zu erfüllen gedenkt
% - Beschreibung bis in das kleinste Detail, wie sich die Software unter bestimmten Bedingungen verhalten soll
% - muss stets aktuell gehalten werden

Die Anforderungsanalyse dient dazu, die Grundlage für den erfolgreichen Verlauf eines Softwareprojekts 
zu schaffen, indem sie sicherstellt, dass die Ziele und Anforderungen des Projekts klar definiert 
und verstanden werden. Dies reduziert die Wahrscheinlichkeit von Missverständnissen, Fehlern und 
Änderungen im späteren Verlauf des Projekts, was Zeit und Ressourcen spart und die Erfolgschancen erhöht.
Die Zielgruppe für dieses Protokoll sollen Privatnutzer sein, woraus sich die nachstehenden funktionalen 
und nicht-funktionalen Anforderungen ergeben:
% #TODO: Thies will Quellen dafür, dass man eine Anforderungsanalyse macht!


\section{Funktionale Anforderungen}


% Dies sind Aussagen zu den Diensten, die das System leisten sollte, zur Reaktion des Systems auf bestimmte 
% Eingaben und zum Verhalten des Systems in bestimmten Situationen. 

Funktionale Anforderungen beziehen sich auf die spezifischen Funktionen und Aufgaben, die eine Software oder ein System erfüllen muss, um die Bedürfnisse und Erwartungen der Benutzer zu erfüllen. Sie beschreiben, was das System tun soll, welche Aktionen es ausführen muss und welche Ergebnisse es liefern sollte. Mitunter werden sie auch dazu verwendet, um festzuhalten, was das System nicht können soll. All diese Anforderungen sind entscheidend, um sicherzustellen, dass die entwickelte Software oder wie in diesem Fall, das entwickelte Protokoll die erwarteten Funktionen erbringt. Sie dienen als Grundlage für das Design, die Entwicklung, die Validierung und die Verifizierung von Software-Systemen und sind ein wichtiger Bestandteil des Anforderungsmanagements im Software-Engineering-Prozess \parencite[S. 124-126]{Sommerville_AnfAnalyse}.

Um die Anforderungen an das zu entwickelnde Protokoll zu definieren, wird es in die folgenden
Funktionen unterteilt:

\begin{itemize}
    \item Peer-Discovery und Routing
    \item Verbindungsmanagement
    \item Nachrichtenformatierung
    \item Sicherheit und Verschlüsselung
    \item Plattformunabhängigkeit
    \item Fehlerbehandlung und Wiederholungsmechanismen
\end{itemize}

\noindent Die folgenden Abschnitte beschreiben die funktionalen Anforderungen an das Protokoll.

\subsection{Peer-Discovery und Routing}
\label{subsec:peer_discovery}

Das Protokoll muss es den Benutzern ermöglichen, sich gegenseitig zu finden, um miteinander kommunizieren zu können. Dazu muss es eine Möglichkeit geben, die IP-Adresse eines Benutzers zu ermitteln, wenn der Benutzername des Ziels bekannt ist. Wenn die IP-Adresse eines Benutzers bekannt ist, muss das Protokoll in der Lage sein, eine Verbindung zu diesem Benutzer herzustellen. Dies ist notwendig, um die Dezentralität des Protokolls zu gewährleisten. Sollte ein Art von zentralem Server benötigt werden, muss die Verschlüsselung der Nachrichten so implementiert werden, dass dem Server nicht vertraut werden muss und dieser damit nicht in der Lage ist, die Nachrichten zu entschlüsseln und zu lesen.
\subsection{Verbindungsmanagement}
\label{subsec:verbindungsmanagement_req}

Das Verbindungsmanagement in einem Peer-to-Peer Instant-Messaging-Protokoll umfasst verschiedene Aspekte, die für eine zuverlässige, stabile und sichere Kommunikation zwischen den Peers essentiell sind.

Zunächst spielt der Verbindungsaufbau eine wichtige Rolle. Dieser Mechanismus ermöglicht es den Peers, miteinander in Verbindung zu treten. Durch die Verwendung von IP-Adressen und Ports oder anderen Identifikationsmechanismen wird sichergestellt, dass die Kommunikation initiiert werden kann. Dieser Prozess sollte sicher und authentifiziert ablaufen, um die Integrität des Netzwerks zu gewährleisten. Die Überwachung der Verbindungsstabilität ist ein weiterer wichtiger Aspekt des Verbindungsmanagements. Durch regelmäßige Überprüfungen sollte sichergestellt werden, dass die Verbindung zwischen den Peers aktiv bleibt und eventuelle Probleme frühzeitig erkannt werden können. Ein weiterer Schlüsselaspekt ist die Verbindungsbeendigung. Ein ordnungsgemäßer Mechanismus zur Beendigung von Verbindungen ist wichtig, um Ressourcen freizugeben und mögliche Sicherheitsrisiken zu minimieren. Die Verbindungsbeendigung kann durch Benutzeraktionen wie das Abmelden ausgelöst werden oder aufgrund von Fehlern im Netzwerk auftreten. Im Falle vorübergehender Unterbrechungen, beispielsweise durch Netzwerkausfälle, sollte das Verbindungsmanagement Mechanismen zur automatischen Wiederherstellung von Verbindungen bereitstellen. Die Verbindungsauthentifizierung hingegen stellt sicher, dass die Kommunikation nur zwischen vertrauenswürdigen Parteien stattfindet. Dieser Sicherheitsaspekt ist entscheidend, um unautorisierte Zugriffe zu verhindern und die Vertraulichkeit der übertragenen Daten zu gewährleisten.
\subsection{Nachrichtenformatierung}

Die Nachrichtenformatierung ist ein wichtiger Aspekt eines Instant-Messaging-Protokolls. Sie definiert, wie die Nachrichten strukturiert sind und welche Informationen sie enthalten. Die Nachrichtenformatierung ist entscheidend für die Funktionalität des Protokolls, da sie die Grundlage für die Kommunikation zwischen den Peers bildet. Die Nachrichtenformatierung muss so gestaltet sein, dass sie die folgenden Anforderungen erfüllt:

\begin{itemize}
    \item Die Nachrichten müssen in einem standardisierten Format vorliegen, um die Interoperabilität zwischen den verschiedenen Implementierungen des Protokolls zu gewährleisten.
    \item Die Nachrichten müssen die erforderlichen Informationen enthalten, um die Kommunikation zwischen den Peers zu ermöglichen.
    \item Die Nachrichten müssen so strukturiert sein, dass sie von den Peers verarbeitet werden können.
\end{itemize}
\subsection{Verschlüsselung der Kommunikation}

% Wenn Public Key des anderen bekannt ist, gilt das als authentifizierter Schlüsselaustausch. Bietet Schutz 
% vor Man in the Middle Attacken, da bekannt ist, wem der Public Key gehört. Da in meinem Fall beide
% authentifiziert sind, ist es ein wechselseitiger authentifizierter Schlüsselaustausch.

Für die Verschlüsselung der Kommunikation soll das Public-Key-Verfahren verwendet werden.
Das Public-Key-Verfahren ist ein asymmetrisches Verschlüsselungsverfahren, das zwei Schlüssel verwendet,
einen öffentlichen und einen privaten Schlüssel. Das soll gewährleisten, dass die Nachrichten nur von
dem Benutzer gelesen werden können, für den sie bestimmt sind und erkannt werden kann, ob die Nachricht
manipuliert wurde. Zudem erhöht es die Sicherheit des Schlüsselaustauschs, da es praktisch unmöglich ist,
den privaten Schlüssel aus dem öffentlichen Schlüssel zu berechnen.


% Der öffentliche Schlüssel wird verwendet, um Nachrichten
% zu verschlüsseln, und der private Schlüssel wird verwendet, um Nachrichten zu entschlüsseln. Der öffentliche
% Schlüssel wird an alle Benutzer verteilt, während der private Schlüssel nur dem Benutzer bekannt ist.

%#TODO: DH bzw ECDH muss in den Grundlagen erklärt werden

% Für die Verschlüsselung wird der Elliptic Curve Diffie-Hellman-Schlüsselaustausch verwendet.
% Der Elliptic Curve Diffie-Hellman-Schlüsselaustausch ist ein Schlüsselaustauschprotokoll, das auf dem
% Diffie-Hellman-Schlüsselaustausch basiert. Der Diffie-Hellman-Schlüsselaustausch ist ein Schlüsselaustauschprotokoll,
% das verwendet wird, um einen gemeinsamen geheimen Schlüssel zwischen zwei Parteien zu erzeugen. Der
% gemeinsame geheime Schlüssel wird verwendet, um die Nachrichten zu verschlüsseln und zu entschlüsseln.
% Der Elliptic Curve Diffie-Hellman-Schlüsselaustausch verwendet elliptische Kurven, um die Sicherheit des
% Schlüsselaustauschs zu erhöhen. Die elliptische Kurve ist eine mathematische Funktion, die verwendet wird,
% um die Punkte zu berechnen, die für die Verschlüsselung und Entschlüsselung der Nachrichten verwendet werden.
% Der Elliptic Curve Diffie-Hellman-Schlüsselaustausch ist sicher, da es praktisch unmöglich ist, den
% gemeinsamen geheimen Schlüssel aus den öffentlichen Schlüsseln zu berechnen.
\subsection{Plattformunabhängigkeit}

Das Protokoll sollte auf verschiedenen Betriebssystemen und Gerätetypen nahtlos funktionieren. Um Plattformunabhängigkeit zu gewährleisten, sollte das Protokoll auf offenen, standardisierten Technologien basieren. Hierzu gehören beispielsweise Netzwerkprotokolle wie TCP, UDP oder IP. Die Verwendung plattformübergreifender Standards stellt sicher, dass die Kernfunktionalitäten des Protokolls von den meisten Betriebssystemen unterstützt werden. Die Implementierung der Protokollspezifikationen sollte in verschiedenen Programmiersprachen möglich sein. Dies ermöglicht es, das Protokoll in verschiedenen Anwendungen zu verwenden. Diese Unabhängigkeit ist ein wichtiger Aspekt, um die Verbreitung des Protokolls zu fördern und die Interoperabilität zwischen verschiedenen Anwendungen zu gewährleisten.
\subsection{Fehlerbehandlung und Wiederholungsmechanismen}

Die Fehlerbehandlung und die Implementierung von Wiederholungsmechanismen stellen entscheidende Komponenten im Verlauf einer Peer-to-Peer Instant-Messaging-Kommuni-\\kation dar. Ein zentraler Aspekt der Fehlerbehandlung ist die Erkennung von Übertragungsfehlern. Das Protokoll sollte in der Lage sein, Fehlerzustände während des Nachrichtenaustauschs zu identifizieren. Dies können beispielsweise fehlerhafte Pakete, verlorene Verbindungen oder andere unvorhergesehene Probleme sein. Die Fehlererkennung ermöglicht es, schnell auf Probleme zu reagieren und entsprechende Maßnahmen einzuleiten. Die Wiederholungsmechanismen sind eng mit der Fehlerbehandlung verbunden und dienen dazu, sicherzustellen, dass fehlgeschlagene Übertragungen erneut versucht werden. Dies könnte durch automatisches erneutes Senden von Nachrichten oder das Auslösen spezifischer Protokollmechanismen geschehen. Die Wiederholungsmechanismen sind darauf ausgerichtet, die Zustellung von Nachrichten trotz vorübergehender Probleme im Netzwerk zu gewährleisten.
 

%\subsection{Benutzeranforderungen und -erwartungen}

Benutzer erwarten die folgenden Funktionen...
%\subsection{Sicherheitsanforderungen an das Protokoll}

Sicherheitsanforderungen sind ...
\section{Nicht-funktionale Anforderungen}

%Dies sind Beschränkungen der durch das System angebotenen Dienste oder Funktionen. Das schließt 
%Zeitbeschränkungen, Beschränkungen des Entwicklungsprozesses und einzuhaltende Standards ein.

Nicht-funktionale Anforderungen sind Anforderungen, die sich nicht auf eine 
spezifische Funktionalität einer Software-Anwendung beziehen, sondern auf die Gesamtstruktur und damit auf
Qualitätsmerkmale und Aspekte, wie beispielsweise die Leistung, Zuverlässigkeit, Antwortzeit und Benutzerfreundlichkeit 
der Software. 
Diese Anforderungen beschreiben, \textit{wie} die Software funktionieren sollte, anstatt \textit{was} sie tun sollte. 
Nicht-funktionale Anforderungen sind genauso wichtig wie funktionale Anforderungen, da sie einen 
erheblichen Einfluss auf die Gesamtleistung und die Benutzerzufriedenheit haben können
\parencite[S. 126-130]{Sommerville_AnfAnalyse}.

\noindent Die folgenden nicht-funktionale Anforderungen soll das Protokoll bieten:

\begin{itemize}
    \item Performanz
    \item Sicherheit
    \item Zuverlässigkeit
    \item Kompatibilität
    %\item Interoperabilität
    \item Privatsphäre
    \item Skalierbarkeit
\end{itemize}

\noindent Die folgenden Abschnitte beschreiben die nicht-funktionalen Anforderungen an das Protokoll.

\subsection{Performanz}

Die Leistungsanforderungen an ein Peer-to-Peer Instant-Messaging-Protokoll beziehen sich auf die Effizienz und Geschwindigkeit der Nachrichtenübertragung, um sicherzustellen, dass Benutzer eine reaktionsschnelle und nahtlose Kommunikation erfahren. Die Leistungsfähigkeit des Protokolls beeinflusst direkt die Benutzerzufriedenheit und die Gesamterfahrung der Instant-Messaging-Anwendung, die das Protokoll verwendet. 


